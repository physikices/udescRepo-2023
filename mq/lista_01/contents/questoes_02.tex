% --------------------------------------------- %
% Q01
% --------------------------------------------- %
 \addcontentsline{toc}{section}{Problema 01}
 \begin{prob}
	Considere os vetores de estados
	\begin{align} 
		|\psi\rangle &=
		\begin{pmatrix}
			5i\\
			2\\
			-1
		\end{pmatrix}\condition{e $
			\displaystyle{
				|\phi\rangle =
				  \begin{pmatrix}
						3\\
						8i\\
						-9i									 
				\end{pmatrix}}
		$} 
	\end{align}
	\begin{enumerate}[label=\alph *)]
		\item Estes vetores estão normalizados? Se não, normalize-os.
		\item Estes vetores são ortogonais?
	\end{enumerate}

	\begin{sol}

	\end{sol}		
 \end{prob}
	% --------------------------------------------- %

	% --------------------------------------------- %
	% Q02
	% --------------------------------------------- %
	\addcontentsline{toc}{section}{Problema 02}
	\begin{prob}
		Mostre que os operadores $\hat{p}$ e $\hat{p}^{2}$ são hermitianos. Lembrando que:
		\begin{align}
			\hat{p}&=-i \hbar \frac{d}{dx}
		\end{align}
	\end{prob}
	% --------------------------------------------- %

	% --------------------------------------------- %
	% Q03
	% --------------------------------------------- %
	\addcontentsline{toc}{section}{Problema 03}
	\begin{prob}
		Os autoestados do poço infinito são autoestados do momento? Justifique.
		\par\noindent\textbf{Sugestão:} Teste se a equação atuação do operador  momento  nos autoestados do poço infinito $\hat{p} \psi_{n}$ geram uma equação de autovalores. Se gerarem uma equação de autovalores, então $\psi_{n}$ (que são autoestados do Hamiltoniano) serão também  autoestados do momento.
	\end{prob}
	% --------------------------------------------- %

	% --------------------------------------------- %
	% Q04
	% --------------------------------------------- %
	\addcontentsline{toc}{section}{Problema 04}
	\begin{prob}
		\textbf{(Problema 3.11 do Griffiths)} Encontre a função de onda $\Phi(p,t)$, para uma partícula no estado fundamental  do hoscilador harmômico. Qual é a probabilidade (com 2 algarismos significativos) de que uma medida do momento $p$ de uma partícula  neste estado produza um valor fora do range clássico.
		\par\noindent\textbf{Sugestão 01:} Procure numa tabela matemática por uma "Distribuição Normal" ou "Função Erro" ou calcule com algum programa (wxMaxima, Mathematica, etc...) ou ainda consulte as notas da aula 08 (em particular o exercício feito no final desta aula).
		\par\noindent\textbf{Sugestão 02:} Estude primeiramente o exemplo 3.4 do Griffiths (página 108).
	\end{prob}
	% --------------------------------------------- %

	% --------------------------------------------- %
	% Q05
	% --------------------------------------------- %
	\addcontentsline{toc}{section}{Problema 05}
	\begin{prob}
		Na lista 1 foi mostrado que $\displaystyle{\left[\hat{p}, f(x)\right]=-i \hbar \frac{\partial f(x)}{\partial x }}$. Utilizando este resultado, obtenha a relação de incerteza entre o operador hamiltoniano do oscilador harmônico
		\begin{align}
			\hat{H}&=\frac{\hat{p}^{2}}{2m}+\frac{m \omega^{2}x^{2}}{2}
		\end{align}
		e o operador $(\sigma_{H} \sigma_{p})$. Comente se os operadores hamiltoniano e momento são compatíveis.
	\end{prob}
	% --------------------------------------------- %

	% --------------------------------------------- %
	% Q06
	% --------------------------------------------- %
	\addcontentsline{toc}{section}{Problema 06}
	\begin{prob}
		Um hamiltoniano de um certo sistema de 2 níveis é dado por
		\begin{align}
			\hat{H}&=\varepsilon \left(|1\rangle\langle 1|-|2\rangle\langle  2|+|1\rangle\langle  2|+|2\rangle\langle  1|\right)
		\end{align}
		onde $|1\rangle$, $\langle  2|$ é uma base ortonormal e $\varepsilon$ é um número com dimensão de energia.

		\begin{enumerate}[label=\alph *)]
				\item Encontre a matriz que representa este Hamiltoniano (sugestão: encontre os elementos de matriz fazendo os sanduíches do operador).
				\item Calcule seus autovalores e autovetores normalizados.
		\end{enumerate}
	\end{prob}
	% --------------------------------------------- %

	% --------------------------------------------- %
	% Q07
	% --------------------------------------------- %
	\addcontentsline{toc}{section}{Problema 07}
	\begin{prob}
		Um operador $\hat{A}$, representando um observável $A$, têm dois autoestados normalizados $\psi_{1}$ e $\psi_{2}$, com autovalores $a_{1}$ e $a_{2}$ respectivamente. O operador $\hat{B}$, representando o observável $B$, têm dois autoestados normalizados $\phi_{1}$ e $\phi_{2}$, com autovalores $b_{1}$ e $b_{2}$, respectivamente. Os autoestados se relacionam por:
		\begin{align}
			|\psi_{1}\rangle &= \frac{1}{5}\left(3|\psi_{1}\rangle+4|\psi_{2}\rangle\right)\condition{e $|\psi_{2}\rangle=\frac{1}{5}\left(4|\psi_{1}\rangle-3|\psi_{2}\rangle\right)$}  
		\end{align}
		\begin{enumerate}[label=\alph *)]
				\item O observável $A$ é medido e o valor $a_{2}$ é obtido. Qual é o estado do sistema imediatamente após esta medida?
				\item Se $B$ é medido em seguida, quais são os possíveis resultados e quais são suas possibilidades?
				\item Logo após a medida de $B$, $A$ é medido novamente. Qual a  é a probabilidade de obter $a_{1}$ novamente? \textbf{Cuidado:} considerar o problema completo, onde (item a) medida $a_{2}$, logo após medimos um  dos estados de $B$ (item b), e por fim  medimos $a_{1}$.
		\end{enumerate}
	\end{prob}

	% --------------------------------------------- %

	% --------------------------------------------- %
	% Q08
	% --------------------------------------------- %
	\addcontentsline{toc}{section}{Problema 08}
	\begin{prob}
		Considere um sistema de dois níveis os kets $|\alpha_{1}\rangle$ e $|\alpha_{2}\rangle$ formam uma base ortonormal. Uma nova base $|\beta_{1}\rangle$ e $|\beta_{2}\rangle$ se relacionam com a antiga por:
		\begin{align}
			|\beta_{1}\rangle &= \frac{1}{\sqrt{2}}\left(|\alpha_{1}\rangle+|\alpha_{2}\rangle\right)\condition{e $\displaystyle\frac{1}{\sqrt{2}}\left(|\alpha_{1}\rangle-|\alpha_{2}\rangle\right)$}
		\end{align}
		Um operador $\hat{P}$ é representado na base $|\alpha_{1}\rangle$ e $|\alpha_{2}\rangle$ pela matriz
		\begin{align}
			P &= 
			\begin{pmatrix}
				1 & \epsilon \\
				\epsilon & 1
			\end{pmatrix}
			=
			\begin{pmatrix}
			\langle \alpha_{1}|\hat{P}|\alpha_{1}\rangle & \langle \alpha_{1}|\hat{P}|\alpha_{2}\rangle \\
			\langle \alpha_{2}|\hat{P}|\alpha_{1}\rangle & \langle \alpha_{2}|\hat{P}|\alpha_{2}\rangle
			\end{pmatrix}
		\end{align}
		\begin{enumerate}[label=\alph *)]
			\item Encontre os autovalores deste operador e escreva-o na forma diagonal.
			\item Encontre a representação de $\hat{P}$ na base $|\beta_{1}\rangle$ e $|\beta_{2}\rangle$.
		\end{enumerate}

		\textbf{Sugestão:}

		Escreva $\hat{P}$ na forma matricial abaixo
		\begin{align}
			P &=
			\begin{pmatrix}
				\langle \beta_{1}|\hat{P}|\beta_{1}\rangle & \langle \beta_{1}|\hat{P}|\beta_{2}\rangle \\
				\langle \beta_{2}|\hat{P}|\beta_{1}\rangle & \langle \beta_{2}|\hat{P}|\beta_{2}\rangle \\
			\end{pmatrix}
			=
			\begin{pmatrix}
				\langle \beta_{1}|\mathbb{\hat{1}}\hat{P}\mathbb{\hat{1}}|\beta_{1}\rangle & \langle \beta_{1}|\mathbb{\hat{1}}\hat{P}\mathbb{\hat{1}}|\beta_{2}\rangle \\
				\langle \beta_{2}|\mathbb{\hat{1}}\hat{P}\mathbb{\hat{1}}|\beta_{\hat{1}}\rangle & \langle \beta_{2}|\mathbb{\hat{1}}\hat{P}\mathbb{\hat{1}}|\beta_{2}\rangle \\
			\end{pmatrix}
		\end{align}

		onde a identidade $\mathbb{\hat{1}}$ na base $\alpha_{1}$ e $\alpha_{2}$ é dada por

		\begin{align}
			\mathbb{\hat{1}} &= \sum_{i=1}^{2}|\alpha_{i}\rangle\langle \alpha_{i}|=|\alpha_{1}\rangle\langle \alpha_{1}|+|\alpha_{2}\rangle\langle \alpha_{2}|
		\end{align}
	\end{prob}
	% --------------------------------------------- %
