% --------------------------------------------- %
% Q01
% --------------------------------------------- %
 \addcontentsline{toc}{section}{Problema 01}
 \begin{prob}
	 Encontre as autofunções $(\psi_{n_{x}n_{y}n_{z}}(x,y,z))$ e os auto valores de energia para o oscilador harmônico quântico isotrópico 3D:
	 \begin{align}
		 V(x,y,z) &= \frac{1}{2}m \omega^{2}x^{2}+\frac{1}{2}m \omega^{2}y^{2}+\frac{1}{2}m \omega^{2}z^{2}
	 \end{align}
	 Para isso,utilize os resultados do oscilador quântico 1D, estudado no módulo I \textit{(não é necessário resolver quase nada!)}.

	 \noindent \textbf{Obs.: As autofunções podem ser escritas em termos de uma constante de normalização $A_{n_{x}n_{y}n_{z}}$ Não é necessário encontrar essa constante!}
	 \begin{sol}
		 Escrevendo o operador Hamiltoniano para o oscilador 3D, em termo do operador momento $\vec{p}=-i \hbar\nabla$
		 \begin{align}
			 H_{x,y,z} &= \frac{\vec{p}\;^{2}}{2m}+\frac{1}{2}m \omega^{2}\left(x^{2}+y^{2}+z^{2}\right) = H_{x}+H_{y}+H_{z}
		 \end{align}
		 Uma vez que o Hamiltoniano é separável, podemos escrevê-lo em termos de um sistema de três Hamiltonianos unidimensionais correspondendo ao movimento nas direções $x$, $y$ e $z$, além disso, considerando a equação de autovalor para o operador Hamiltoniano
		 \begin{align}
			 H_{x,y,z} \psi_{n_{x,y,z}}(x,y,z) &= E_{n_{x,y,z}} \psi_{n_{x,y,z}}(x,y,z)\condition{com $\displaystyle{E_{n_{i}}=\sum_{i=1}^{3}\left(n_{i}+\frac{1}{2}\right)\hbar \omega}$}
		 \end{align}
		 tem-se então que
		 \begin{subequations}
			 \begin{align}
				 H_{x} \psi_{n_{x}(x)} &= \frac{p_{x}^{2}}{2m}+\frac{1}{2}m \omega x^{2}=\left(n_{x}+\frac{1}{2}\right)\hbar \omega \label{eq:ohq-1}\\
				 H_{y} \psi_{n_{y}(y)} &= \frac{p_{y}^{2}}{2m}+\frac{1}{2}m \omega y^{2}=\left(n_{y}+\frac{1}{2}\right)\hbar \omega\label{eq:ohq-2}\\
				 H_{z} \psi_{n_{z}(z)} &= \frac{p_{z}^{2}}{2m}+\frac{1}{2}m \omega z^{2}=\left(n_{z}+\frac{1}{2}\right)\hbar \omega\label{eq:ohq-3}
			 \end{align}
		 \end{subequations}
		 Os autovalores de energia sai direto de
		 \begin{align}
			 \begin{split}
				 E_{n_{x,y,z}} &= \left(n_{x}+\frac{1}{2}\right)\hbar \omega+\left(n_{y}+\frac{1}{2}\right)\hbar \omega+\left(n_{z}+\frac{1}{2}\right)\hbar \omega\\
											 &=\boxed{\left(n_{x}+n_{y}+n_{z}+\frac{3}{2}\right)\hbar \omega\condition{com $\displaystyle{n_{x,y,z}=0,1,2,...}$}}
			 \end{split}
		 \end{align}
		 Para as auto funções, tem-se que
		 \begin{align}
			 \label{eq:p1-a}
			 \psi_{n_{x,y,z}(x,,y,z)}=\psi_{n_{x}}(x)\psi_{n_{y}}(y)\psi_{n_{z}}(z)
		 \end{align}
		 em que $\psi_{n_{i}}$ é a \textit{i-ésima} autofunção do oscilado harmômico unidimensional, por exemplo, a solução geral da eq. \eqref{eq:ohq-1} é dada por
		 \begin{align}
			 \label{eq:p1-b}
			 \psi_{n_{x}}(x) &= A_{n_{x}}\mathcal{H}_{n}(x)\mathrm{e}^{-\frac{m \omega}{2\hbar}x^{2}}\condition{com $\displaystyle{\mathcal{H}_{n}(x)=(-1)^{n}\mathrm{e}^{x^{2}}\left(\frac{d}{dx}\right)^{n}\mathrm{e}^{-x^{2}}}$}
		 \end{align}
		 \newpage
		 \noindent portanto, levando \eqref{eq:p1-b} em \eqref{eq:p1-a} obtemos
		 \begin{align}
			 \boxed{
				 \psi_{n_{x,y,z}}(x,y,z)=A_{n_{x,y,z}}\mathcal{H}_{n}(x)\mathcal{H}_{n}(y)\mathcal{H}_{n}(z)\mathrm{e}^{-\frac{m \omega}{2\hbar}(x^{2}+y^{2}+z^{2})}\condition{com $\displaystyle{n=0,1,2...}$}
			 }
		 \end{align}
	 \end{sol}
 \end{prob}
 % --------------------------------------------- %
 % --------------------------------------------- %
 % Q02
 % --------------------------------------------- %
 \addcontentsline{toc}{section}{Problema 02}
 \begin{prob}
	 Partindo da equação radial
	 \begin{align}
		 \frac{d}{dr}\left(r^{2}\frac{dR(r)}{dr}\right)-\frac{2mr^{2}}{\hbar^{2}}\left[V(r)-E\right]R(r)=l\left(l+1\right)R(r)
	 \end{align}
	 e fazendo a troca de variáveis $u(r)=rR(r)$, obtenha a equação
	 \begin{align}
		 -\frac{\hbar^{2}}{2m}\frac{d^{2}u(r)}{dr^{2}}+\left[V(r)+\frac{\hbar^{2}}{2m}\frac{l\left(l+1\right)}{r^{2}} \right]u(r)=Eu(r)
	 \end{align}
	 \begin{sol}
		 Fazendo a substituição obtem-se
		 \begin{align}
			 \label{eq:p2-a}
			 \frac{d}{dr}\left[r^{2}\frac{d}{dr}\left(\frac{u(r)}{r}\right)\right]-\frac{2mr^{2}}{\hbar^{2}}\frac{u(r)}{r}\left[V(r)-E\right] &= l \left(l+1\right)\frac{u(r)}{r}
		 \end{align}
		 Resolvendo primeiramente o termo das derivadas (\textit{apenas por comodidade ocultaremos a denpência em $r$ da função $u(r)$no tratamento a abaixo})
		 \begin{align}
			 \label{eq:p2-b}
			 \begin{split}
				 \frac{d}{dr}\left[r^{2}\frac{d}{dr}\left(\frac{u}{r}\right)\right] &= \frac{d}{dr}\left[r^{2}\left(\frac{1}{r}\frac{du}{dr}-\frac{u}{r^{2}}\right)\right]\\
																																						&= 2r \left(\frac{1}{r}\frac{du}{dr}-\frac{u}{r^{2}}\right)+r^{2}\left(-\frac{1}{r^{2}}\frac{du}{dr}+\frac{1}{r}\frac{d^{2}u}{dr^{2}}+\frac{2u}{r^{3}}-\frac{1}{r^{2}}\frac{du}{dr}\right) \\
																																						&= \cancel{2\frac{du}{dr}}-\cancel{\frac{2u}{r}}-\cancel{\frac{du}{dr}}+r \frac{d^{2}u}{dr^{2}}+\cancel{\frac{2u}{r}}-\cancel{\frac{du}{dr}}\\
																																						&= r \frac{d^{2}u}{dr^{2}}
			 \end{split}
		 \end{align}
		 substituindo \eqref{eq:p2-b} em \eqref{eq:p2-a} e multiplicando tudo por $-\hbar^{2}/2mr$, obtêm-se
		 \begin{align}
			 \begin{split}
				 -\frac{\hbar^{2}}{2mr}\left(r\frac{d^{2}u(r)}{dr^{2}}\right)+u(r)V(r)-u(r)E &=-\frac{\hbar^{2}}{2mr} l \left(l+1\right)\frac{u(r)}{r}\\
			 \end{split}
		 \end{align}
		 \begin{align}
			 \boxed{
				 -\frac{\hbar^{2}}{2m}\frac{d^{2}u(r)}{dr^{2}}+\left[V(r)+\frac{\hbar^{2}}{2m}\frac{l\left(l+1\right)}{r^{2}}\right]u(r)=u(r)E
			 }
		 \end{align}
	 \end{sol}
 \end{prob}
 % --------------------------------------------- %

 % --------------------------------------------- %
 % Q03
 % --------------------------------------------- %
 \addcontentsline{toc}{section}{Problema 03}
 \begin{prob}
	 \begin{enumerate}[label=\alph *)]
		 \item Partindo da equação para a parte radial da função de onda $R(r)$
			 \begin{align}
				 \label{eq:p3-eqRadialHidrogenio}
				 \frac{d}{dr}\left(r^{2}\frac{dr(r)}{dr}\right)-\frac{2mr^{2}}{\hbar^{2}}\left[V(r)-E\right]R(r)=l (l+1)R(r)
			 \end{align}
			 mostre que, para o caso onde $l=0$ e $E=-|E|$ (estado ligado), esta equação pode ser escrita como 
			 \begin{align}
				 \label{eq:p3-prSimplif}
				 \frac{d^{2}R(r)}{dr^{2}}+\frac{2}{r}\frac{dR(r)}{dr}-\frac{2m}{\hbar^{2}}V(r)R(r) = \frac{2m}{\hbar^{2}}\abs{E}R(r)
			 \end{align}
			 e que, para o potencial coulombiano entre o próton e o elétron, temos ainda
			 \begin{align}
				 \label{eq:p3-segSimplif}
				 \frac{d^{2}R(r)}{dr^{2}}+\frac{2}{r}\frac{dR(r)}{dr}+\frac{2}{ar}R(r) &= K^{2}R(r)
			 \end{align}
			 sendo
			 \begin{align}
				 \label{eq:p3-paramsRbohr}
				 a &= \frac{4 \pi \varepsilon_{0} \hbar^{2}}{me^{2}}\condition{e $\displaystyle{K^{2}=\frac{2m|E|}{\hbar^{2}}}$}
			 \end{align}
			 Na equação acima, $a$ é o raio de Bohr (da órbita prevista pelo modelo atômico de Bohr).
		 \item Seguindo a solução assintótica (grande $r$) discutida na \textbf{seção 10.5 do Moysés vol. 4}, obtenha a parte radial da função de onda normalizada $R(r)$ e o módulo do autovalor de energia do estado fundamental.
		 \item Seguindo a mesma referência citada acima, calcule o valor de $r$ para o qual a probabilidade radial é máxima e o valor médio de $r$. \textbf{Se necessário, utilize fórmula ou programa para a integral.}
	 \end{enumerate}

	 \begin{sol}
		 \begin{enumerate}[label=\alph *)]
			 \item Para $l=0$ e $E=-|E|$, basta substituir na equação \eqref{eq:p3-eqRadialHidrogenio}, calcular as derivadas e ajustar os termos, assim
				 \begin{align}
					 \begin{split}
						 \frac{d}{dr}\left(r^{2}\frac{dR(r)}{dr}\right)-\frac{2mr^{2}}{\hbar^{2}}\left[V(r)+\abs{E}\right]R(r) &= 0\\
						 2r \frac{dR(r)}{dr}+r^{2}\frac{d^{2}R(r)}{dr^{2}}-\frac{2mr^{2}}{\hbar^{2}}\left[V(r)+\abs{E}\right]R(r)&=0
					 \end{split}
				 \end{align}
				 \begin{align}
					 \label{eq:p3-segSimplif-2}
					 \boxed{
						 \frac{d^{2}R(r)}{dr^{2}}+\frac{2}{r}\frac{dR(r)}{dr}-\frac{2m}{\hbar^{2}}V(r)R(r) = \frac{2m}{\hbar^{2}}\abs{E}R(r)
					 }
				 \end{align}
				 Sendo o potencial coulombiano entre o próton e o elétron dado por
				 \begin{align}
					 \label{eq:p3-potProtonEletron}
					 V(r) &= -\frac{e^{2}}{4 \pi \varepsilon_{0}}
				 \end{align}
				 levando a eq. \eqref{eq:p3-potProtonEletron} em \eqref{eq:p3-segSimplif-2} obtemos
				 \begin{align}
					 \label{eq:p3-segSimplif-2_paramsRbohr}
					 \begin{split}
						 \frac{d^{2}R(r)}{dr^{2}}+\frac{2}{r}\frac{dR(r)}{dr}+\frac{2m}{\hbar^{2}}\left(\frac{e^{2}}{4 \pi \varepsilon_{0} r}\right)R(r) &= K^{2}R(r)\\
						 \frac{d^{2}R(r)}{dr^{2}}+\frac{2}{r}\frac{dR(r)}{dr}+\frac{me^{2}}{4 \pi \varepsilon_{0}\hbar^{2}}\left(\frac{2R(r)}{ r}\right) &= K^{2}R(r)
					 \end{split}
				 \end{align}
				 usando o raio de Bohr eq. \eqref{eq:p3-paramsRbohr} em \eqref{eq:p3-segSimplif-2_paramsRbohr}, ficamos com
				 \begin{align}
					 \label{eq:p3-atomHidrogBohr}
					 \boxed{
						 \frac{d^{2}R(r)}{dr^{2}}+\frac{2}{r}\frac{dR(r)}{dr}+\frac{2}{a}\frac{R(r)}{r}=K^{2}R(r)
					 }
				 \end{align}
			 \item Obtendo a solução assintótica da parte radial da eq. \eqref{eq:p3-atomHidrogBohr}
				 \begin{align}
					 \begin{split}
						 \frac{d^{2}R(r)}{dr^{2}}+\cancelto{0}{\frac{2}{r}}\frac{dR(r)}{dr}+\frac{2}{a}\cancelto{0}{\frac{1}{r}}R(r)-K^{2}R(r) &= 0\\
						 \frac{d^{2}R(r)}{dr^{2}}-K^{2} &= 0
					 \end{split}
				 \end{align}
				 a solução geral da equação acima é conhecida e já eliminando a parte que diverge, ficamos com
				 \begin{align}
					 \label{eq:p3-solRadial}
					 R(r) &= A \mathrm{e}^{-Kr}
				 \end{align}
				 Encontrando os valores possíveis para $K$ na expressão acima é possível encontrar o autovalor de $\abs{E}$, logo
				 \begin{align}
					 \label{eq:p3-primDerivada}
					 \begin{split}
						 \frac{dR(r)}{dr} &= -KA \mathrm{e}^{-Kr}\\
															&= -KR(r)
					 \end{split}
				 \end{align}
				 e
				 \begin{align}
					 \label{eq:p3-segDerivada}
					 \begin{split}
						 \frac{d^{2}R(r)}{dr^{2}} &= K^{2}A \mathrm{e}^{-Kr}\\
																			&= K^{2}R(r)
					 \end{split}	
				 \end{align}
				 substituindo as eqs. \eqref{eq:p3-primDerivada} e \eqref{eq:p3-segDerivada} em \eqref{eq:p3-atomHidrogBohr}, ficamos com
				 \begin{align}
					 \begin{split}
						 \cancel{K^{2}R(r)}-\frac{2}{r}KR(r)+\frac{2}{a}\frac{R(r)}{r} &= \cancel{K^{2}R(r)}\\
						 K &= \frac{1}{a}
					 \end{split}
				 \end{align}
				 usando $K^{2}=1/a^{2}$ na eq. \eqref{eq:p3-paramsRbohr} obtemos o autovalor de energia
				 \begin{align}
					 \frac{2m\abs{E}}{\hbar^{2}} &= \frac{1}{a^{2}}\\
				 \end{align}
				 \begin{align}
					 \boxed{
						 \abs{E}=\frac{\hbar^{2}}{2ma^{2}}
					 }
				 \end{align}
				 precisamos normalizar a $R(r)=A\mathrm{e}^{2r/a}$, logo
				 \begin{align}
					 \begin{split}
						 \int\limits_{v}\abs{R(r)}^{2}\,d^{3}r &= 1\\
						 4 \pi \int\limits_{0}^{\infty}A^{2}\mathrm{e}^{-2r/a}r^{2}\,d{r} &= 1\\
						 4 \pi A^{2} \left[\underbrace{ -r^{2}\frac{\mathrm{e}^{-2r/a}}{2/a}\Bigg|^{0}_{\infty}}_{=0}+\frac{a}{\cancel{2}}\int\limits_{0}^{\infty}\mathrm{e}^{-2r/a}\cancel{2}r\,d{r}\right] &= 1\\
						 4 \pi A^{2}a \left[\underbrace{r\left(-\frac{a}{2}\mathrm{e}^{-2r/a}\right)\Bigg|^{0}_{\infty} }_{=0}+\frac{a}{2}\int\limits_{0}^{\infty}\mathrm{e}^{-2r/a}\,d{r}\right] &= 1\\
						 2 \pi A^{2} a^{2}\left[-\frac{a}{2}\mathrm{e}^{-2r/a}\Bigg|^{0}_{\infty} \right] &= 1\\
						 2\pi A^{2} a^{2} \left[\frac{a}{2}\right] &= 1\\
						 A &= \frac{1}{\sqrt{\pi a^{3}}}
					 \end{split}
				 \end{align}
				 ou seja
				 \begin{align}
				 	\boxed{
						R(r)=\frac{1}{\sqrt{\pi a^{3}}}\mathrm{e}^{-r/a}
					}	
				 \end{align}
				 \textcolor{gray}{[continua...]}
		 \end{enumerate}
	 \end{sol}

 \end{prob}
 % --------------------------------------------- %

 % --------------------------------------------- %
 % Q04
 % --------------------------------------------- %
 \addcontentsline{toc}{section}{Problema 04}
 \begin{prob}
	 Para a equação de $v$, do átomo de hidrogênio,
	 \begin{align}\label{eq:p4-1}
		 \rho v^{\prime\prime}(\rho)+2(l+1-\rho)v^{\prime}(\rho)+\left[\rho_{0}-2(l+1)\right]v(\rho)=0
	 \end{align}
	 encontre a relação de recorrência
	 \begin{align}
		 c_{j+1}	= \frac{2(j+l+1)-\rho_{0}}{j(j+1)+2(l+1)(j+1)}c_{j}
	 \end{align}

	 \begin{sol}
		 Dado que
		 \begin{align}
			 v(\rho) &= \sum_{j}c_{j} \rho^{j}
		 \end{align}
		 tem-se que
		 \begin{subequations}
			 \begin{align}
				 v^{\prime}(\rho) &= \sum_{j}jc_{j} \rho^{j-1}\label{eq:p4-2a}\\
				 v^{\prime\prime}(\rho) &= \sum_{j}j(j-1)c_{j} \rho^{j-2}\label{eq:p4-2b}
			 \end{align}	
		 \end{subequations}
		 substituindo \eqref{eq:p4-2a} e \eqref{eq:p4-2b} em \eqref{eq:p4-1} ficamos com
		 \begin{align}
			 \begin{split}
				 \rho\sum_jj(j-1)c_{j} \rho^{j-2}+2(l+1-\rho)\sum_{j}jc_{j} \rho^{j-1}+\\+[\rho_{0}-2(l+1)]\sum_{j}c_{j} \rho^{j} &= 0\\
				 \sum_{j}j(j-1)c_{j} \rho^{j-1}+2(l+1)\sum_{j}jc_{j} \rho^{j-1}-2\sum_{j}jc_{j} \rho^{j}+\\+[\rho_{0}-2(l+1)]\sum_{j}c_{j} \rho^{j} &= 0
			 \end{split}
		 \end{align}
		 ajustando os índices dos dois primeiros somatórios acima, para $j-1=j^{\prime}$
	 \end{sol}
	 \begin{align}
		 \begin{split}
			 \sum_{j}(j+1)jc_{j+1} \rho^{j}+2(l+1)\sum_{j}(j+1)c_{j+1} \rho^{j}+\\-2\sum_{j}jc_{j} \rho^{j}+[\rho_{0}-2(l+1)]\sum_{j}c_{j} \rho^{j} &= 0
		 \end{split}	
	 \end{align}
	 ocultando o somatório e igulando os coeficentes de $\rho^{j}$ obtemos
	 \begin{align}
		 \begin{split}
			 c_{j+1}(j+1)j+c_{j+1}(j+1)2(l+1) &= 2jc_{j}+2(l+1)c_{j}-c_{j} \rho_{0}\\
			 c_{j+1}\left[j(j+1)+2(l+1)(j+1)\right] &= \left[2j+2(l+1)-\rho_{0}\right]c_{j}\\
		 \end{split}	
	 \end{align}
	 logo,
	 \begin{align}
		 \boxed{
			 c_{j+1} = \frac{2\left(j+l+1\right)-\rho_{0}}{j(j+1)+2(l+1)(j+1)}
		 }
	 \end{align}
 \end{prob}
\textcolor{gray}{[continua...]}

