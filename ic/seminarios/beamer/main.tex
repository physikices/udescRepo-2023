\documentclass[aspectratio=169]{beamer}

% Pacote de estilo da UDESC
\usepackage{style/udesc}
\usepackage{listings}
\usepackage{hyperref}
\setbeamertemplate{itemize items}[circle]
\usepackage[abnt-emphasize=bf,abnt-and-type=e,alf]{abntex2cite}%Citações ABNT

% Incluir arquivos da pasta figuras
\graphicspath{{./figuras/}}

\setbeamertemplate{frametitle continuation}{}

% aPacote de texto aleatório
\usepackage{lipsum}

\lstset{ 
	basicstyle=\footnotesize,        % the size of the fonts that are used for the code
	breakatwhitespace=false,         % sets if automatic breaks should only happen at whitespace
	breaklines=true,                 % sets automatic line breaking
	captionpos=b,                    % sets the caption-position to bottom
	deletekeywords={...},            % if you want to delete keywords from the given language
	escapeinside={\%*}{*)},          % if you want to add LaTeX within your code
	extendedchars=true,              % lets you use non-ASCII characters; for 8-bits encodings only, does not work with UTF-8
	firstnumber=0,                % start line enumeration with line 1000
	frame=single,	                   % adds a frame around the code
	keepspaces=true,                 % keeps spaces in text, useful for keeping indentation of code (possibly needs columns=flexible)
	language=Java,                 % the language of the code
	morekeywords={*,...},            % if you want to add more keywords to the set
	numbers=none,                    % where to put the line-numbers; possible values are (none, left, right)
	numbersep=0pt,                   % how far the line-numbers are from the code
	rulecolor=\color{black},         % if not set, the frame-color may be changed on line-breaks within not-black text (e.g. comments (green here))
	showspaces=false,                % show spaces everywhere adding particular underscores; it overrides 'showstringspaces'
	showstringspaces=false,          % underline spaces within strings only
	showtabs=false,                  % show tabs within strings adding particular underscores
	stepnumber=2,                    % the step between two line-numbers. If it's 1, each line will be numbered
	tabsize=1,	                   % sets default tabsize to 2 
	basicstyle=\fontsize{7}{8}\selectfont\ttstyle,
	keywordstyle=\color{blue},
	commentstyle=\fontsize{7}{8}\selectfont\ttstyle\color{gray},
	stringstyle=\color{orange},
}

% Início do documento
\begin{document}

%%
%%	Incluir \capa para os slides
%% 
\titulo{Distribuições Partônicas}
\subtitulo{Parametrizações}
\newcommand{\autor}{Rodrigo Ribamar Silva do Nascimento}
\newcommand{\github}{github.com/physikices}
\newcommand{\email}{rodrigo.nascimento@edu.udesc.br}
\newcommand{\website}{}
\frase{2023 - IC Física de Partículas}
\universidade{Universidade do Estado de Santa Catarina}
\capa

\AtBeginSection[]{
	\begin{frame}<beamer>
		\frametitle{Seções}
		\tableofcontents[currentsection]
\end{frame}}

\begin{frame}{Problema Principal}
	\framesubtitle{$n$ de glúons}
	\begin{block}{Qual a quantidade precisa de glúons no interior do próton?}
		Is every even number the sum of two primes?
		\cite{article2016}
	\end{block}
\end{frame}

\begin{frame}
	\frametitle{Sample frame title}

	In this slide, some important text will be
	\alert{highlighted} because it's important.
	Please, don't abuse it.

	\begin{block}{Remark}
		Sample text
	\end{block}

	\begin{alertblock}{Important theorem}
		Sample text in red box
	\end{alertblock}

	\begin{examples}
		Sample text in green box. The title of the block is ``Examples".
	\end{examples}
\end{frame}

\begin{frame}
	\frametitle{Two-column slide}
	\begin{columns}
		\column{0.5\textwidth}
		This is a text in first column asda asd as  da s da s d asdasdasdasd asdas dasd asd.
		$$\beta(\alpha_s)=\mu^{2}\frac{d\alpha_s}{d\mu^2}$$
		\begin{itemize}
			\item First item
			\item Second item
		\end{itemize}

		\column{0.5\textwidth}
		\imagem{img/alice.png}{Alice}{img01}
	\end{columns}
\end{frame}

\begin{frame}{Tikz Picture}
	\begin{figure}[H]
		\tikzset{
			basic/.style  = {draw, text width=5em, drop shadow,  rectangle},
			root/.style   = {basic, thin, align=center,
			fill=gray!45 , text width=5em},
			level 2/.style = {basic, thin,align=center, fill=gray!30,
			text width=5em},
			level 3/.style = {basic, thin, align=left, fill=gray!20, text
			width=5em, node distance = 40pt} 
		}
		\begin{tikzpicture}[
			level 1/.style={sibling distance=30mm},
			edge from parent/.style={->,draw},
			>=latex]

			% root of the the initial tree, level 1
			\node[root] {BRD1 ($\alpha_{..},\beta_{..}$)}
				% The first level, as children of the initial tree
				child {node[level 2] (c1) {$(\alpha_{..},\beta_{i.})$}}
				child {node[level 2] (c2) {$ (\alpha_{.j},\beta_{..})$}}
				child {node[level 2] (c3) { $(\alpha_{..},\beta_{.j})$}}
				child {node[level 2] (c4) { $(\alpha_{i.},\beta_{..})$}};

			% The second level, relatively positioned nodes
			\begin{scope}[every node/.style={level 3}]
				\node [below of = c1, xshift=10pt] (c11) { $(\alpha_{i.},\beta_{i.})$};
				\node [below of = c11 ] (c12) {BRD9 $(\alpha_{.j},\beta_{i.})$};
				%\node [below of = c12] (c13) {$(\alpha_{..},\beta_{i.})$};

				\node [below of = c2, xshift=10pt] (c21) { $(\alpha_{.j},\beta_{.j})$};
				\node [below of = c21 ] (c22) {BRD9 $(\alpha_{.j},\beta_{i.})$};

				\node [below of = c3, xshift=10pt] (c31) { $(\alpha_{.j},\beta_{.j})$};
				\node [below of = c31 ] (c32) {BRD8 $(\alpha_{i.},\beta_{.j})$};

				\node [below of = c4, xshift=10pt] (c41) { $(\alpha_{i.},\beta_{i.})$};
				\node [below of = c41 ] (c42) { $(\alpha_{i.},\beta_{.j})$};

				%\node [below of = c2, yshift=-15pt, xshift=10pt] (c21) {Membership driven Re-Keying};
				%\node [below of = c21] (c22) {Time driven Re-Keying};

				%\node [below of = c3, xshift=15pt] (c31) {Ring-based Cooperation};
				%\node [below of = c31] (c32) {Hierarchical Cooperation};
				%\node [below of = c32] (c33) {Broadcast Cooperation};
			\end{scope}

			% lines from each level 1 node to every one of its "children"
			\foreach \value in {1,2}
			\draw[->] (c1.195) |- (c1\value.west);

			\foreach \value in {1,2}
			\draw[->] (c2.195) |- (c2\value.west);

			\foreach \value in {1,2}
			\draw[->] (c3.195) |- (c3\value.west);
			\foreach \value in {1,2}
			\draw[->] (c4.195) |- (c4\value.west);

		\end{tikzpicture}\\
		%\caption{Schematic Presentation of BRD models~\citep{Jansen05} } 
	\end{figure}
\end{frame}


\begin{frame}[allowframebreaks]
	\frametitle{Referencias}
	\bibliography{referencias.bib}
\end{frame}

\contato{%
	Contato: \\
	\autor{} \\
	\email{} \\
	\github{} \\
	\website{}
}

\capadetras{}

\end{document}
