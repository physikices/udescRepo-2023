\documentclass{udesc_report}
\usepackage[brazil]{babel}
\usepackage{stdtheme}

\begin{document}
% [FIXME] translate \refs: Table, Appendix, etc,
\selectlanguage{brazil}
%-----------------------------------------------%
% Pretextual Elements
%-----------------------------------------------%
\pretextual
\imprimircapa
\imprimirfolhaderosto

\begin{folhadeaprovacao}
	\thispagestyle{empty}
  \begin{center}
    {\ABNTEXchapterfont\large\imprimirautor}

    \vspace*{\fill}\vspace*{\fill}
    \begin{center}
      \ABNTEXchapterfont\bfseries\Large\imprimirtitulo
    \end{center}
    \vspace*{\fill}
    
    \hspace{.45\textwidth}
    \begin{minipage}{.5\textwidth}
        \imprimirpreambulo
    \end{minipage}%
    \vspace*{\fill}
   \end{center}
        
   Trabalho aprovado. \imprimirlocal, 06 de Julho de 2023:

   \assinatura{\textbf{\imprimirorientador} \\ Orientadora} 
   \assinatura{\textbf{\imprimircoorientador} \\ Supervisor}
   \assinatura{\textbf{Prof. Dr. Jacimar Nahorny} \\ Membro Convidado}
   %\assinatura{\textbf{Professor} \\ Convidado 3}
   %\assinatura{\textbf{Professor} \\ Convidado 4}
      
   \begin{center}
    \vspace*{0.5cm}
    {\large\imprimirlocal}
    \par
    {\large\imprimirdata}
    \vspace*{1cm}
  \end{center}
\end{folhadeaprovacao}

%-----------------------------------------------%
% [XXX] Do not remove the space below!
%-----------------------------------------------%

%-----------------------------------------------%
\includepdf[pages=-]{assets/documentos_merged.pdf}
%-----------------------------------------------%
\setlength{\absparsep}{18pt} % ajusta o espaçamento dos parágrafos do resumo
\begin{resumo}
	\thispagestyle{empty}
	Este trabalho tem por objetivo relatar as atividades desenvolvidas ao longo do Estágio Currícular Supervisionado IV, componente curricular obrigatório do curso de Licenciatura em Física da \ac{UDESC}. Foi realizado na Escola de Educação Básica Giovani Pasqualini Faraco, situada junto ao município de Joinville. Dentre as principais atividades desenvolvidas, fez-se a leitura e análise dos documentos oficiais da unidade escolar; a observação de aulas da disciplina de Física ministradas pelo professor supervisor, seguida de análises de como se articulam os momentos pedagógicos; a observação e caracterização do ambiente escolar e recursos didáticos disponíveis na unidade concedente e o desenvolvimento de uma Sequência de Ensino contendo dez aulas programadas para a turma de 2ª série do Novo Ensino Médio noturno. Este estágio desenvolveu-se simultâneamente à implantação do Novo Ensino Médio nas turmas de 2ª série, de modo que a Sequência de Ensino desenvolvida, desdobrou-se sobre o tema \textit{Furacões}, sujeito à Trilha: Modelagem de Fenômenos Naturais, Sociais e seus Impactos, componente currícular da parte flexível do Novo Ensino Médio, abordado no contexto da Área de Ciências da Naturezas e suas Tecnologias. A sequência foi organizada em três etapas, na primeira buscou-se apresentar o fenômeno com base nas suas principais características, na segunda conduziu-se o estudo qualitativo à cerca das variáveis fundamentais ao seu entendimento, por fim discutiu-se as causas e efeitos do fenêomeno em escala global afim de promover a compreensão sistêmica do fenômeno.

	\textbf{Palavras-chave}: \firstkey ; \secondkey ; \thirdkey.
\end{resumo}

% -----------------------------------------------%
 %% inserir lista de ilustrações, tabelas, listagem 
% de códigos, abreviaturas, símbolos
% -----------------------------------------------%
\pdfbookmark[0]{\listfigurename}{lof}
\listoffigures*
\cleardoublepage
% -----------------------------------------------%
% inserir lista de TABELAS
% -----------------------------------------------%
\pdfbookmark[0]{\listtablename}{lot}
\listoftables*
\cleardoublepage

% -----------------------------------------------%
% inserir lista de QUADROS
% -----------------------------------------------%
\pdfbookmark[0]{\listofquadrosname}{loq}
\listofquadros*
\cleardoublepage

% -----------------------------------------------%
% inserir lista de CÓDIGOS
% -----------------------------------------------%
\pdfbookmark[0]{\lstlistlistingname}{lol}
\begin{KeepFromToc}
\lstlistoflistings
\end{KeepFromToc}
\cleardoublepage
% -----------------------------------------------%

% -----------------------------------------------%
% inserir lista de abreviaturas e siglas
% -----------------------------------------------%
\pdfbookmark[0]{Lista de abreviaturas e siglas}{loa}
\cleardoublepage
% --%%%%%%%%%%%%%% Como usar o pacote acronym
% \ac{acronimo} -- Na primeira vez que for citado o acronimo, o nome completo irá aparecer
%                  seguido do acronimo entre parênteses. Na proxima vez somente o acronimo
%                  irá aparecer. Se usou a opção footnote no pacote, entao o nome por extenso
%                  irá aparecer aparecer no rodapé
%
% \acf{acronimo} -- Para aparecer com nome completo + acronimo
% \acs{acronimo} -- Para aparecer somente o acronimo
% \acl{acronimo} -- Nome por extenso somente, sem o acronimo
% \acp{acronimo} -- igual o \ac mas deixando no plural com S (ingles)
% \acfp{acronimo}--
% \acsp{acronimo}--
% \aclp{acronimo}--


\chapter*{Lista de abreviaturas e siglas}%
% \addcontentsline{toc}{chapter}{Lista de abreviaturas e siglas}
\markboth{Lista de abreviaturas e siglas}{}

% Para diminuir o espaçamento entre linhas no ambiente de listas acronym
% \let\oldbaselinestretch=\baselinestretch%
% \renewcommand{\baselinestretch}{.2}%
% \large\normalsize%



\begin{acronym}
	\let\oldbaselinestretch=\baselinestretch%
	\renewcommand{\baselinestretch}{.2}%
	\large\normalsize%
	\acro{BNCC}{Base Nacional Comum Curricular}
	\acro{LDB}{Lei de Diretrizes de Bases da Educação Nacional}
	\acro{NEM}{Novo Ensino Médio}
	\acro{PCSC}{Proposta Curricular de Santa Catarina}
	\acro{PNE}{Plano Nacional de Educação}
	\acro{CNE}{Conselho Nacional de Educação}
	\acro{EEB}{Escola de Educação Básica}
	\acro{GPF}{Giovani Pasqualini faraco}
	\acro{EF}{Ensino Fundamental}
	\acro{PPP}{Projeto Político Pedagógico}
	\acro{ATP}{Assistente Técnico Pedagógico}
	\acro{DVD}{\textit{Digital Versatile Disc}}
	\acro{EI}{Ensino por Investigação}
	\acro{NI/D}{Não-Interativo/Dialógico}
	\acro{I/D}{Interativo-Dialógico}
	\acro{HC}{História da Ciência}
	\acro{TIC}{Tecnologia da Informação e Comunicação}
	\acro{ENEM}{Exame Nacional do Ensino Médio}
	\acro{I/DA}{Interativo/Dialógico de Autoridade}
\end{acronym}

% -----------------------------------------------%
% inserir lista de símbolos
% -----------------------------------------------%
\begin{simbolos}  
  \item[$\Delta$] Variações de Grandezas Físicas
  \item[$X_i$] Medida Inicial da Grandeza $X$
  \item[$L$] Grandeza Física Associada ao Comprimento
  \item[$\alpha$] Coeficiente de Dilatação Linear
  \item[$T$] Grandeza Física Associada a Temperatura
  \item[$A$] Grandeza Física Associada a Área Superficial
  \item[$\beta$] Coeficiente de Dilatação Superficial
  \item[$V$] Grandeza Física Associada ao Volume
  \item[$\gamma$] Coeficiente de Dilatação Volumétrica 
  \item[$X_f$] Medida final da Grandeza $X$
\end{simbolos}
% -----------------------------------------------%

% -----------------------------------------------%
% inserir o SUMÁRIO
% -----------------------------------------------%
\pdfbookmark[0]{\contentsname}{toc}
\tableofcontents*
\cleardoublepage
% -----------------------------------------------%

%-----------------------------------------------%
% CONTENTS
%-----------------------------------------------%
\chapter{Introdução} % (fold)
\label{chap:Introdução}
\lettrine{D}{esde} o seu surgimento no século XI  aos dias atuais, o estágio \textit{(do lat: stagium)} é associado à aprendizagem posta em prática em ambiente controlado sob a tutela e supervisionamento técnico apropriado \cite{COLOMBO:2014}, visando-se estabelecer convenientemente a interrelação entre os saberes adquiridos ao longo dos percursos formativos e o exercício da futura profissão do acadêmico, neste sentido os estágio representam antes uma oportunidade concreta 

% chapter Introdução (end)


\postextual
%-----------------------------------------------%
% Bibliography
%-----------------------------------------------%
\bibliography{src/refs.bib}
%-----------------------------------------------%
% Appendix
%-----------------------------------------------%
\begin{apendicesenv}
	\chapter{Este é um apendice} % (fold)
	\label{cha:Este é um apendice}
	Este é um apêndice
\section{Esta é uma seção do apendice} % (fold)
\label{sec:Esta é uma seção do apendice}
Esta seção esta sendo chamada pelo comando autoref \autoref{cha:Este é um apendice} e \autoref{sec:Esta é uma seção do apendice}
% section Esta é uma seção do apendice (end)

	% chapter Este é um apendice (end)
\end{apendicesenv}

%-----------------------------------------------%
% Attachments
%-----------------------------------------------%
\begin{anexosenv}
	\chapter{Avaliação do Professor Supervisor} % (fold)
	\label{chap:Avaliação do Professor Supervisor}
	\includepdf[pages=-]{contents/attachments/ava_prof.pdf}
	% chapter Avaliação do Professor Supervisor (end)
	\chapter{Material da Trilha} % (fold)
	\label{chap:Material da Trilha}
	\includepdf[pages=-]{contents/attachments/CBTCem-cad3.pdf}
	% chapter Material da Trilha (end)
\end{anexosenv}


%-----------------------------------------------%
\end{document}

