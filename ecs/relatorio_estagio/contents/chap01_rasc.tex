\lettrine{O} termo estágio \textit{(do lat: stagium)} passou por diversas mudanças desde o seu surgimento no século XI aos dias atuais, como relatam os autores \cite{COLOMBO:2014}. Atualmente está associado ao período em que a aprendizagem  adquirida, ao longo dos percurusos formativos do acadêmico, é posta em prática em ambiente controlado, sob a tutela e o supervisionamento técnico apropriado, visando-se promover o aprendizado de competências próprias da atividade profissional e à contextualização curricular, conforme a Lei nº 11.788 de 25 de setembro de 2008 \cite{BRASIL:2008}. No âmbito educacional, o estágio na formação docente é a base para a experienciação de práticas pedagógicas e buscas por metodologias que instiguem os estudantes, é nesta etapa que o acadêmico pode vivenciar a realidade escolar, os principais desafios e dificuldades que permeiam a profissão docente.

A chegada da \textit{era digital} e a massificação dos meios de comunicação, trouxeram consigo um problema desafiador para as metodologias de ensino dito \textit{passivas}, passivas porque não levam em consideração a forma como os alunos constroem efetivamente o conhecimento, não há espaço para o desenvolvimento de autonomia intelectual dentro desta perspectiva, concentrando toda a carga do processo de ensino na figura do professor, pelo menos é o que diz \citeonline{MORAES:2018}, após mais de 20 anos participando do planejamento e coordenação de políticas públicas ligadas ao uso de tecnologias educacionais no Brasil. O resultado disso tudo é percebido em estatísticas que monitoram a performance do país no que diz respeito ao tema.

Apenas para citar alguns destes números e trazer à tona um pouco da dimensão do problema, de acordo com os resultados oficiais das avaliações aplicadas à rede pública de ensino, o percentual de alunos com aprendizagem adequada em língua portuguesa, no 5º ano do ensino fundamental evoluiu de 35,6\% em 2011, para 56,5\% em 2019, nos anos finais, de 21,8\% para 35,9\%, e, no ensino médio, de 23,3\% para 31\% no mesmo período\footnote{Dados do Anuário Brasileiro da Educação Básica}. Apesar de representarem um certo avanço, tais números ressoam bem preocupantes quando comparados aos de outros países no \ac{PISA}, no melhor dos cenários o país ocupa a 60ª posição do ranking mundial em leitura, a 68ª em ciências e a 74ª posição em matemática, isso num total de 78 países avaliados em 2018\footnote{Fonte: http://portal.mec.gov.br/}.

Evidentemente não se deve inferir que este panorama desenvolveu-se única e exclusivamente em detrimento desta nova conjuntura estabelecida após a virada do século, onde as novas tecnologias da informação ocupam lugar de destaque ao conceder eficientemente meios de propagação, disseminação e compartilhamento de ideias. Basta olharmos para décadas atrás e enumerar a quantidade de projetos e iniciativas educacionais descontinuados por não atenderem e/ou não se (re)adequarem, aos propósitos de uma sociedade cada vez mais diversificada, dinâmica, complexa e exigente. 

O que a pesquisa em educação tem evidenciado, é que na prática qualquer iniciativa vertical que não pressuponha a inversão dos papeis em sala de aula, tornando o aluno centro do processo de ensino-aprendizagem, autor da construção de seu conhecimento, sujeito participativo, ativo e protagonista de todo o processo, está fadada ao fracasso, -- talvez devessemos daí atribuir o sucesso das redes sociais, e mídias interativas, ao fomentar em seus usuários a impressão de autonomia, poder de decisão e controle do que deseja-se consumir, ou fazer-se de consumo, embora saibamos que esta "autonomia" ocorre sempre de forma moderada e subordinada aos vieses das empresas detentoras de tal tecnologia.


%
% Se fosse possível sintetizar todas as demandas do setor, em torno de um único termo que represente o ``santo graal'' da mudança de paradigma a qual encontra-se em vigor o processo educacional, este termo deve estar -- sem dúvidas, relacionado ao desenvolvimento do \textit{protagonismo dos estudantes}, não somente em sala de aula, como também fora dela.
%


Partindo deste princípio, este trabalho tem por finalidade apresentar as atividades desenvolvidas ao longo da disciplina de \ac{ECS4003}, em que buscou-se, aplicar uma \ac{SEI} baseada em \cite{MARIO:2017} de maneira subsidiária à administração de uma \textit{trilha do conhecimento}, elemento constituínte da \textit{parte flexível} dos currículos que integram a base do ensino médio catarinense.

Tal abordagem não seria possível sem  antes um aprofundamento na literatura pertinente, bem como em atenção aos referenciais normativos da educação nacional e estadual. De maneira sucinta e breve, apresento-os na sequência.
% chapter Introdução (end)

\section{Fundamentação} % (fold)
\label{sec:Referenciais}
No Brasil, além da Lei de nº 11.788 que regulamenta os estágios em âmbito nacional, têm-se ainda um outro complexo (e extenso) conjunto de leis e instruções normativas que norteiam e regulamentam toda a base educacional do país, a discussão e exposição detalhada sobre cada um destes dispositivos foge ao escopo deste trabalho, porém, o leitor interessado pode consultá-las através do portal \href{https://www.jusbrasil.com.br/}{jusbrasil}. No tópico seguinte apresentaremos apenas as instruções mais pertinentes à compreensão do que aqui pretende-se expor.

\subsection{Normativas} % (fold)
\label{sec:Referênciais normativos}
Previsto pela \ac{LDB} bem como pelo \ac{PNE}, é homolagado em 14 de abril de 2018, sob a gestão do então ministro da Educação, Sr. Rossieli Soares, o documento de caráter normativo entitulado \ac{BNCC}. O texto define um \textit{``conjunto orgânico e progressivo de aprendizagens essenciais que todos os alunos devem desenvolver ao longo das etapas e modalidades da Educação Básica''} \cite{BRASIL:2017}, neste documento definiu-se um total de dez competências gerais cujo as quais a educação deve:

\begin{citacao}
	``afirmar valores e estimular ações que contribuam para a transformação da sociedade, tornando-a mais humana, socialmente justa e, também, voltada a preservação da natureza.'' \Ibidem[pp. ~7]{BRASIL:2017}
\end{citacao}

No que tange a etapa relacionada ao que convencionou-se chamar de \ac{NEM}, e mais específicamente à área de \ac{CNT}, o texto ainda define três competências, a saber \Ibidem[pp. ~554--558]{BRASIL:2017}:
\begin{enumerate}
	\item Analisar fenômenos naturais e processos tecnológicos, com base nas interações e relações entre matéria e energia, para propor ações individuais e coletivas que aperfeiçoem processos produtivos, minimizem impactos socioambientais e melhorem as condições de vida em âmbito local, regional e global.
	\item Analisar e utilizar interpretações sobre a dinâmica da Vida, da Terra e do Cosmos para elaborar argumentos, realizar previsões sobre o funcionamento e a evolução dos seres vivos e do Universo, e fundamentar e defender decisões éticas e responsáveis.
	\item Investigar situações-problema e avaliar aplicações do conhecimento científico e tecnológico e suas implicações no mundo, utilizando procedimentos e linguagens próprios das Ciências da Natureza, para propor soluções que considerem demandas locais, regionais e/ou globais, e comunicar suas descobertas e conclusões a públicos variados, em diversos contextos e por meio de diferentes mídias e \ac{TDIC}.
\end{enumerate}

A \ac{BNCC} provê a oferta dos conteúdos mínimos exigidos nos currículos pelas diretrizes, dando autonomia às escolas e redes de ensino a elaborarem seus próprios currículos e projetos pedagógicos considerando-se as diferentes realidades locais provenientes de cada região. O estado de Santa Catarina, por exemplo, é pioneiro na elaboração de propostas curriculares, tendo a sua primeira publicada no ano de 1991 \cite{CATARINA:1991}, o que vem sendo continuamente revisada e reformulada conforme as demandas do estado.

Atualmente a \ac{PCSC} encontra-se alinhada às diretrizes e aos parâmetros curriculares norteadores da educação nacional, atua solidarimente à \ac{BNCC} e  através do \ac{ProBNCC}, dá suporte e subsídio para a consolidação do \ac{CBTCem}, no ano de 2021 lança em um total de quatro cadernos a proposta do estado para a implementação e consecução do \ac{NEM}, conforme segue abaixo \cite{CATARINA:2021}:

\begin{enumerate}[label=\Roman *)]
	\item \textbf{Caderno 1} - Disposições Gerais: textos introdutórios e gerais do Currículo Base do Ensino Médio do Território Catarinense;
	\item \textbf{Caderno 2} - Formação Geral Básica: textos da Formação Geral Básica, por Área do Conhecimento, do Currículo Base do Ensino Médio do Território Catarinense;
	\item \textbf{Caderno 3} - Parte Flexível do Currículo: Portfólio de Trilhas de Aprofundamento que fazem parte dos Itinerários Formativos no Território Catarinense;
	\item \textbf{Caderno 4} - Parte Flexível do Currículo: Portfólio de Componentes Curriculares Eletivos que fazem parte dos Itinerários Formativos no Território Catarinense.
\end{enumerate}

Na visão destes documentos, pode-se conceber os currículos do \ac{NEM} organizados essencialmente em duas etapas, a primeira relacionada ao conjunto de habilidades e competências das áreas de conhecimento que consolidam e aprofundam as aprendizagens conduzidas no \ac{EF}, tendo por carga horária máxima prevista para um total de 1.800 horas, e uma segunda parte denotada pelo termo \textit{flexível}\footnote{Segundo consta em \cite{CATARINA:2021d}, o termo \textit{flexível} está relacionado aos \ac{CCES} elaborados préviamente pela comissão técnica responsável e que são disponibilizados por meio de portifólios e intinerários formativos à escolha dos estudantes.} a qual é composta pelos \textit{Intinerários Formativos} e com carga horária total mínima de 1.200 horas. Como dito anteriormente, esta forma organizacional dos currículos, visa a atender às exigências da \ac{BNCC}, no entanto, vem recebendo críticas severas de autores como \cite{OSTERMANN:2021,ERICK:2020} que questionam quanto aos objetivos; nível de complexidade e aplicabilidade destas propostas.

No momento de escrita deste relatório, a implementação do \ac{NEM} encontra-se em processo de consulta pública, prevista para encerrar-se no mês subsequente do presente momento. Nesta consulta o \ac{MEC} busca dialogar com: a comunidade escolar, os profissionais da educação, as equipes técnicas, a sociedade civil, os pesquisadores e especialistas em geral, a fim de obter subsídios para avaliar e tomar providências sobre a revisão e reestruturação da política nacional do ensino médio, o que de fato revela as controvérsias ainda existentes na adoção e implementação da nova Base.

A sério, muito poder-se-ia, analisar e trazer à tona as discussões e ponderamentos levantados à cerca da implementação da \ac{BNCC} e em particular do \ac{NEM}, contudo, dado os objetivos deste trabalho, é sensato ater-se ao exposto apenas como pequeníssima amostra dos desafios e perspectivas inerentes, não somente, porém preponderantemente aos profissionais que lidam diariamente com a temática -- Educação.
