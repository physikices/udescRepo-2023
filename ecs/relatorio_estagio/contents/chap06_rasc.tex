
No dia 05 de abril, após conversa com o professor supervisor, definiu-se para as aulas de regências o tema: \textit{"Furacões"}, integrado à Trilha -- \textit{Modelagem de Fenômenos Naturais, Sociais e Seus Impactos\footnote{Caderno 3 -- Portfólio de Trilhas de Aprofundamento/ 5. Trilha de Aprofundamento Integrada Entre Áreas do Conhecimento/ 5.3 Modelagem de [\ldots] \cite[pp.~191-205]{CATARINA:2021c}.}}. A turma de aplicação da Trilha foi a do 2º ano do \ac{NEM}, turno noturno, com quatro aulas semanais, duas delas ocorrendo às quartas-feiras e as restantes nas sextas-feiras. Ainda neste dia, fixou-se a data de início da primeira aula para o dia 12 do mesmo mês. No total reservou-se dez aulas (cinco encontros) para aplicação da sequência.

Desde então, iniciou-se a fase de consultas aos repositórios, prioritariamente (mas não exclusivamente) da \ac{BDTD}, e também do \ac{PPGECMT} do \ac{CCT}, onde buscou-se por Produtos Educacionais que tratassem sobre o tema, para isto utilizou-se de palavras chaves como: \textit{Furacões; Ciclones; Tempestades; Física dos Furacões,} e etc.
\textcolor{red}{[DESENVOLVER MELHOR]}
O resultado das consultas, retornou publicações destinadas as áreas de pesquisa científica em climatologia, meteorologia e afins, não observado material didático apropriado, decidiu-se por refinar a consulta utilizando palavras chaves como: \textit{Estudo da Atmosfera no Ensino Médio; Meteorologia no Ensino Médio;}

No decorrer das consultas, encontrou-se publicações das áreas climatológicas e meteorológicas, todas de cunho técnico-científico (não didático-pedagógico), em virtude disto, optou-se por adaptar uma sequência didática fundamentada principalmente nos seguintes trabalhos:
\begin{enumerate}[label=\roman *)]
	\item Apostila desenvolvida ao estudo da disciplina de meteorologia básica da Universidade Federal do Paraná \cite{GRIMM:1999};
	\item Sequência de ensino investigativa sobre a previsão do tempo para o ensino médio \cite{MARIO:2017};
	\item Artigo sobre o impacto dos ciclones extratropicais no Brasil \cite{MARRAFON:2021};
	\item Dissertação de mestrado do curso de pós-graduação em meteorologia, material em que obteve-se as informações técnicas à cerca do Furacão Catarina \cite{ROCHA:2014} 
\end{enumerate}

\section{Materiais e Metodologia} % (fold)
\label{sec:Materiais e Metodologia}

Na elaboração da sequência didática, além dos trabalhos elencados acima, utilizou-se também o texto base da Trilha como orientação, material este que pode ser consultado na seção em Anexo \ref{chap:Material da Trilha}. O cronograma de aplicação da sequência, contendo o planejamento das datas, conteúdos e metodologias utilizadas é visto no \autoref{qua:sequencia-did}.

\textcolor{red}{[DESCREVER A METODOLOGIA UTILIZDA]}
% section Materiais e Metodologia (end)

\begin{quadro}[!ht]
\resizebox{\textwidth}{!}{%
\begin{tabular}{|r|l|r|l|c|}
\hline
\multicolumn{1}{|c|}{\textbf{Data}} &
  \multicolumn{1}{c|}{\textbf{Tema}} &
  \multicolumn{1}{c|}{\textbf{Aulas}} &
  \multicolumn{1}{c|}{\textbf{Conteúdo}} &
  \textbf{Metodologia} \\ \hline
\multirow{2}{*}{12/04} & \multirow{2}{*}{Introdução aos Tópicos da Trilha} & 01 & Apresentação dos fenômenos atmosféricos                  & Vídeo - Compilação   \\ \cline{3-5} 
                       &                                                   & 02 & Caracterização dos fenômenos atmosféricos                & Vídeo - Entrevista   \\ \hline
\multirow{2}{*}{14/04} & \multirow{2}{*}{Formação dos Ventos}              & 03 & Temperatura \& Pressão no contexto da atmosfera          & Expositiva/Dialogada \\ \cline{3-5} 
                       &                                                   & 04 & Gradiente de Pressão                                     & Expositiva/Dialogada \\ \hline
\multirow{2}{*}{19/04} & \multirow{2}{*}{Mecanismos de Troca de Energia}   & 05 & Células de convecção \& Convergência em baixas latitudes & Expositiva/Dialogada \\ \cline{3-5} 
                       &                                                   & 06 & Condensação \& Formação de Nuvens                        & Expositiva/Dialogada \\ \hline
\multirow{2}{*}{26/04} & \multirow{2}{*}{A Atmosfera como um Gás Ideal}    & 07 & Relação entre Pressão, Volume e Temperatura              & Slides/Simulação     \\ \cline{3-5} 
                       &                                                   & 08 & Derivação da Lei dos Gases Ideais                        & Quadro/Simulação     \\ \hline
\multirow{2}{*}{28/04} &
  \multirow{2}{*}{\begin{tabular}[c]{@{}l@{}}Conectando os Saberes e\\ Fechamento da Sequência\end{tabular}} &
  09 &
  \begin{tabular}[c]{@{}l@{}}Movimento Atmosférico Global (Células de Hadley) \&\\ Força de Coriolis\end{tabular} &
  Slides/Vídeos \\ \cline{3-5} 
                       &                                                   & 10 & Equilíbrio Térmico Global \& Forçante Antrópica          & Slides               \\ \hline
\end{tabular}%
}
\caption{Planejamento da sequência didática}
\label{qua:sequencia-did}
\end{quadro}

A fim de auxiliar nas aulas de regência, compilou-se um \href{https://www.youtube.com/watch?v=uPh4T-9gDJM}{vídeo}\footnote{Publicado na plataforma YouTube sob a url: \url{https://youtu.be/uPh4T-9gDJM}} de $3\min$ a partir de vídeos reais encontrados na internet, que mostram desde a formação do fenômeno\footnote{Ciclones e tornados de maneira geral.} a partir de frames produzidos por satélites geoestacionários\footnote{Somente ciclones que possuem raio de atuação dos ventos e tempo de vida muito superiores aos tornados.}, até a fase de dissipação no continente ocasionando danos à costa terrestre e consequentemente às cidades litorâneas. Também foi produzido uma sequência de slides como consta na seção do \autoref{chap:Slides}. Por fim, produziu-se um questionário avaliativo sobre todo o assunto da sequência, este questionário encontra-se disponível para consulta no \autoref{chap:Avaliação}.

\section{Perfil da Turma} % (fold)
\label{sec:Perfil da Turma}

A turma do 2º ano (6) do \ac{NEM} noturno, é composta oficialmente por 36 integrantes matriculados e destes, apenas 21 seguem cursando\footnote{Fonte: Demostrativo da Unidade Escolar -- \ac{SED}}. Durante as aulas da trilha, observou-se em média a presença de 14 alunos nas aulas das quartas-feiras, já nas das sextas, este número reduziu-se para a metade. Confirmando os alertas apontados pelos professores quando na ocasião do Conselho de Classe. Diante deste problema e ainda com o agravante de que nem sempre os estudantes presentes numa aula, estavam presentes na aula anterior, necessitou-se sempre dedicar uma boa parte da primeira aula de cada dia, para revisar os conteúdos da aula anterior, e evitou-se ao máximo o avanço da sequência nas aulas das sextas, uma vez que este dia coincide com o público de menor audiência da semana.

Do ponto de vista comportamental, o 2º(6) revelou-se uma turma quieta e bem educada, formada por pequenos grupos distribuídos aos cantos da sala. De maneira geral observou-se uma turma tímida, porém quando o professor insiste nas indagações, rompem um pouco a timidez e passam a participar aos pouco.

Quanto ao envolvimento nas atividades avaliativas, demostrou-se uma turma bem responsiva, pontuais e de iniciativa, se eventualmente perdem alguma atividade, procuram de imediato o professor para repô-la(s). De maneiera geral a qualidade dos trabalhos se mostrou suficiente, porém observou-se uma certo nível de insegurança ao exporem suas próprias percepções ou conclusões, fato constatado na identificação de respostas comumente atribuídas a retorno de pesquisas em sites de buscas, bem como na semelhança entre trabalhos de grupos/indivíduos diferentes.
% section Perfil da Turma (end)

\section{Aspectos Gerais da(s) Trilha(s)} % (fold)
\label{sec:Aspectos Gerais da Trilha}
\cite{CADORI:2022}

% section Aspectos Gerais da Trilha (end)
