\chapter{Regências} % (fold)
\label{chap:Regências}
No dia 05 de abril após reunião com o professor supervisor, foi designado o 2º Ano 6, do turno noturno, para a aplicação das regências, convencionou-se o início das atividades o dia 12 do corrente mês.

\setlength\intextsep{0pt}
\begin{wraptable}{r}[0pt]{0.6\textwidth}
	\centering
	\resizebox{1\linewidth}{!}{%
		\begin{tabular}{|lcc|cc|}
			\hline
			\multicolumn{1}{|c|}{\multirow{3}{*}{\textbf{BNCC}}} &
			\multicolumn{1}{c|}{\multirow{3}{*}{\textbf{\begin{tabular}[c]{@{}c@{}}Área do\\ Conhecimento\end{tabular}}}} &
			\multirow{3}{*}{\textbf{\begin{tabular}[c]{@{}c@{}}Componentes\\ Curriculares\end{tabular}}} &
			\multicolumn{2}{c|}{\textbf{Carga Horária}} \\ \cline{4-5} 
			\multicolumn{1}{|c|}{} &
			\multicolumn{1}{c|}{} &
														&
														\multicolumn{2}{c|}{\textbf{2ª série}} \\ \cline{4-5} 
			\multicolumn{1}{|c|}{} &
			\multicolumn{1}{c|}{} &
														&
			\multicolumn{1}{c|}{\begin{tabular}[c]{@{}c@{}}Carga\\ Horária\\ Semanal\\ ($\h/a$)\end{tabular}} &
			\begin{tabular}[c]{@{}c@{}}Carga\\ Horária\\ Anual\\ ($\h$)\end{tabular} \\ \hline
			\multicolumn{1}{|l|}{\multirow{12}{*}{\textbf{\begin{tabular}[c]{@{}l@{}}BNCC --\\ Formação\\ Geral\\ Básica\end{tabular}}}} &
			\multicolumn{1}{c|}{\multirow{4}{*}{\begin{tabular}[c]{@{}c@{}}Linguagens e suas\\ Tecnologias\end{tabular}}} &
			\begin{tabular}[c]{@{}c@{}}Línhua Portuguesa\\ e Literatura\end{tabular} &
			\multicolumn{1}{c|}{1} &
			30 \\ \cline{3-5} 
			\multicolumn{1}{|l|}{} &
			\multicolumn{1}{c|}{} &
			\begin{tabular}[c]{@{}c@{}}Educação\\ Física\end{tabular} &
			\multicolumn{1}{c|}{1} &
			30 \\ \cline{3-5} 
			\multicolumn{1}{|l|}{} &
			\multicolumn{1}{c|}{} &
			Arte &
			\multicolumn{1}{c|}{1} &
			30 \\ \cline{3-5} 
			\multicolumn{1}{|l|}{} &
			\multicolumn{1}{c|}{} &
			\begin{tabular}[c]{@{}c@{}}Língua Estrangeira\\ Inglês\end{tabular} &
			\multicolumn{1}{c|}{1} &
			30 \\ \cline{2-5} 
			\multicolumn{1}{|l|}{} &
			\multicolumn{1}{c|}{\multirow{3}{*}{\begin{tabular}[c]{@{}c@{}}Ciências da Natureza\\ e suas Tecnologias\end{tabular}}} &
			Química &
			\multicolumn{1}{c|}{1} &
			30 \\ \cline{3-5} 
			\multicolumn{1}{|l|}{} &
			\multicolumn{1}{c|}{} &
			Física &
			\multicolumn{1}{c|}{1} &
			30 \\ \cline{3-5} 
			\multicolumn{1}{|l|}{} &
			\multicolumn{1}{c|}{} &
			Biologia &
			\multicolumn{1}{c|}{1} &
			30 \\ \cline{2-5} 
			\multicolumn{1}{|l|}{} &
			\multicolumn{1}{c|}{\multirow{4}{*}{\begin{tabular}[c]{@{}c@{}}Ciências Humanas e \\ Sociais Aplicadas\end{tabular}}} &
			Geografia &
			\multicolumn{1}{c|}{1} &
			30 \\ \cline{3-5} 
			\multicolumn{1}{|l|}{} &
			\multicolumn{1}{c|}{} &
			História &
			\multicolumn{1}{c|}{1} &
			30 \\ \cline{3-5} 
			\multicolumn{1}{|l|}{} &
			\multicolumn{1}{c|}{} &
			Filosofia &
			\multicolumn{1}{c|}{1} &
			30 \\ \cline{3-5} 
			\multicolumn{1}{|l|}{} &
			\multicolumn{1}{c|}{} &
			Sociologia &
			\multicolumn{1}{c|}{1} &
			30 \\ \cline{2-5} 
			\multicolumn{1}{|l|}{} &
			\multicolumn{1}{c|}{\begin{tabular}[c]{@{}c@{}}Matemática e suas\\ Tecnologias\end{tabular}} &
			Matemática &
			\multicolumn{1}{c|}{1} &
			30 \\ \hline
			\multicolumn{3}{|l|}{\textbf{Carga Horária Total -- Formação Geral Básica}} &
			\multicolumn{1}{c|}{\textbf{12}} &
			\textbf{360} \\ \hline
			\multicolumn{1}{|c|}{\multirow{4}{*}{\textbf{\begin{tabular}[c]{@{}c@{}}Intinerário\\ Formativo\end{tabular}}}} &
			\multicolumn{2}{l|}{Projeto de Vida} &
			\multicolumn{1}{c|}{2} &
			60 \\ \cline{2-5} 
			\multicolumn{1}{|c|}{} &
			\multicolumn{2}{l|}{\begin{tabular}[c]{@{}l@{}}Segunda Língua\\ Estrangeira\end{tabular}} &
			\multicolumn{1}{c|}{1} &
			30 \\ \cline{2-5} 
			\multicolumn{1}{|c|}{} &
			\multicolumn{2}{l|}{\begin{tabular}[c]{@{}l@{}}Componente Curricular\\ Eletivo 1\end{tabular}} &
			\multicolumn{1}{c|}{0} &
			0 \\ \cline{2-5} 
			\multicolumn{1}{|c|}{} &
			\multicolumn{2}{l|}{Trilha de Aprofundamento} &
			\multicolumn{1}{c|}{10} &
			300 \\ \hline
			\multicolumn{3}{|l|}{\textbf{Carga Horária Total -- Intinerário Formativo}} &
			\multicolumn{1}{c|}{\textbf{13}} &
			\textbf{390} \\ \hline
			\multicolumn{3}{|l|}{\textbf{Carga Horária Semanal/Anual}} &
			\multicolumn{1}{c|}{\textbf{25}} &
			\textbf{750} \\ \hline
		\end{tabular}%
	}
	\caption{Matriz Curricular D do Novo Ensino Médio Noturno -- 2º série. Adaptado de \cite{CADORI:2022}}
	\label{tab:matriz-2_6}
\end{wraptable}
O 2º Ano 6 do turno noturno, ou simplesmente 2(6), foi enquadrado no currículo do \ac{NEM}, recém implantado na \ac{UE}, tendo esta em vigor a Matriz Curricular D de código 4079, sujeita a carga horária conforme consta na \autoref{tab:matriz-2_6}. Das seis opções de Trilhas ofertadas pela \ac{GPF}\footnoteC{\textit{v. \autoref{qua:trilhas-areas} e \ref{qua:trilhas-integradas}}}, referentes ao intinerário formativo desta etapa, foi disponibilizada para a turma, a Trilha de Aprofundamento Integrada Entre Áreas do Conhecimento, a de cód. 4537 -- \textit{Modelagem de Fenômenos Naturais, Sociais e Seus Impactos}, -- de onde sugeriu-se o tratamento das atividades de regência, por meio da abordagem temática a desenvolver-se sobre o tema: \textit{O evento natural extremo, de origem meteorológica conhecido por "Furacão".} Segue-se daí o contexto inicial para a elaboração da Sequência.




\section{O Perfil da Turma} % (fold)
\label{sec:Perfil da Turma}

O 2(6) é oficialmente composto por 36 (trinta e seis) estudantes dos quais 21 (vinte e um) seguem cursando. O número máximo de presença contabilizada numa aula de regência totalizou 18 (dezoito) alunos e o mínimo 7 (sete). Os dias da semana em que registrou-se o maior número de presença foram as quartas-feiras, os de menor número ocorreram sempre às sextas-feiras. Foi uma das turmas citadas no Conselho de Classe\footnoteC{\textit{v. \autoref{sub:Acompanhamento do Conselho de Classe}}} com problemas de frequências dentre os seus integrantes.

Do ponto de vista comportamental, o 2(6) revelou-se uma turma quieta e bem educada, formada por pequenos grupos distribuídos aos cantos da sala. Quando estimulados durante as aulas demoram um tempo até adotarem uma postura de participação ativa, mas cedem com algum esforço e insistência do professor\footnote{Nas aulas do professor titular a turma participou melhor.}. A maior parte da turma trabalha durante o dia o que de fato pode contribuir para o processo de evasão e baixa frequência observada. 

De acordo com o Conselho de Classe, a turma costuma obter bons resultados nas atividades avaliativas, referindo-se àqueles que frequentam regularmente às aulas.
% section Perfil da Turma (end)

\section{O Ensino por Trilhas} % (fold)
\label{sec:O Ensino por Trilhas}
As Trilhas de Aprofundamentos fazem parte do Intinerário Formativo, e portanto correspondem à parte flexível dos currículos do \ac{NEM}, a ela é destinada a maior parte de toda a carga horária do Intinerário, -- para o caso do 2(6), por exemplo, aproximadamente 77\% da carga horária do Intinerário é destinado ao ensino da Trilha -- logo, este componente torna-se um dos principais focos de todo o percurso.

Segundo o Caderno 1 do \ac{CBTCEM}\footnote{Cf. \cite{CATARINA:2021}}, o objetivo do ensino por Trilhas é o de promover, além da articulação entre objetos de conhecimento das respectivas áreas e seus componentes curriculares, também o tratamento metodológico contextualizado, diversificado e transdisciplinar, facilitando a articulação entre os diferentes campos dos saberes. É recomendado no ensino por Trilhas que se perpasse com maior ou menor ênfase pelos quatro eixos estruturantes definidos pelo \ac{CBTCEM}.

As Trilhas devem contemplar uma área do conhecimento (Trilhas de Aprofundamento por Áreas), integrar duas ou mais áreas do conhecimento (Trilhas de Aprofundamento Integradas), ou a educação técnica e profissional (Trilhas de Aprofundamento da Educação Profissional). Não são seriadas e devem ser disponibilizadas através de processos de escutas aos professores e aos estudantes a partir de portfólio especificado.

\subsection{A Tilha Selecionada} % (fold)
\label{sub:A Tilha Selecionada}
A Trilha Modelagem de Fenômenos Naturais, Sociais e seus Impactos\footnoteC{\textit{v. seção em Anexo -- \ref{chap:Material da Trilha}}} é uma das Trilhas de Aprofundamento Integradas Entre Áreas do Conhecimento que contemplam, obviamente, mais de uma área (a rigor contempla todas as áreas, porém não necessariamente todas as componentes curriculares, conforme interpretação dada ao texto). Na matriz ofertada pela \ac{UE} (Matriz D -- noturno), possui carga horária de 10 horas-aula semanais, das quais 3 são restritas ao ensino das componentes curriculares relacionadas a área de Ciências da Natureza e suas Tecnologias. Como já dito em capítulos anteriores\footnoteC{\textit{v. \autoref{sec:O Processo de Escolha e Oferta das Trilhas de Aprofundamento}}} esta Trilha em específico, não passou pelo critério de escolha dos alunos devido a pouca quantidade de turmas e indivíduos participantes desta etapa formativa no momento.

De acordo com o seu texto, esta Trilha tem por foco tratar da \textit{``modelagem de fenômenos naturais e sociais como chuva, tornados, deslizamentos, ciclos biogeoquímicos, desemprego, aumento na produção de riquezas, taxas de mortalidade, crescimento econômico, entre outros.''}

Por modelagem o texto entende que: \textit{``é um processo de ensino e aprendizagem que age de forma integrada com as diversas áreas do conhecimento''}, e ainda, \textit{``corrobora-se ser uma forma de estudar e revelar as características essenciais dos fenômenos estudados.''} O texto ainda sugere que, \textit{``os fenômenos naturais podem ser modelados por: maquetes, escultura, música, gráficos, fórmulas (modelos algébricos), desenhos, entre outros.''} 

Possui três unidades curriculares, sendo elas:
 \begin{enumerate}[label=\alph *)]
 		\item Unidade Curricular 1: Desastres  naturais no território catarinense: impactos econômicos e sociais;
		\item Unidade Curricular 2: Desafios e possibilidades de (com)viver pós-pandemia no mundo do trabalho;
		\item Unidade Curricular 3: A Natureza e sua força: fenômenos meteorológicos e seus impactos no mundo.
 \end{enumerate}

Como orientação metodológica, o texto da Trilha sugere: a Aprendizagem Baseada em Problema; a Metodologia da Problematização; a Leitura e análise de obras como: 1) Umashimenkanka [Trazendo luz para a vida] e 2) Straight (Em linha reta).

Finaliza orientando que se busque formas de avaliação de acordo com a \ac{BNCC} e que privilegie o desenvolvimento do protagonismo dos estudantes.
% subsection A Tilha Selecionada (end)
% section O Ensino por Trilhas (end)


\section{Planejamento} % (fold)
Na etapa de planejamento surgiram diversas dificuldades, a primeira e mais evidente veio da preocupação em prototipar uma proposta pedagógica em atenção aos preceitos da Trilha eleita para a turma, ressalta-se de antemão que o único material disponível de início foi o seu texto base, a leitura do mesmo fora do contexto dos cadernos do \ac{CBTCEM}, cujo os quais lhes dá o devido suporte, se mostrou inapropriada para todos os fins, além disso:
\begin{enumerate}[label=\alph *)]
	\item Mesmo que de posse dos cadernos do \ac{CBTCEM}, os textos não são claros e concisos, há uma série de referências cruzadas e terminologias, é preciso um bom tempo de estudos direcionados principalmente aos quatro primeiros Cadernos, bem como as referências feitas à \ac{BNCC};
	\item O estudo por Trilhas de Aprofundamento como disposto pelos cadernos do \ac{CBTCEM} é algo relativamente novo, de modo que o desenvolvimento de materiais pedagógicos, sugestões de experimentos, produtos didáticos e etc, deveram aparecer nos repositórios somente no decorrer dos próximos meses/anos, o que implica que toda proposta direcionada a este fim, deva ser construída e embasada desde o princípio, o que demanda tempo de desenvolvimento.
	\item O tema Furacão é amplo, pode ser tratado de diversas formas e com diferentes enfoques, é fundamental que se construa com o aluno conceitos preliminares indispensáveis à "modelagem"~do fenômeno que se queira desenvolver e devido a complexidade do assunto, deve-se tomar cuidado na elaboração das abordagens para que não ocorram descaracterizações por excessos;
	\item Para conceber estratégias que priorizem abordagens transdisciplinares sobre algum tema, é presumidamente esperado que se conheça a fundo sobre a fenomenologia do tema em análise, a fim de reconhecer-se as diversas contribuições das diferentes áreas do conhecimento ao objeto em estudo. 
\end{enumerate}

Contudo, conduziu-se consultas intensivas aos Cadernos do \ac{CBTCEM} paralelamente ao estudo da fenomenologia dos furacões em que possibilitou a construção de uma Sequência de Ensino empreendida em três etapas, visando-se superar as dificuldades acima listadas e promover ao máximo o que é pretendido pela Trilha em correlação com o tema sugerido, sendo elas: 1) apresentação do fenômeno e a caracterização conforme as suas peculiaridades; 2) estudo qualitativo das principais causas do fenômeno, destacando as variáveis mais importantes para a sua conformação; 3) modelagem por meio da compreensão do fenômeno a nível sistêmico e global. 
\subsection{Materiais e Metodologia} % (fold)
\label{sub:Materiais e Metodologia}
A organização da sequência nas três etapas descritas anteriormente, foi baseada na Metodologia Dialética de Conhecimento, uma metodologia ativa que:

\begin{citacao}
	``Entende o homem como um ser ativo e de relações. Assim, entende que o conhecimento não é \textit{transferido} ou \textit{depositado} (conforme a concepção tradicional), nem é inventado pelo sujeito (concepção espotaneísta), mas sim construído pelo sujeito na sua relação com os outros e com o mundo'' \cite[pp.~2]{VASCONCELLOS:1992}
\end{citacao}
De maneira sucinta a teoria dialética aponta que o conhecimento se dá a partir de três grandes momentos: a mobilização para o conhecimento \textit{(Síncrese)}; a construção do conhecimento \textit{(Análise)} e a elaboração da Síntese do Conhecimento \textit{(Síntese)}.

Para o desenvolvimento da primeira etapa da sequência, a de apresentação e caracterização do fenômeno, construiu-se um vídeo compilado (VT01)\footnote{Acessível em: \url{https://youtu.be/uPh4T-9gDJM}} com duração de $3\min$ a partir de 12 outros vídeos reais sobre o fenômeno. A caracterização do fenômeno foi promovida por meio de um outro vídeo (VT02)\footnote{Acessível em: \url{https://youtu.be/riXLskNJx20}}, este contido em $11\min$ de duração. Intercalado entre um vídeo e outro, inseriu-se momentos de discussões para que os estudantes sintam um pouco da dinâmica que se deseja estabelecer no decorrer da Sequência, -- a de participação ativa e efetiva nas aulas -- além disso, esta etapa teve por objetivos o de colocar os estudantes em contato com o objeto de estudo, assim como orienta o referencial.

Para a segunda etapa da sequência, apoiou-se primeiramente na literatura técnica diretamente relacionada ao fenômeno, o que encontra-se devidamente fundamentada no campo de estudo da meteorologia. O principal material utilizado nesta etapa, foi a apostila de Meteorologia Básica \cite{GRIMM:1999}, encontrada no repositório do departamento de Física da Universidade Federal do Paraná que também é ligado ao Grupo de Pesquisas em Meteorologia da instituição. Este material possibilitou o mapeamento dos conceitos basilares do fenômeno como, por exemplo: Pressão atmosférica; O Gradiente de pressão; Células de convecção, Fenômenos de convergência/divergência atmosférica e etc. Buscou-se cuidadosamente fazer a transposição didática destes conceitos de forma a adequá-los ao programa da disciplina do professor supervisor previsto para a turma.

A metodologia empregada nesta segunda etapa, teve por objetivos o de promover a participação ativa dos estudantes por meio de situações-problemas elaboradas a partir do cotidiano do aluno. A seleção das situações-problemas, foram inspiradas na pesquisa desenvolvida por \citeonline{MARIO:2017}, neste estudo, o autor utiliza uma ferramenta de pesquisa para investigar o nível de engajamento dos estudantes numa Sequência de Ensino Investigativa sobre a Previsão do Tempo. Assim, dada a proximidade (mas não a semelhança) dos assuntos abordados na pesquisa com o tema das regências, bem como a adequação aos referenciais das Metodologias de Aprendizagem Ativas é que se escolheu esta abordagem.

Por último, desenvolveu-se uma sequência de Slides para auxiliar na terceira etapa, cujo o objetivo foi o de sintetizar todo o estudo e proceder com a modelização do fenômeno, esta sequência de Slides pode ser vista no \autoref{chap:Slides}. Como recurso didático, além da sequência de Slides, selecionou-se uma simulação computacional do Grupo de Pesquisas \ac{PhET}\footnote{Acessível em: \url{https://phet.colorado.edu/pt_BR/simulations/gases-intro/activities}} destinada ao estudo dos gases ideais, e que neste contexto foi utilizado para simular a atmosfera terrestre e estabelecer relações com os conceitos estudados nas etapas anteriores. Um terceiro vídeo (VT03)\footnote{Acessível em: \url{https://youtu.be/dt_XJp77-mk}} de $3\min$ de duração, que trata do efeito Coriolis foi selecionado para apresentar a força de Coriolis e através dela exemplificar a complexidade na elaboração dos modelos meteorológicos reais de que tratam do assunto. A partir da discussão e análise do movimento das grandes massas de ar a nível global e da sua influência no equilíbio energético do planeta, bem como da formação das diversas regiões climáticas observadas, deve-se construir a visão sistêmica do fenômeno ressaltando-se a contribuição das diversas áreas do conhecimento.
% subsection Materiais e Metodologia (end)

Todo este desenvolvimento ficou condicionado à um total de 10 aulas e para dar fluidez à Sequência, foi destinado 4 aulas semanais (3 da Trilha + 1 da \ac{BNCC}) para aplicação da Sequência. O cronograma de aulas ficou como consta no \autoref{qua:sequencia-did}, logo a seguir: 
\vspace{10pt}

\label{sec:Planejamento}
\begin{quadro}[!ht]
	\resizebox{\textwidth}{!}{%
		\begin{tabular}{|r|l|r|l|c|}
			\hline
			\multicolumn{1}{|c|}{\textbf{Data}} &
			\multicolumn{1}{c|}{\textbf{Tema}} &
			\multicolumn{1}{c|}{\textbf{Aulas}} &
			\multicolumn{1}{c|}{\textbf{Conteúdos}} &
			\textbf{Recursos} \\ \hline
			\multirow{2}{*}{12/04} & \multirow{2}{*}{Introdução aos Tópicos da Trilha} & 01 & Apresentação dos fenômenos atmosféricos                  & VT01   \\ \cline{3-5} 
														 &                                                   & 02 & Caracterização dos fenômenos atmosféricos                & VT02   \\ \hline
			\multirow{2}{*}{14/04} & \multirow{2}{*}{Formação dos Ventos}              & 03 & Temperatura \& Pressão no contexto da atmosfera          & Expositiva/Dialogada \\ \cline{3-5} 
														 &                                                   & 04 & Gradiente de Pressão                                     & Expositiva/Dialogada \\ \hline
			\multirow{2}{*}{19/04} & \multirow{2}{*}{Mecanismos de Troca de Energia}   & 05 & Células de convecção \& Convergência em baixas latitudes & Expositiva/Dialogada \\ \cline{3-5} 
														 &                                                   & 06 & Condensação \& Formação de Nuvens                        & Expositiva/Dialogada \\ \hline
			\multirow{2}{*}{26/04} & \multirow{2}{*}{A Atmosfera como um Gás Ideal}    & 07 & Relação entre Pressão, Volume e Temperatura              & Slides/Simulação     \\ \cline{3-5} 
														 &                                                   & 08 & Derivação da Lei dos Gases Ideais                        & Quadro/Simulação     \\ \hline
			\multirow{2}{*}{28/04} &
			\multirow{2}{*}{\begin{tabular}[c]{@{}l@{}}Conectando os Saberes e\\ Fechamento da Sequência\end{tabular}} &
			09 &
			\begin{tabular}[c]{@{}l@{}}Movimento Atmosférico Global (Células de Hadley) \&\\ Força de Coriolis\end{tabular} &
			Slides/VT03 \\ \cline{3-5} 
																																																											&                                                   & 10 & Equilíbrio Térmico Global \& Forçante Antrópica          & Slides               \\ \hline
		\end{tabular}%
	}
	\caption{Cronograma de aplicação da Sequência didática}
	\label{qua:sequencia-did}
\end{quadro}
% section Planejamento (end)


\section{Análise das Regências} % (fold)
\label{sec:Análise das Regências}
Considerando que nem o tema da Sequência, tampouco a própria seleção da Trilha passou pelo crivo dos estudantes, é possível assumir -- numa perspectiva bem otimista -- que a Sequência performou bem até a quinta aula, em que foi possível "sentir"~o envolvimento emocional de alguns estudantes nas rodadas de perguntas, respostas, diálogos e apresentações de hipóteses.

Ao longo das primeiras aulas teve-se a impressão de que, por tratar-se de uma aula conduzida por alguém a quem ainda não haviam desenvolvido alguma initimidade, sentiram-se inibidos pela metodologia utilizada, uma vez que ninguém gosta de se expor à quem pouco se conhece e ainda com o agravante de ter que responder à questões das quais não se tem muita clareza, todavia a participação de alguns alunos desenvolveu-se consistentemente nas aulas subsequentes e principalmente nas de sexta-feira, sem o grande grupo. Intuitivamente é possível imaginar que numa situação oposta à esta, em que a relação professor-aluno(s) tenha sido construída de maneira mais sólida, a proposta alcance rumos bem mais promissores.

Da quinta aula em diante, o nível de complexidade exigido para abordar do tema somado a falta de destreza de lidar rapidamente com situações em que deve-se repensar a metodologia no decorrer da aula (ou das próximas aulas), foram fatores determinantes para quebrar um pouco desta consistência. Isso também sugere que se faça ajustes na proposta inserindo-se mais aulas em que os alunos possam operacionalizar melhor os conceitos, ao passo que também disponham de maior tempo para os solidificar. 

\subsection{Aulas 001/002} % (fold)
\label{sub:Aulas 001/002}
Na primeira aula deu-se início as atividades da Trilha. Utilizou-se do vídeo VT01 para apresentar o fenômeno e aguçar a curiosidade da turma de forma a promover o envolvimento de todos, esta etapa também teve por objetivos, o de justificar a necessidade do estudo e desenvolvimento de pesquisas a fim de minimizar os impactos gerados pelos ciclones. Durante a mostra, observou-se um bom envolvimento da turma através de suas reações.

Ao final da exposição dedicou-se um tempo para discussões à cerca do exposto, neste momento solicitou-se a participação dos estudantes de forma que expusessem suas impressões livremente, a partir de suas próprias percepções. Como resposta pode-se dizer que houve uma participação expressiva por parte dos estudantes, alguns compartilharam a experiência de parentes que sofreram a passagem do furacão Catarina em 2004 \footnote{Segundo \cite{PEZZA:2005}, primeiro e único registro do fenômeno na porção do Atlântico Sul.}, a maior parte relatou não possuir conhecimentos algum sobre o fenômeno porém consideram relevante o estudo.

As questões geradoras desta discussão versaram basicamente sobre:

\begin{itemize}
	\item O que conhecem sobre o fenômeno;
	\item Se possuem alguma experiência sobre o assunto e que desejam compartilhar;
	\item O que mais chamou a atenção na exposição;
	\item Qual a relevância de se estudar/conhecer sobre o fenômeno;
	\item Dentre outras;
\end{itemize}



Por fim, foi pedido que escrevessem sobre o que foi discutido durante a aula em formato de texto corrido, ressaltando-se os principais pontos que julgaram importantes e/ou interessante sobre o assunto. A atividade valeu nota de participação para os alunos e registro de atividade da Trilha. Foi recolhido um total de 18 respostas para esta atividade.

A segunda aula ocorreu sequencialmente à primeira e após o intervalo de recreio, teve por objetivos iniciar as devidas caracterizações dos fenômenos atmosféricos, para tanto, utilizou-se da entrevista de $11\min$ com o professor Dr. Pedro Dias do Instituto de Astronomia, Geofísica e Ciências Atmosféricas da Universidade de São Paulo \footnote{A entrevista com o professor Pedro Dias (IAG/USP) pode ser livremente acessada em: \url{https://www.youtube.com/watch?v=riXLskNJx20}}. Nesta entrevista é dado o enfoque aos sistemas ciclonais, onde o professor explica de forma precisa e dialogada os principais conceitos; características; causas e efeitos dos ciclones. Como o objetivo principal desta entrevista em si, é o de desmistificar e informar aos espectadores o funcionamento do fenômeno, deixa margens para uma eventual investigação mais detalhada sobre o assunto. Neste sentido fez o uso desta entrevista como fonte norteadora de pesquisas, uma vez que a abordagem adotada encontra-se erguida numa base conceitual apropriada, bem fundamentada e advinda de uma autoridade.

A recepção da entrevista por parte dos alunos, não se mostrou muito eficaz, uma vez que observou-se uma boa parcela de estudantes não atentos à entrevista. Este fato ficou confirmado por meio da não-participação efetiva da turma na rodada de discussões. Nesta etapa, -- a de discussões --  buscou-se destacar nas falas do professor entrevistado as principais carecterísticas do fenômeno, no intuito de eleger-se um conjunto das quais são de suma importância para a construção de um entendimento sobre o assunto, como dito, este mapeamento prévio deve servir como norte para as investigações futuras da turma. Dado a não receptividade desta atividade, coube ao professor estagiário eleger as principais características do fenômeno trazidos superficialmente pela entrevista, discutir brevemente a sua importância para o fenômeno e buscar o consentimento da turma. Os tópicos elencados foram: velocidade dos ventos; fontes de energia do fenômeno; quais as causas e possíveis consequências associadas ao fenômeno. 
% subsection Aulas 001/002 (end)

\subsection{Aulas 003/004} % (fold)
\label{sub:Aulas 003/004}
A terceira e quarta aula da Trilha ocorreram na sexta-feira, dia 14 de abril. Contou com a presença de sete estudantes. Teve por objetivo iniciar o estudo descritivo dos fenômenos atmosféricos a partir do mapeamento elaborado na aula passada. Como primeiro tópico, investigou-se o movimento das massas de ar atmosférico, dentro de um ponto de vista mais geral, para a partir dai compreender-se uma das partes mais significativas e características do fenômeno, os vórtices ar.

A dinâmica em sala de aula, se deu de forma bem fluída e descontraída. Inicialmente, dois alunos não estavam participando ativamente da aula e usavam aparelhos celulares com fones de ouvidos durante, o que dificultou a inclusão destes nas dinâmicas. 

Estipulou-se como problema central o de compreender como ocorrem os deslocamento das massas de ar e levantou-se a questão seguinte: \textit{"Reflitam um pouco e elaborem um esquema de como se formam os ventos? O que faz o ar fluir de um determinado local à outro?"}. Esta questão tem por intuito o de fazê-los refletir à cerca de eventos comuns do cotidiano e que as vezes nem nos questionamos sobre como ocorre, mas que guardam mecanismos desafiadores à compreensão.

Estima-se que quantidade de estudantes em sala (sete no total), favoreceu a aplicação desta atividade que se deu de forma, com já dito, leve e descontraída, apenas como exemplo podemos citar a seguinte situação: Numa etapa de formulação de hipóteses, um estudante questionou se os movimentos de rotação e translação da Terra estariam associados à estas causas \footnote{Sabe-se que o movimento de rotação da Terra, da origem à força de Coriolis, o que de certo modo influência tanto nas regiões de incidência do fenômeno, como na caracterização do sentido de rotação destes sistemas. Já o movimento de translação, também exerce certa influência através do deslocamento das ZCITs no decorrer do ano, porém não pode-se afirmar que o movimento de qualquer massa de ar se da exclusivamente por ventura destes movimentos.}, como este assunto será tratado mais adiante na sequência, fez-se necessário investigar a origem de sua afirmação para dai dar-lhes algum encaminhamento, então procedeu-se da seguinte maneira:

\begin{center}
	\begin{minipage}{.9\textwidth}
		\begin{itemize}
			\item [\textbf{Prof.:}] Mas como assim? Da um exemplo do que você quer dizer... \textit{[Um pouco de dissimulação pra pensar usando o tempo de resposta do aluno à favor do professor, ao passo em que explora-se mais sobre a hipótese do aluno.]}
			\item [\textbf{Aluno:}] Ah! tipo, a Terra ta se movimentando no espaço, dai sei lá, gera o vento!? \textit{[risos]} Não sei bem...
			\item [\textbf{Prof.:}] Você ta dizendo no movimento de translação?
			\item [\textbf{Aluno:}] É, mas acho que pode ser no de rotação também
			\item [\textbf{Prof.:}] hmmm... Entendo!
			\item [\textbf{Prof.:}] O que vocês acham à respeito disso?
			\item [\textbf{A1:}] Eu acho que pode ser
			\item [\textbf{Prof.:}] Ta mas aí o que estaria empurrando o ar? A gente sabe que no espaço tem... ...o quê que tem no espaço?
			\item [\textbf{Todos:}] Nada... Vácuo...
		\end{itemize}
	\end{minipage}
\end{center}


Identificando que a hipótese não havia sido contruída por meio de alguma busca relacionada, e que tratava-se apenas de uma dúvida comum, conduziu-se o grupo de maneira a perceberem a atmosfera terrestre como pertencente à própria Terra e portanto sujeita aos mesmos movimentos executados por ela. Não demorou muito para assimilarem isto e até trouxeram o movimento dos planetas gasosos como uma forma de exemplo.

Outro momento interessante se deu pela súbita participação de um dos alunos que, até então, não havia participado ativamente, ao lembrar do trecho da entrevista em que o professor P. Dias refere-se aos ciclones como \textit{sistemas de baixa pressão atmosférica}, este aluno contribui para a discussão ao propor como ideia chave que: \textit{a região de baixa pressão está de alguma forma associada à velocidade dos ventos do ciclone, no exato olho do furacão deve-se haver a mínima pressão local \footnote{Não foram exatamente esta as palavras do estudante, e sim a ideia que quis passar.}}. Isto levou a turma a considerar esta ideia no debate, o que evoluiu a dinâmica para \textit{como} a pressão pode ser a fonte causadora do deslocamento das massas de ar.

Após o intervalo do recreio, fez-se uma breve explanação sobre o conceito de pressão, explicando de maneira qualitativa a definição do conceito. Sugeriu-se que pensassem, sobre que força atuante e que área deve ser considerada para o caso do movimento atmosférico.

Esta etapa foi bastante desafiadora e gerou muita discussão, termos como: pressão atmosférica; densidade do ar; ar quente; ar frio e etc, foram trazidos à tona pelos estudantes e tomaram o centro das discussões, embora todos estes termos sejam relevantes para o tratamento da situação, apareceram de forma misturadas e confusas dentro do contexto de aula, logo, necessitou-se fazer uma intervenção cercando o termo \textit{pressão}.

Em síntese, após conversarem entre si e pesquisando chegaram a conclusão de que a diferença de pressão é de fato a grande causadora do movimento das massas de ar, mas não tinham noção de como este processo ocorre, e menos ainda nos ciclones.

Para ajudá-los, foi feito esquemas representativos no quadro, em conjunto com uma breve explanação do que é a pressão atmosférica e de como, ocorre a distribuição de moléculas de ar ao longo da altitude, também utilizou-se de exemplos conhecidos do mundo dos esportes para dar suporte as ideias do efeito de rarefação atmosférica, todavia ainda assim, a transposição destas ideias para o fenômeno em estudo não é direta, diversas dúvidas surgiram não sendo possível saná-las na falta de uma unidade própria para tratar só destes conceitos. Evidentemente, o mecanismo completo de deslocamento do ar nestes sistemas é altamente complexo e sofisticado, de modo que o tratamento adequado destes mecanismos ainda que razoavelmente numa abordagem simplificada, requer um certo grau de destreza didática, não dispondo disto, inevitavelmente viu-se o ciclo de perguntas e respostas ceder lugar à rodadas cada vez mais alongadas de exposição e conceituação, tornando os últimos momentos da aula puramente expositivo.
% subsection Aulas 003/004 (end)

\subsection{Aulas 005/006} % (fold)
\label{sub:Aulas 005/006}
A quinta e sexta aula da sequência ocorreram no dia 19/04, uma quarta-feira, com um público maior de alunos (18 alunos). Teve por finalidade discutir o segundo tópico elencado, onde faz-se referência às fontes de energia dos sistemas ciclônicos. Essencialmente buscou-se apresentar os mecanismos de evaporação e condensação da água como os principais agentes responsáveis pelo mantenimento dos sistemas ciclônicos.

Após longa e detalhada revisão da aula anterior, introduziu-se ao tema de estudo com a frase: \textit{"Qual a relação entre o vapor d'água, a temperatura dos oceanos e a formação dos ciclones?"}~ Com isso a temperatura mínima dos oceanos para a ciclogênese foi problematizada, o que segundo a entrevista deve estar, pelo menos, aos $26\Celsius$ ou acima disso. A fim de promover as discussões, perguntou-se: Qual a temperatura de evaporação da água? Os que se prontificaram a responder o fizeram em consenso, afirmando que: "A água ferve aos $100\Celsius$", destas respostas constatou-se que os alunos confundem evaporação com ebulição, conforme destaca \cite{LANG:2016}, ou seja, não deveria haver água no estado vapor/gasoso à temperatura dos $26\Celsius$, o que claramente contradiz as evidências.

Para dar suporte às discussões ponderou-se: "Se a água evapora somente aos $100\Celsius$, como as roupas molhadas secam no varal em dias frios e inclusive chuvosos?"~Poucos manifestaram-se em conceder alguma resposta, em particular, um aluno disse que: \textit{nunca havia se perguntado sobre isto antes}, e um outro, num tom jocoso afirmou que: \textit{o professor faz perguntas "esquisitas", mas que fazem sentido}. Intuiu-se nestes dois exemplos que estas perguntas despertaram uma certa curiosidade no grupo, mas que por falta de uma base conceitual não foi possível a construção de respostas minimamente aceitáveis, o que pode ter inibido a participação do grupo.

Como procedimento, adotou-se uma etapa de recapitulação do conceito de temperatura, onde em alternados turnos de fala buscou-se construir o conceito de \textit{pressão de vapor} qualitativamente. Desenhos e diagramas no quadro serviram de auxílio na elaboração de esquemas explicativos. Nesta etapa foi identificado que os alunos ainda estudariam os mecanismos de troca de energia de maneira sistemática, portanto apenas associou-se ao mecanismo de evaporação a admissão de energia térmica da superfície do oceano para o vapor d'água, e consequentemente para a parcela de ar, por meio da ideia de grau de agitação entre as moléculas, cujo o qual conceitualmente os alunos já estavam mais familiarizados. 

Após o intervalo foi a vez de discutir o processo de condensação da água. Esta aula tomou rumos muito parecidos com a anterior. A dinâmica criada teve por intuito o de fazê-los perceber o processo de condensação como um mecanismo de troca de energia, porém inverso ao de evaporação, para tanto, foi proposto a análise da seguinte situação, bem comum do cotidiano: "Ao colocarmos água gelada dentro de um copo, vemos a formação de pequeníssimas gotas d'água em seu exterior, de onde vem estas gotículas?"~Nesta questão responderam corretamente que as gotículas eram provenientes do ar externo ao copo. Quando feito o seguinte ponderamento já em sequência: "...e de onde saiu a água que estava no ar e que foi parar agora no copo gelado?"~ Responderam: \textit{"Dos rios, oceanos e etc"} indicando alguma retenção das discussões anteriores e/ou de assuntos já estudados.

De forma análoga, fez-se uma exposição explicativa utilizando-se de diagramas e figuras no quadro, para justificar o processo de condensação como um mecanismo de liberação de energia térmica. 

Para conectar estas ideias com o fenômeno estudado, necessitou-se apresentar os conceitos de \textit{convergência} e \textit{divergência}, esta parte se deu basicamente de forma expositiva, ao final associou-se tempos "ruins"~com possibilidades de chuvas à convergências em baixas altitudes ou, equivalentemente, regiões de baixa pressão atmosférica (casos em que ocorrem os ciclônes) e tempos "bons"~com a presença de céu limpo à divergência em baixas altitudes e convergência em altas altitudes, sistemas conhecido como \textit{anti-ciclones}. Aos sistemas de convergência em baixas altitudes elucidou-se como a captação de energia térmica dos oceanos em detrimento da admissão do vapor d'água (mecanismo de evaporação) pelas correntes de ar, já a formação de nuvens (mecanismo de condensação) e divergência em altitudes, com o despojo de energia térmica por meio da condensação da água além, do aparecimento de chuvas torrenciais, situações típicas dos ciclones.
% subsection Aulas 005/006 (end)

\subsection{Aulas 007/008} % (fold)
\label{sub:Aulas 007/008}
A sétima e oitava aula da Trilha ocorreram numa quarta-feira, dia 26 de abril, houve a presença de boa parte da sala. Teve por objetivos o de sistematizar as discussões elaboradas até o momento, bem como o de introduzir um modelo simplificado para o tratamento quantitativo das dinâmicas atmosféricas. Como recursos, utilizou-se apresentações em slides, além de simulações interativas disponibilizadas pelo grupo \ac{PhET}.

Ao contrário das aulas anteriores, a revisão dos conteúdos estudados foi apresentada por meio de slides, e não apenas utilizando o quadro, assim foi possível destacar com melhor ênfase os conceitos obtidos. A participação dos alunos foi oportunizada na etapa de simulações, em que as relações entre os parâmetros $P$, $V$ e $T$, representando respectivamente \textit{a pressão, o volume e a temperatura} de uma determinada quantidade de ar, foram devidamente exploradas por meio de questionamentos como, por exemplo: -- \textit{O que deve ocorrer com a pressão se alterarmos o volume, mas, sem variar a temperatura do gás/ar?} -- os quais eram respondidos espontâneamente e num tom de cooperatividade entre a turma. No geral a atividade performou bem, com destaque a momentos de intensos debates e compartilhamento de ideias no decorrer das rodadas de previsões/testes. A simulação também se prestou para a contrução das interpretações a nível microscópico, uma vez que apropriadamente forneceu-lhes formas de "visualizar"~ o que, em suma, encontrava-se representado pelo ar, ainda que de maneira idealizada.

A aula após o intervalo ficou destinada à enunciar e correlacionar as observações obtidas em cada caso da aula anterior, com as respectivas leis que as representam, a saber: a Lei de Boyle, das transformações isotérmicas; a Lei de Charles  das transformações isobáricas e a Lei de Gay-Lussac das transformações isocóricas. Na sequência estabeleceu-se a equação de estado do gás ideal, dada pela Lei de Clapeyron, cuidando-se na passagem matemática entre as Leis intermediárias ditas anteriormente e a sua formulação, discutiu-se ainda os limites de validade desta, bem como os cuidados a que se deve ter ao tomar este modelo como aproximação para o ar atmosférico. Por fim, aplicou-se uma atividade avaliativa na forma de um questionário \footnote{O questionário aplicado pode ser consultado na seção do \autoref{chap:Avaliação} da página \pageref{chap:Avaliação}} contendo oito questões no total, a entrega do questionário ficou marcada para a próxima aula.
% subsection Aulas 007/008 (end)

\subsection{Aulas 009/010} % (fold)
\label{sub:Aulas 009/010}

As duas últimas aulas da Trilha, ocorreram na sexta-feira dia 28 de abril, com a presença de nove alunos. Teve por objetivos fechar a sequência didática, discutindo a relevância atribuida aos fenômenos atmosféricos na formação do perfil climático do planeta e a importância dos sistemas ciclônicos na distribuição da energia térmica localizada entre os trópicos, contribuindo fortemente para a formação do clima.

A primeira aula foi dedicada a construir-lhes a noção de \textit{aquecimento diferencial} e apresentar os modelos meteorológicos existentes para o movimento de grandes massas de ar, ideias essas fundamentais para o entendimento do fenômeno em um nível sistêmico, para tanto, investigou-se de início a incidência dos raios solares sobre o planeta ao longo de um ano, -- coube neste momento mencionar a origem das estações em decorrência do eixo de inclinação da eclíptica, algo crucial para a formação dos climas. Também foi introduzido uma noção básica sobre o \textit{albedo terrestre}, e as consequências positivas e negativas sobre a concentração de gases do efeito estufa na atmosfera, sendo o vapor d'água um de seus principais agentes. Com estas ideias expostas e utilizando-se imagens contidas nos slide, mostrou-se o perfil médio de aquecimento terrestre, destacando a zona intertropical como a região que mais armazena energia térmica proveniente da radiação solar no decorrer do ano. Partindo-se destas considerações, iniciou-se a elaboração dos modelos climáticos.

Nesta etapa apresentou-se primeiramente o modelo unicelular para o movimento atmosférico a \textit{Célula de Hadley}, proposto por George Hadley em 1735. Fez-se esta exposição tal como ela o foi concebida originalmente, por meio de idealizações. Problematizou-se este modelo a medida em que gradativamente vai-se relaxando as condições idealizadas e confrontando o esperado com o observado, evidenciando as incossitências do modelo. Como alternativa ao modelo de Hadley, apresentou-se o modelo tricelular de Ferrel, proposto por William Ferrel em 1865. Neste modelo a rotação da Terra deve ser considerada, por consequência, este modelo considera naturalmente os efeitos devidos à Força de Coriolis. Neste momento, apresentou-se rapidamente o Efeito de Coriolis \footnote{vídeo VT02 disponível em; \url{https://youtu.be/dt_XJp77-mk}.} e para conceder apoio ao modelo de Ferrel, confrontou-se com imagens em slides as previsões obtidas a partir do modelo para a dinâmica atmosférica, com as imagens observadas, justificou-se as pequenas discrepâncias entre ambas, como decorrentes do aquecimento diferencial. Já para o aquecimento diferencial em si, utilizou-se de imagens demostrando o deslocamento da \ac{ZCIT} para o norte do globo quando este hemisfério aproxima-se das estações mais quentes do ano.

Uma outra evidência discutida nesta aula a favor do modelo de Ferrel, diz respeito a quebra da região de circulação dos ventos alísios e contra-alísios em três regiões distintas do globo (modelo tricelular), este fato sugere a formação de regiões climáticas altamente distintas entre si, em decorrência das características associadas a cada parcela de ar em circulação por cada célula (ar quente e úmido/ar frio e seco), o que corrobora com o observado para as regiões desérticas do globo existindo onde ocorrem a convergência do ar seco e frio em altas altitudes, e portanto, regiões de alta pressão atmosférica e a localização das regiões de grandes florestas e rios em torno do equador, onde há a convergência do ar quente e úmido, porém, em baixas altitudes, ou seja, regiões de baixa pressão atmosférica. Este fato foi abordado por imagens de mapas geográficos de regiões desertos/florestas. 

Para a última aula, buscou-se trabalhar a dimensão social correlata às questões socio-ambientais que cercam a dinâmica atmosférica e em última instância, os eventos climáticos de que decorrem os ciclones. Para tanto, buscou-se consolidar a necessidade do fenômeno para a saúde do sistema climático, haja visto as discussões anteriormente levantadas, questionou-se: -- \textit{"O que podemos fazer à respeito do que vimos nestas aulas?"} -- Como respostas, foi ouvido de alguns alunos: -- \textit{"Nada"}. Questionou-se então: -- \textit{"Mas então o que temos feito?"} Não observando respostas reformulou-se: -- \textit{"Bom, estamos falando que furacões surgem por causa do aquecimento de massas de ar, que por sua vez tem a sua origem, dentre outras tantas coisa, no aumento da temperatura dos oceanos. Tem algo aí que estamos fazendo?"} Desta vez disseram algo como: -- \textit{Aquecimento global... poluição... etc.} Utilizou-se destas falas para discutir as diversas formas de atuar perante a problemática. Em primeiro lugar, apresentou-se a importância de manter-se bem informado principalmente para poder se proteger do fenômeno, ressaltou-se a importância em obedecer aos alertas emitidos pela vigilância civil e apresentou-se alguns dos órgãos responsáveis tanto para a emissão destes alertas, quanto para o monitoramento das atividades atmosféricas em oceano, discutiu-se também o papel da pesquisa e da mídia, alertou-se à cerca dos perigos da disseminação da má informação e do fazer-se repercutir \textit{fake news}, algumas notícias encontradas em sites e em redes sociais foram citadas a título de exemplos. Por fim, apresentou-se dados das projeções para a temperatura do planeta, se absolutamente nada for feito a partir do agora e apresentou-se uma pequena iniciativa \footnote{Carbon Z -- aplicativo que permite realizar o cálculo da emissão de $\text{CO}_{2}$ de pessoas, indústrias e/ou empresas, bem como formas de neutralizar através do plantio de árvores nativas em áreas degradadas. Acessível em: \url{https://carbonzapp.com.br/}} baseada na economia de baixo carbono como ponto de partida para a tomada de consciência a respeito das mudanças climáticas.

% subsection Aulas 009/010 (end)

% section Análise das Regências (end)
% chapter Regências (end)
