\chapter{Considerações Finais} % (fold)
\label{chap:Considerações Finais}
O exercício do Estágio Curricular foi de fundamental importância para o acadêmico no desenvolvimento de atividades típicas de sua futura profissão dentro da realidade do campo de atuação, também contribuiu significativamente para a integração de conhecimentos com vistas à aquisição de competências, no decorrer deste primeiro contato com os desafios da profissão. No campo pedagógico, possibilitou a tomada de ações reais próprias da profissão docente de maneira orientada, supervisionada e avaliativa\footnote{\textit{v. Anexo -- \ref{chap:Avaliação do Professor Supervisor}}}, aprofundando os conhecimentos adquiridos ao longo da vida acadêmica, e o desenvolvimento de atitudes e habilidades em compromisso com a realidade social da profissão.

Ao longo das atividades desenvolvidas no Estágio, foi possível vivenciar de maneira ativa, a forma com que vem se estabelecendo as adequações dos currículos pela reforma do ensino médio, promovida no território catarinense  essencialmente pelas orientações contidas no \ac{CBTCEM}. Viu-se que apesar das diversas dificuldades enfrentadas pelas escolas e professores, há bastante esforços para atender tais exigências, dentro da realidade e limitações em que está inserida a unidade escolar. A partir das análise das aulas do professor supervisor, percebeu-se que diversas das abordagens de ensino desejadas pela reforma, já vem sendo aplicadas como por exemplo: o Ensino por Investigação, o Alinhamento às Dimensões do Conteúdo, o Enriquecimento das Abordagens Discursivas, dentre outras, o que está de acordo com as principais pesquisas atuais em ensino.

Da análise da estrutura escolar, percebeu-se a alocação de esforços direcionados à modernizar-se, como por exemplo: a revitalização das salas de aulas do \ac{NEM} e a revitalização do laboratório de informática, ainda observou-se dificuldades na contratação de profissionais para instalação e manuseio destes materiais/espaços por parte dos órgãos responsáveis, o que deve ser superado muito em breve.

No que compete as atividades de docência, foi possível perceber como os índices educacionais do ensino médio se refletem nas salas de aulas do turno noturno, justificando a necessidade da reforma, no entanto, o aumento da carga horária provocada pela adequação da nova matriz curricular do \ac{NEM}, pode ter sido um dos fatores de maior impacto observado para o esvaziamento das salas de aula deste período, como apontam alguns estudos ao relatar que \textit{``...o aumento da evasão escolar no noturno tende a ser mais alto a partir do final do segundo ano, quando jovens chegam próximo aos 18 anos e sinalizam migração para o Ensino de Jovens e Adultos.''} \cite{SINTE:2023}.

Outro ponto marcante das atividades de docência, ocorreu no momento de construção da Sequência de Ensino em que envolveu a Trilha de Aprofundamento, uma vez que estes componentes propõem assuntos extremamente abrangentes, diferenciados e conectados à diversas áreas do conhecimento, deve-se ter cautela em criar as abordagens e os recortes devidos, sem que inadivertidamente submeta o tema central à execessos o que, ocasionalmente poderá repercutir na construção de um conhecimento, ou demasiadamente superficial ao assunto da Trilha, ou extremamente sofisticada ao nível de formação dos alunos, comprometendo em qualquer um destes casos, a experiência pedagógica apropriada.


%
%


% Não se nasce professor, nem se torna professor. Professores são construídos, dia após dia nas salas de aula da Educação Infantil, do Ensino Fundamental, do Ensino Médio, da Educação de Jovens e Adultos, do Ensino Superior e tantas quantas outras salas de aulas existirem. Professor é uma das poucas profissões que se autoconstrói, nem me recordo de mais alguma. O autoconstruir-se desta profissão é um processo, enfatizo, diário de realizar aquilo que já o é e também o que não se é, ou ainda, aquilo que haverá de ser. A jornada do professor é longa, necessita de humildade, de autoconhecimento e de muito café.
%
% No decorrer de todo este percurso, percebemos a importância do investimento na formação docente. As intempéries da profissão requer, mais do que nunca, que se olhe também para saúde destes profissionais. Neste momento o Brasil passa por mais uma reforma educacional, tal reforma tem sido anunciada desde algum tempo, a necessidade da reforma educacional tem sido associada aos baixos índices de desempenho nos diversos mecanismos de avaliação da educação do país do mundo
%
% Não é de hoje que o Ensino no Brasil é alvo de disputas entre duas concepções de mundo, de um lado defende-se uma educação utilitarista pautada na qualificação para o trabalho, submetida à lógica imediatista do mercado, de outro defende-se uma educação humanista cujo o dever é formar para a cidadania mediada pela tomada de consciência de si como sujeito-histórico, produto do meio em que vive e principal agente de mudanças.
%

% chapter Considerações Finais (end)
