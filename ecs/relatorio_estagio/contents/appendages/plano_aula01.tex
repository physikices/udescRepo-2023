% -----------------------------------------------%
%  HEADER
% -----------------------------------------------%
\thispagestyle{empty}
\begin{center}
	\begin{minipage}[!]{\linewidth}
		\begin{minipage}[!]{.19\linewidth}
			\includegraphics[width=\linewidth]{assets/logo.png}
		\end{minipage}
		\begin{minipage}[!]{.8\linewidth}
			\center
			\ABNTEXchapterfont\normalsize\MakeUppercase{\imprimirinstituicao}
			\par
			\vspace*{10pt}                     
			\ABNTEXchapterfont\normalsize\MakeUppercase{\centro}
			\par
			\vspace*{10pt}           
			\ABNTEXchapterfont\normalsize\MakeUppercase{\disciplina}
		\end{minipage}        
	\end{minipage}
	\\ \vspace{0.5cm}
	\rule{\textwidth}{.5pt}   
\end{center}
%-----------------------------------------------%
% Ficha de Identificação
%-----------------------------------------------%
\textual
\begin{center}
	\section{Plano de Aula: Introdução aos Tópicos da Trilha} % (fold)
	\label{sec:Plano de Aula: Introdução aos Tópicos da Trilha}
\end{center}
\par\noindent\textbf{Estagiário(a):} Rodrigo Nascimento \hfill{}\textbf{Orientadora:} Prof(a). Ana Paula Grimes
\par\noindent\textbf{U.E.:} EEB Giovani P. Faraco\hfill{}\textbf{Supervisor:} Prof. Mário Calegari
\par\noindent\textbf{Série:} 2º Ano\hfill{}\textbf{Turma:} Nº(6)
\par\noindent\textbf{Aula:} 001/002\hfill{}\textbf{Data:} 12/04/2023\hfill{}\textbf{Duração:} $2\times 40\min$
\rule{\textwidth}{.5pt}
%-----------------------------------------------%
% Início do Plano de Aula
%-----------------------------------------------%
\bigskip{}  
\noindent
\begin{center}
	\textbf{Furacões}
\end{center}
\par\noindent\textbf{Resumo da aula:} Esta aula é dedicada à introdução ao tema da trilha: \textit{"Modelagem de Fenômenos Naturais, Sociais e seus Impactos"}, cujo o foco principal é estudar a formação de fenômenos atmosféricos.
\par\noindent\textbf{Habilidades BNCC:} EM13CNT202

\section*{Objetivo de Aprendizagem}
\begin{itemize}
	\item Conhecer e caracterizar fenômenos naturais de origem atmosférica;
	\item Diferenciar os fenômenos atmosféricos conforme as suas variadas classes e definições;
	\item Verificar a ocorrência destes fenômenos inclusive em território catarinense;
\end{itemize}

\medskip{}

\noindent\textbf{Núcleo Conceitual:} \emph{Meteorologia; Fenômenos Naturais; Perturbações Atmosféricas.}
\newpage

\section*{Procedimento Didático} 
\noindent\emph{1º Momento:} Fenômenos Atmosféricos: Furacões.
\par\noindent\rule{.3\textwidth}{.5pt}  
\par\noindent\textbf{Tempo previsto:} 25 minutos

\noindent\textbf{Dinâmica:} Apresentar-se aos alunos e questionar-lhes quais conteúdos relacionados à trilha estão trabalhando em outras disciplinas, explicar que nas próximas oito aulas, usarão boa parte do que aprenderam nas aulas de Física e em outras disciplinas para compreender e modelar um dos fenômenos naturais mais impressionantes e catastróficos do planeta.

Abrir uma discussão com a turma sobre fenômenos atmosféricos, perguntar-lhes o que conhecem sobre o assunto e se já vivenciaram algo que gostariam de compartilhar. Ouvir atentamente o grupo e tentar relacionar com a temática da trilha, espera-se que palavras como: furacões, tornados, tufões, ciclones e etc apareçam espontâneamente, caso isso não ocorra, o professor pode questionar-lhes sobre qual evento atmosférico mais potente de que já viram/ouviram falar, pedir-lhes que comentem sobre este fenômeno.

Fechar esta etapa quando sentir que a turma encontra-se satisfeita com a discussão, prepará-los para a sequência de vídeos em que é mostrado diversas imagens de ciclones e tornados, avisá-los que algumas imagens podem conter cenas severas de destruição de casas e construções, verificar se há consenso entre a turma sobre isso, caso não houver, pular as etapas do vídeo que aparecem tais cenas.

\vspace{50pt}
\noindent\emph{2º Momento:} Vídeo 01 (Furacões e seus Impactos)
\par\noindent\rule{.3\textwidth}{.5pt}    
\par\noindent\textbf{Tempo previsto:} 25 minutos


\noindent\textbf{Dinâmica:} O primeiro vídeo contém 03 min de duração e encontra-se publicado na plataforma YouTube\footnote{https://www.youtube.com/watch?v=uPh4T-9gDJM\&t=3s}. Tem por objetivo, apresentar o fenômeno e chamar a atenção para a importância da pesquisa e desenvolvimento em meteorologia. É possível notar pelas imagens, a predominância de ocorrência do fenômeno no Hemisfério Norte do planeta, principalmente na região dos EUA, este fato revela que este país, além de geograficamente encontrar-se na rota dos ciclones, mantém um centro de pesquisa extremamente ativo para alertar a população e tomar providências o mais rápido o possível a fim de minimizar os danos provocados pelo fenômeno. O poder destrutivo destes sistemas é também explorado pelas imagens justificando a necessidade de pesquisas intensas nessa área de conhecimento.

Após a passagem do vídeo, o professor deve iniciar uma segunda discussão buscando levantar entre os alunos hipóteses de como se formam tais sistemas, a ideia é  mapear entre o grupo as principais concepções a respeito das condições físicas, geográficas, climáticas, sociais e etc, que apresentam-se favoráveis a formação do fenômeno e o tornam tão perigoso. Nesta etapa o professor não deve ajudá-los. Deve tomar nota das hipóteses levantadas antes de passar para o próximo vídeo. As seguintes perguntas podem servir de orientação:

\begin{enumerate}[label=\alph *)]
		\item O que é e onde (continente/oceano/pólos...) se formam os Furacões?
		\item A localização geográfica (Hemisfério Norte/Sul) é relevante na caracterização de furacões?
		\item Para o quê, do ponto de vista da natureza, serve um Furacão?
		\item Que condições são necessárias para a formação de Furacões?
		\item Porquê não se tem registros de ocorrência de Furacões sobre a faixa de latitude $0\Celsius$ (Equador)
		\item Há diferenças entre as terminologias: Furacão, Tufão, Ciclone e Tornado?
		\item Pode ocorrer Furacões no Brasil?
		\item Os efeitos do aquecimento global podem influenciar na ocorrência de Furacões? De que forma? 
\end{enumerate}

O professor deve recolher estas respostas para efeitos de comparação ao final da sequência, além de servir de material para basear as próximas atividades da turma. 

Ceder 10 min da aula para que transcrevam o resultado destas discussões.


\vspace{50pt}
\noindent\emph{3º Momento:} Entrevista com o Professor Pedro Dias IAG/USP (Vídeo)
\par\noindent\rule{.3\textwidth}{.5pt}
\par\noindent\textbf{Tempo previsto:} 25 minutos
\par\noindent\textbf{Dinâmica:} Logo após o recolhimento das respostas acima, passar o vídeo da entrevista com o professor Pedro Dias\footnote{https://www.youtube.com/watch?v=riXLskNJx20} (11 min), PhD em Ciências Atmosféricas e professor titular do Instituto Astronômico e Geofísico da USP. Nesta entrevista, o professor responde às principais questões acerca dos fenômenos meteorológicos que dão origem às pertubações atmosféricas de maneira suscinta, clara e objetiva, fornecendo um panorama geral sobre o funcionamento destes fenômenos, mas sem aprofundar-se muito na fenomemologia descritiva do problema.

Seguindo a sequência, o professor deve discutir em conjunto com a sala os pontos da entrevista que eventualmente podem não ter ficado claro e em conjunto propor a seguinte desafio:

\emph{Prever completamente o comportamento de fenômenos atmosféricos tal como Ciclones e Tornados, é uma tarefa extremamente difícil e complexa, além de figurar como um problema em aberto nas ciências meteorológicas, isso não significa que tudo está perdido, pelo contrário, é uma área que se presta muito bem ao propósito de alertar a população, propor soluções e evitar tragédias. Tal como qualquer outra área do conhecimento, a meteorologia está em contínuo desenvolvimento por meio da pesquisa, modelagens e avanços computacionais.}

\emph{Disciplinas como Física, Matemática e Geografia são essenciais para a construção dos modelos meteorológicos, dito isso, será que com o que sabemos até o momento destas disciplinas, é possível a construção de um modelo básico que dê conta de explicar as principais afirmações dadas pelo professor Pedro Dias?}

Usar o restante da aula para apresentar e discutir a proposta de trabalho com os alunos.
% -----------------------------------------------%
