%-----------------------------------------------%
% Modelo de Plano de Aula com três momentos pedagógicos
%
% Autor: Rodrigo Nascimento (2022-08-12)
%-----------------------------------------------%

\documentclass[
% -- opções da classe memoir --
12pt,				% tamanho da fonte
openright,			% capítulos começam em pág ímpar (insere página vazia caso preciso)
oneside,			% twoside para impressão em verso e anverso. Oposto a oneside
a4paper,			% tamanho do papel. 
% -- opções da classe abntex2 --
chapter=TITLE,		% títulos de capítulos convertidos em letras maiúsculas
%section=TITLE,		% títulos de seções convertidos em letras maiúsculas
%subsection=TITLE,	% títulos de subseções convertidos em letras maiúsculas
%subsubsection=TITLE,% títulos de subsubseções convertidos em letras maiúsculas
% -- opções do pacote babel --
english,			% idioma adicional para hifenização
%	french,				% idioma adicional para hifenização
%	spanish,			% idioma adicional para hifenização
brazil				% o último idioma é o principal do documento
]{abntex2}
\selectlanguage{brazil}
%-----------------------------------------------%
% Informações do DOCUMENTO
%-----------------------------------------------%
\instituicao{Universidade do Estado de Santa Catarina -- UDESC}
\titulo{Estágio Curricular Supervisionado -- IV}
\autor{Rodrigo Nascimento}
\local{Joinville - SC}
\data{\mydate}
\tipotrabalho{Plano de Aula}
\orientador{Prof(a). Dr(a). Ana P. Grimes de Souza}
\coorientador{Prof. Me. Mário Heleno Calegari}
%-----------------------------------------------%
% Para alterar o parâmetros dos comandos orientador
% e coorientador.
%-----------------------------------------------%
% \renewcommand{\orientadorname}{Orientadora:}
\renewcommand{\coorientadorname}{Supervisor:}
%-----------------------------------------------%

\newcommand{\centro}{Centro de Ciências Tecnológicas -- CCT }
\newcommand{\departamento}{Departamento de Física -- DFIS}
\newcommand{\curso}{Licenciatura em Física }
\newcommand{\disciplina}{Estágio Curricular Supervisionado IV-- ESC4003}
\newcommand{\firstkey}{Estágio Supervisionado}
\newcommand{\secondkey}{Ensino de Física}
\newcommand{\thirdkey}{Ensino Médio}


%-----------------------------------------------%

%	Todas as indicações de pacotes e configurações estão no arquivo de estilo
%  chamado texmodel-udesc.sty.
\usepackage{texmodel-udesc}

%-----------------------------------------------%
% Estilo de cabeçalho que só contém o número da 
% página e uma linha
%-----------------------------------------------%
\makepagestyle{cabecalholimpo}
\makeevenhead{cabecalholimpo}{\thepage}{}{} % páginas pares
\makeoddhead{cabecalholimpo}{}{}{\thepage} % páginas ímpares
%\makeheadrule{cabecalholimpo}{\textwidth}{\normalrulethickness} % linha
%-----------------------------------------------%

%-----------------------------------------------%
% HEADER
%-----------------------------------------------%
\begin{document}

\thispagestyle{empty}
\begin{center}
	\begin{minipage}[!]{\linewidth}
		\begin{minipage}[!]{.19\linewidth}
			\includegraphics[width=\linewidth]{img/logo.png}           
		\end{minipage}
		\begin{minipage}[!]{.8\linewidth}
			\center
			\ABNTEXchapterfont\normalsize\MakeUppercase{\imprimirinstituicao}
			\par
			\vspace*{10pt}                     
			\ABNTEXchapterfont\normalsize\MakeUppercase{\centro}
			\par
			\vspace*{10pt}           
			\ABNTEXchapterfont\normalsize\MakeUppercase{\disciplina}
		\end{minipage}        
	\end{minipage}
	\\ \vspace{0.5cm}
	\rule{\textwidth}{.5pt}   
\end{center}
%-----------------------------------------------%
% Ficha de Identificação
%-----------------------------------------------%
\textual
\begin{center}
	\textbf{Plano de Aula: Fechamento da Sequência}
\end{center}
\par\noindent\textbf{Estagiário(a):} \imprimirautor\hfill{}\textbf{Orientadora:} Prof(a). Ana Paula Grimes
\par\noindent\textbf{U.E.:} EEB Giovani P. Faraco\hfill{}\textbf{Supervisor:} Prof. Mário Calegari
\par\noindent\textbf{Série:} 2º Ano\hfill{}\textbf{Turma:} Nº(6)
\par\noindent\textbf{Aula:} 009/010\hfill{}\textbf{Data:} 28/04/2023\hfill{}\textbf{Duração:} $2\times 40\min$
\rule{\textwidth}{.5pt}
%-----------------------------------------------%
% Início do Plano de Aula
%-----------------------------------------------%
\bigskip{}  
\noindent
\begin{center}
	\textbf{Movimentos Atmosféricos em Escalas Globais e Ciclogênese}
\end{center}
\par\noindent\textbf{Resumo da aula:} Neste encontro dar-se-á o fechamento da sequência apresentando o principal modelo explicativo do movimento global das massas de ar. Tal modelo não somente faz uso dos conceitos abordados em aulas anteriores como também demonstra a complexidade e necessidade de tais movimentos para a formação/manutenção do clima; para a dissipação da energia térmica armazenada em torno da ZCIT -- \textit{Zona de Convergência Intertropical} e o papel dos sistemas Ciclônicos dentro deste contexto. Ao final será apresentado e discutido ainda a Forçante Antrópica como agente catalizador do aquecimento global.
\par\noindent\textbf{Habilidades BNCC:} EM13CNT105; EM13CNT206; EM13CNT309.

\section{Objetivo de Aprendizagem}
\begin{itemize}
	\item Conhecer os principais motivos que originam os movimentos das grandes massas de ar no planeta; 
	\item Relacionar os diferentes movimentos atmosféricos globais, através do modelo das três células \textit{Célula de Hadley; Célula de Ferrel; Célula Polar};
	\item Perceber os Ciclones como mecanismo natural de distribuição da energia térmica acumulada.
\end{itemize}

\medskip{}

\noindent\textbf{Núcleo Conceitual:} \emph{Geofísica; Movimentos Atmosféricos Globais; Ciclones}
\newpage

\section{Procedimento Didático} 
% --------------------------------------------- %
\noindent\emph{1º Momento:} Movimentos atmosféricos e as zonas de convergências
\par\noindent\rule{.3\textwidth}{.5pt}  
\par\noindent\textbf{Tempo previsto:} 25 minutos

\noindent\textbf{Dinâmica:} Iniciar a exposição recordando-os brevemente sobre os tópicos discutidos e investigado no decorrer da sequência, como por exemplo:
\begin{itemize}
		\item Caracterização do Fenômeno; seus impactos e locais de ocorrência \textit{(Aulas: 001/002)}
		\item A influência da temperatura na pressão e por consequência na umidificação das camadas de ar próximas ao oceano \textit{(Aulas 003/004 e 005/006)}
		\item A relação das variáveis termodinâmicas no estudo e caracterização da atmosfera terrestre dentro do modelo do gás ideal \textit{(Aulas 007/008)}
\end{itemize}
Após a etapa acima, informar-lhes que nesta exposição, faremos uma breve explanação sobre como todos estes conceitos são utilizados na meteorologia, com a finalidade de auxiliar os sistemas de monitoramento; diagnóstico; classificação e redução de danos.

Abrir os slides a partir da seção entitulada: \textit{"Movimentos Afmosféricos -- Escala Global"} e discorrer sobre o seu conteúdo de forma dialogada com os estudantes.

Pontos chave da exposição:
\begin{enumerate}[label=\alph *)]
		\item Aquecimento diferencial
			\begin{itemize}
					\item Causas astronômicas
					\item Causas geomorfológicas
			\end{itemize}
		\item Modelos para o movimento atmosférico
			\begin{itemize}
					\item Modelo unicelular
					\item Modelo de três células
			\end{itemize}
		\item O efeito de coriolis como o responsável pela deflexão das grandes massas de ar
		\item Observáveis a favor do modelo das três Células
			\begin{itemize}
					\item Zonas de convergências e a ZCIT
					\item Os ventos alísios e contra-alísios
					\item Observações por satélites geoestacionários
					\item Regiões de maior ocorrência dos desertos/florestas de taiga (climas secos)
					\item Regiões de ocorrências das florestas úmidas (climas úmidos)
			\end{itemize}
\end{enumerate}
Para auxiliar nesta apresentação, usar o vídeo \href{https://youtu.be/dt_XJp77-mk}{"Coriolis Effect-MIT" }sobre o efeito coriolis e cortes do vídeo \href{https://youtu.be/8w3o6_cn-O8}{"A Year of Weather - 2019" }em que aparecem as zonas de convergência; o perfil das massas de ar e a formação de sistemas ciclônicos ao redor do mundo, num \textit{timelapse} produzido por satélites geostacionários no decorrer de um ano.

Seguindo com a sequência, citar exemplos da importância destes movimentos para o planeta, tendo como um de seus mecanismos mais eficazes os Ciclones.

\vspace{50pt}
% --------------------------------------------- %
\par\noindent\emph{2º Momento:} Ciclones
\par\noindent\rule{.3\textwidth}{.5pt}    
\par\noindent\textbf{Tempo previsto:} 15 minutos

\noindent\textbf{Dinâmica:} Usando os slides, caracterizar os ciclones conforme suas propriedades, classificações e regiões de ocorrência. Salientar que dentre os três tipos de Ciclones existentes, estudamos apenas o mais típico dentre eles -- O Ciclone Tropical, os demais diferem-se deste por: Região de formação e associação com frentes frias. Ressaltar ainda que não há uma maneira simples de prever se um fenômeno como este irá ou não desenvolver-se até a costa e sua previsibilidade tem por base os inúmeros sistemas de monitoramento contínuo sobre as regiões de baixa pressão nos oceanos (isóbaras), bem como de fatores sazonais que produzem o aquecimento diferencial da atmosfera e das forças equilibradoras atuantes na formação destes sistemas, a saber, as forças do gradiente de pressão; a "força" de coriolis e a força de cisalhamento do vento na vertical.

Usar trechos do vídeo "A Year of Weather" para apresentar o giro característico do sistema ciclonais e anticiclonais no hemisfério norte e sul, diferenciando-os conforme o sentido de rotação em cada caso.

Aproveitar os trechos do vídeo para explorar as regiões de maior ocorrência do fenômeno no globo, além do mapa da NASA (slide da página 26) em que há o registro compilado de todas as trajetórias dos ciclones desde 1985 à 2005.

Chamar-lhes a atenção para que vejam a incidência do fenômeno no Brasil. Justificar a baixa incidência do fenômeno em território nacional como decorrente, dentre outros tantos fatores, do enfraquecimento do aquecimento diferencial e da baixa amplitude térmica anual sobre oceanos.

Justificar a não ocorrência de sistemas ciclônicos exatamente sob a linha do equador em virtude da força de coriolis, nula no equador e máxima nos pólos.

\vspace{50pt}
% --------------------------------------------- %
\par\noindent\emph{3º Momento:} A forçante antrópica
\par\noindent\rule{.3\textwidth}{.5pt}
\par\noindent\textbf{Tempo previsto:} 35 minutos

\par\noindent\textbf{Dinâmica:} Este momento deve ser iniciado na seção -- "O que fazer?" (slide de pág. 27). Para dar início o professor pode questionar se com o que viram até então, seriam os sistemas ciclônicos puramente vilões ou parte de um todo absolutamente necessário para o planeta?

A fim de orientar esta discussão o professor pode basear-se nas seguintes proposições e/ou questões:
\begin{enumerate}[label=\alph *)]
		\item Vimos que os movimentos atmosféricos tem a sua origem nas diferenças de temperatura sobre o globo e que isso decorre de causas ainda mais primodiais de que não temos controle, como a inclinação da eclíptica que gera as diferentes estações do ano e consequentemente um aquecimento de porções diferentes do planeta;
		\item Vimos ainda que as porções continentais distribuidas sobre o globo, influência de certa forma no aquecimento dos oceanos;
		\item O que deveria ocorrer se apesar de tudo isso, não fosse de alguma forma possível, a existência destes movimento atmosférico?
		\item Será que o clima seria melhor, pior, será que a vida como conhecemos seria possível?
		\item De alguma forma vocês conseguem enxergar alguma vantagem sobre essa dinâmica do movimento atmosférico para manutenção e formação dos diversos climas existentes no planeta?
		\item Qual o papel dos sistemas ciclônicos neste contexto?
\end{enumerate}
A discussão deve encaminhar-se para o conformidade de que tais sistemas são precisamente necessários para o balanço e distribuição da energia térmica sobre o globo, do contrário a acentuação do sistema climático poderia inviabilizar a vida e a biodiversidade pelo menos da maneira como a conhecemos.

As discussões anteriores subsidiam os slides finais que discorrem sobre -- O que fazer frente ao fenômeno?

De início discutir com o grupo o que definitivamente não fazer. Usar como recurso os dois contraexemplos citados a seguir: 

\textbf{Contraexemplo 01:} \textit{"Everbody Points Their Fans at The Hurricane to Blow it Away"}, questionar-lhes de que forma esta ação é passível de obter-se êxito, dado o estudo aqui abarcado.
\par\textbf{Contexto:} Um grupo de pessoas organizaram-se nas redes sociais, com o intuito de combater a passagem do furacão Ian (cat 4) na costa da Florida em 2022, usando-se de ventiladores\footnote{\url{https://www.facebook.com/events/281828308983096/?active_tab=discussion}}. A ideia do grupo é que quanto mais pessoas aderirem ao movimento e ligarem seus ventiladores na direção do ciclone, esta ação poderia dissipar o fenômeno. Relata-se na matéria que 60 mil estadunidenses aderiram ao movimento.

\textit{Poderamentos:} É provável que esta matéria trate apenas de um \textit{meme}, algo que é bem comum nos dias atuais, também é possível que esta adesão ao movimento, tenha dado-se de forma apenas extrovertida e não legitimada, no entanto, vale ressaltar que não há qualquer respaldo científico que justifique tal ação e que dada a repercursão, alguns mais incautos podem tomá-la por verdade pondo-se em risco, é fundamental o uso consciente das redes sociais de forma responsável. Se criador de conteúdos digitais, cuidar para a informação correta da população de maneira que não gere ambiguidades nem a propagação de falácias ou as famosas \textit{fake news}. Se consumidor de conteúdos digitais, sempre verificar as fontes e buscar fontes confiáveis. No Brasil, o órgão oficial que emite os alertas sobre ciclones é a Marinha em conjunto com a Defesa Civil. 

\textbf{Contraexemplo 02:} Em uma publicação\footnote{\url{https://shre.ink/QTXp}} datada de 26 de agosto do ano de 2019, o jornal Folha de São Paulo afirma que o então presidente dos Estados Unidos, Sr. D. Trump \textit{"Cogitou usar bombas nucleares contra furacões"}. Assim como a ação anterior, questionar aos alunos sobre a possibilidade desta ação obter êxito.
\par\textbf{Contexto:} Na matéria o jornal descreve que D. Trump perguntou aos seus assessores sobre a possibilidade de usar armas nucleares para "destruir" furacões antes mesmo de chegarem à costa, é citado na matéria o site Axios\footnote{\url{https://www.axios.com/2019/08/25/trump-nuclear-bombs-hurricanes}} como fonte da informação.

\textit{Ponderamentos:} Neste contraexemplo vê-se que a questão climática é tão delicada que mesmo líderes globais podem apresentar um profundo desconhecimento sobre as suas causas e sugerir ações que podem inclusive piorar ainda mais a situação. Para este caso em específico, o então mandatário foi rebatido com a resposta da agência responsável pelo monitoramento e alertas das condições atmosféricas e oceânicas de seu pais, a National Oceanic and Atmospheric Administration (NOAA).

Por fim, citar brevemente as medidas de minimização de danos e proteção à vida (slides de página 29). Reservar o final da aula para apresentar as principais a ações causadoras da forçante antrópica, suas evidências e projeções futuras caso nada seja feito.

%-----------------------------------------------%
% Referências
%-----------------------------------------------%
% \bibliography{bibliografia.bib}
%-----------------------------------------------%
% Anexos
%-----------------------------------------------%
\begin{anexosenv}		    
	\chapter{Slides}
	Esta aula conta com uma apresentação em slides como recurso de didático, esta apresentação contém exatas 13 lâminas que podem ser consultadas nas páginas a seguir.
	\includepdf[pages={23-57},nup=2x5]{beamer.pdf}
\end{anexosenv}
%-----------------------------------------------%
% Fim do Plano de Aula
%-----------------------------------------------%
\end{document}
