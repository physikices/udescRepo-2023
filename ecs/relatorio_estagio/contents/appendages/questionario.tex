\begin{quest}
	Classifique os fenômenos meteorológicos estudados (Ciclones/Tornados/Furacões e etc.) conforme as nomenclaturas; velocidade dos ventos; regiões de ocorrências.
\end{quest}

\begin{quest}
	Qual a importância dos ciclones para o clima da terra?		
\end{quest}

\begin{quest}
	Explique como se formam os ventos de maneira geral.
\end{quest}

\begin{quest}
	O que o movimento global das massas de ar tem a ver com os ciclones?
\end{quest}

\begin{quest}
	Por que a maior densidade da atmosfera ocorre próximo à superfície da Terra?		
\end{quest}

\begin{quest}
	Quando a densidade permanece constante e a temperatura sobe, como variará a pressão de um gás?		
\end{quest}

\begin{quest}
	Quando gases na atmosfera são aquecidos a pressão do ar normalmente cai. Comparando com a sua resposta à questão anterior, explique este aparente paradoxo.
\end{quest}

\begin{quest}
	Qual a massa que exerce maior pressão na superfície: uma massa úmida e quente? ou uma massa fria e seca?
\end{quest}
