% --------------------------------------------- %
%  HEADER
% --------------------------------------------- %
\thispagestyle{empty}
\begin{center}
	\begin{minipage}[!]{\linewidth}
		\begin{minipage}[!]{.19\linewidth}
			\includegraphics[width=\linewidth]{assets/logo.png}
		\end{minipage}
		\begin{minipage}[!]{.8\linewidth}
			\center
			\ABNTEXchapterfont\normalsize\MakeUppercase{\imprimirinstituicao}
			\par
			\vspace*{10pt}                     
			\ABNTEXchapterfont\normalsize\MakeUppercase{\centro}
			\par
			\vspace*{10pt}           
			\ABNTEXchapterfont\normalsize\MakeUppercase{\disciplina}
		\end{minipage}        
	\end{minipage}
	\\ \vspace{0.5cm}
	\rule{\textwidth}{.5pt}   
\end{center}
%-----------------------------------------------%
% Ficha de Identificação
%-----------------------------------------------%
\textual
\begin{center}
	\section{Plano de Aula: Mecanismos de Troca de Energia} % (fold)
	\label{sec:Plano de Aula: Mecanismos de Troca de Energia}
	% section Plano de Aula: Mecanismos de Troca de Energia (end)
\end{center}
\par\noindent\textbf{Estagiário(a):} Rodrigo Nascimento\hfill{}\textbf{Orientadora:} Prof(a). Ana Paula Grimes
\par\noindent\textbf{U.E.:} EEB Giovani P. Faraco\hfill{}\textbf{Supervisor:} Prof. Mário Calegari
\par\noindent\textbf{Série:} 2º Ano\hfill{}\textbf{Turma:} Nº(6)
\par\noindent\textbf{Aula:} 005/006\hfill{}\textbf{Data:} 19/04/2023\hfill{}\textbf{Duração:} $2\times 40\min$
\rule{\textwidth}{.5pt}
%-----------------------------------------------%
% Início do Plano de Aula
%-----------------------------------------------%
\bigskip{}  
\noindent
\begin{center}
	\textbf{Evaporação e Condensação da Água}
\end{center}
\par\noindent\textbf{Resumo da aula:} Uma outro mecanismo de fundamental importância para os sistemas de baixa pressão, é resultante da capacidade de mover grandes massas de ar úmido da superfície oceânica e transformá-las em nuvens na troposfera, tal mecanismo é de suma importância para a sobrevivência destes sistemas. Nesta aula, investigaremos os processos de evaporação e condensação da água como mecanismos responsáveis pelas trocas de energia térmica entre sistemas. Como recurso pedagógico, partir-se-á das falas do professor P. Dias traduzidas em situações-problemas  comuns ao universo do estudantes.
\par\noindent\textbf{Habilidades BNCC:} EM13CNT202

\section*{Objetivo de Aprendizagem}
\begin{itemize}
	\item Perceber os processos de convergência do ar e evaporação da água como elementos fundamentais para a estrutura dos ciclones;
	\item Relacionar os processos de condensação com a formação de nuvens;
	\item Reconhecer a evaporação/condensação da água como mecanismos de troca de energia;
\end{itemize}

\medskip{}

\noindent\textbf{Núcleo Conceitual:} \emph{Mecanismos de Troca de Energia; Evaporação; Condensação}
\newpage

\section*{Procedimento Didático} 
\noindent\emph{1º Momento:} Evaporação da água
\par\noindent\rule{.3\textwidth}{.5pt}  
\par\noindent\textbf{Tempo previsto:} 25 minutos

\noindent\textbf{Dinâmica:} Iniciar a aula revisando os principais pontos da aula passada, sendo estes os seguintes:
\begin{enumerate}[label=\alph *)]
		\item A influência da temperatura na mudança de pressão em colunas de ar;
		\item O movimento horizontal do ar devido ao gradiente de pressão, ocorrendo sempre no sentido das regiões de alta pressão atmosférica para a região de baixa pressão (Advecção);
		\item O movimento vertical do ar devido a diferença de densidade (Convecção). 
\end{enumerate}

Feito isso, usar a contiunação da fala do Professor Pedro Dias (IAG/USP) para iniciar o estudo de outro "ingrediente" indispensável para a formação dos Ciclones: O vapor da água dos oceanos.

A fala em destaque é:

\begin{itemize}
	\item[--] \textbf{Jornalista:} Como se formam os furacões?
	\item[--] \textbf{Prof. Pedro Dias:} \textit{[...] continuação...} a força de um furacão vem do calor da água e da evaporação da água, no continente não tem tanta evaporação então, basicamente, você corta o combustível do furacão que é o vapor da água. [Explicando o porque do fenômeno não ocorrer em continente, e dissipar-se rapidamente quando alcança a região litorânea.]
\end{itemize}

Assim que os alunos recordarem desta fala, destacar a parte em que ele cita o vapor d'água como combustível essencial para o fenômeno, questionar-lhes o que ele quer dizer com isso, questionar-lhes qual a relação entre vapor d'água, temperatura e pressão lembrando que a temperatura do oceano na ocorrência de Ciclones é na faixa dos $26\Celsius$.

\noindent\textbf{Obs:} Conforme indicam as pesquisas\footnote{\cite{LANG:2016}} normalmente os alunos tendem a confundir o conceito de Ebulição com o conceito de Evaporação, caso isto ocorra, dedicar um tempo da aula $(\leq 10\min)$ para diferenciar o processo de evaporação do processo de ebulição. Usar como situação-problema a afirmativa:

\begin{itemize}
		\item Como a roupa seca? Será que nesta situação a água alcança o ponto de ebulição?
\end{itemize}

Conduzir a discussão até que se obtenha respostas satisfatória.

Por resposta satisfatória, entende-se que: Quanto maior a temperatura da água, maior será a energia associada às moléculas do líquido, tal condição permite que moléculas da água escape para o ar na interface que une os dois meios, isto ocorre constantemente a qualquer faixa de temperatura abaixo do ponto de ebulição e é limitado pela saturação do vapor d'água no ar. Quando o ar encontra-se saturado, mais difícil é o escape das moléculas de água para o ar e toda molécula excedente é admitida na forma condensada (ponto de orvalho). A pressão e a temperatura alteram o ponto de orvalho, quanto menor a pressão interna, menor o ponto de orvalho.

Para conectar-se novamente à fala do professor, ressaltar que regiões de baixa pressão atmosférica, facilita o processo de admissão do vapor d'água e como o vapor d'água possui menor densidade que o ar seco, essa admissão reforça ainda mais o fenômeno de convecção.

\vspace{50pt}
\noindent\emph{2º Momento:} Expansão adiabática
\par\noindent\rule{.3\textwidth}{.5pt}    
\par\noindent\textbf{Tempo previsto:} 25 minutos


\noindent\textbf{Dinâmica:} Conduzir a estória da parcela de ar quente e agora úmido no interior do centro de baixa pressão, questionar-lhes o que esperam que ocorra a medida que este vapor vai subindo pela região. Caso apresentem dificuldades, transpor o problema para o caso de um balão de festa preenchido por um gás de hélio (simplificar primeiramente sem o ar úmido), fazê-los notar que não existe diferença entre ambas as situações exceto que no caso do balão há paredes limitando o ar de escapar, mas o que ocorreria se não houvesse estas barreiras? O que deve ocorrer com a temperatura neste caso e porque?

Propor que transcrevam para o papel suas hipóteses, destinar $10\min$ para esta atividade.

Passar a transcrição dos alunos para o quadro e discuti-las em conjunto, a intenção é fazê-los perceber que o balão subirá até estourar, mas se não houvesse "paredes" iria continuar aumentando de volume a taxas cada vez menores conforme a altura. A mudança de volume acarreta na diminuição da temperatura interna do balão. 

\newpage
\vspace{50pt}
\noindent\emph{3º Momento:} Transferência de calor latente para a atmosfera alta
\par\noindent\rule{.3\textwidth}{.5pt}
\par\noindent\textbf{Tempo previsto:} 25 minutos
\par\noindent\textbf{Dinâmica:} Sugerir que agora pensem no balão com vapor de água. O que deve ocorrer com o vapor d'água neste caso? Pedir para que pensem novamente no problema e reformulem suas hipóteses no papel. Se ainda assim houver dificuldades para compreender que a água deverá sofrer o processo de condensação, transpor o problema para as duas situações seguintes:
\begin{itemize}
		\item O que ocorre quando coloca-se água gelada em copo à temperatura ambiente? E porque?
		\item E se colocassemos água quente? Muda alguma coisa? Explique
\end{itemize}



Conduzir a discussão até perceberem que a medida que o ar quente sobe e expande adiabáticamente por diferença de pressão (sem troca de energia com o exterior), no interior do balão, este processo é acompanhado da diminuição da temperatura interna do balão, quando a temperatura interna for suficientemente pequena, ocorrerá a condensação do vapor d'água devido à temperatura do ponto de orvalho, este processo é acompanhado de mudança de fase o que transfere energia para a atmosfera em níveis altos na forma de calor latente, neste momento o ar atmosférico ao receber esta energia tenderá a relizar o movimento de divergência (nova advecção), e assim torna-se possível a retroalimentação de todo o sistema. 

