\chapter{Introdução} % (fold)
\thispagestyle{empty}
\label{chap:Introdução}

\lettrine{O} termo estágio sofreu diversas mudanças desde o seu surgimento no século XI aos dias atuais \cite{COLOMBO:2014}. Atualmente, está associado ao período em que o aprendizado adquirido ao longo da formação acadêmica é colocado em prática em um ambiente controlado, com supervisão técnica apropriada, visando promover o desenvolvimento de competências profissionais e a contextualização curricular, de acordo com a Lei nº 11.788 de 25 de setembro de 2008 \cite{BRASIL:2008}.

No contexto educacional, o estágio na formação docente é fundamental para a experiência de práticas pedagógicas e a busca por metodologias que estimulem os estudantes. É nessa etapa que o acadêmico pode vivenciar a realidade escolar, enfrentar os principais desafios e dificuldades inerentes à profissão docente.

A chegada da era digital e a disseminação dos meios de comunicação intensificaram um problema já amplamente explorado pelas pesquisas educacionais, trata-se da -- ineficiência das metodologias de ensino \textit{passivas} na construção de um conhecimento capaz de lidar com o novo paradigma emergente \cite{MORAES:2018,DIESEL:2017}. Tese que ganha força ao analisarmos as estatísticas que monitoram o desempenho educacional do país. Apenas para ilustrar a dimensão do problema, de acordo com os resultados oficiais das avaliações aplicadas na rede pública de ensino, o percentual de alunos com aprendizagem adequada em língua portuguesa no 5º ano do Ensino Fundamental evoluiu de 35,6\% em 2011 para 56,5\% em 2019; nos Anos Finais, de 21,8\% para 35,9\%; e no Ensino Médio, de 23,3\% para 31\% no mesmo período\footnote{Dados do Anuário Brasileiro da Educação Básica}. Apesar desses dados representarem recentemente uma suave tendência de melhora, o problema é antigo, tendo como um marco referencial a Conferência Mundial da Educação para Todos, realizada em Jomtien, na Tailândia no ano de 1990, durante a ocasião o país adota a agenda da \ac{UNESCO} firmando compromissos para com a educação nacional, o que culminou no estabelecimento de dez diretrizes e vinte metas para o setor a serem cumpridas e revisitadas a cada decênio \cite{BRASIL:2001}, inaugurando-se o primeiro \ac{PNE}.

\section{Referenciais Normativos} % (fold)
\label{sec:Referenciais Normativos}
A partir da aprovação do \ac{PNE}, o poder público tem implementado uma série de instruções normativas responsáveis por orientar e regulamentar toda a base educacional do país. A discussão e exposição detalhada sobre cada um destes dispositivos foge ao escopo deste trabalho, no entanto, o leitor interessado pode consultá-las através do portal \href{https://www.jusbrasil.com.br/}{jusbrasil}. A seguir, apresentaremos apenas as instruções mais pertinentes à compreensão do que aqui pretende-se expor.

\subsection{A BNCC} % (fold)
\label{sub:A BNCC}
Prevista pela \ac{LDB} e pelo \ac{PNE}, a \ac{BNCC} é uma política nacional que constitui-se enquanto um documento normativo que:

\begin{citacao}
	``define o conjunto orgânico e progressivo de aprendizagens essenciais que todos os alunos devem desenvolver ao longo das etapas e modalidades da Educação Básica, de modo a que tenham assegurados seus direitos de aprendizagem e desenvolvimento, em conformidade com o que preceitua o \ac{PNE}.'' \cite[pp.~7]{BRASIL:2018}
\end{citacao}
nele tem-se estabelecido dez competências gerais, para que a educação seja capaz de: 

\begin{citacao}
	``afirmar valores e estimular ações que contribuam para a transformação da sociedade, tornando-a mais humana, socialmente justa e, também, voltada a preservação da natureza.'' \Ibidem[pp. ~7]{BRASIL:2017}
\end{citacao}
ao definir esse conjunto de competências a se desenvolver ao longo de toda Educação Básica (Educação Infantil, Ensino Fundamental e Ensino Médio), a \ac{BNCC} adota o enfoque das avaliações internacionais da \ac{OCDE} e da \ac{UNESCO} que insitituiu o \ac{LLECE}, reafirmando assim seu compromisso com a agenda internacional para a educação.
% subsection A BNCC (end)

\subsection{BNCC -- Ensino Médio} % (fold)
\label{sub:A BNCC e o Ensino Médio}
Historicamente o \ac{EM} configura-se como a etapa da formação mais problemática conforme indicam as estatística, para esta etapa a \ac{BNCC} reconhece como desafios,  \textit{``além da necesidade de universalizar o atendimento, [...] garantir a permanência e as aprendizagens dos estudantes, respondendo às suas demandas e aspirações futuras\footnote{Cf. \Ibidem[pp.~461]{BRASIL:2017}}''}. Para superar esses desafios, aponta para uma organização escolar que acolha as diversidades e que garanta aos estudantes o protagonismo de seu próprio processo de escolarização, neste sentido entende a escola como um espaço de acolhimento das juventudes, \textit{``[...] comprometida com a educação integral dos estudantes e com a construção do seu projeto de vida\footnote{Cf. \Ibidem[pp.~464]{BRASIL:2017}}''}. Assim orienta para a construção de currículos que visam a superação do modelo único e fragmentado, por um modelo \textit{diversificado} e \textit{flexível}, reconhecendo a alteração dada pela Lei de nº 13.415/2017, em que estabelece o currículo do \ac{EM} composto em uma parte pela \textit{Base Nacional Comum Curricular} e outra pelos \textit{Intinerários Formativos} organizados por meio da oferta de diferentes arranjos curriculares, conforme a relevância para o contexto local e a possibilidade dos sistemas de ensino \cite{BRASIL:2017b}. As aprendizagens essenciais definidas pela \ac{BNCC} para a etapa relacionada ao \ac{EM} estão organizadas por áreas do conhecimento, a saber:

\begin{enumerate}[label=\Roman *]
		\item – linguagens e suas tecnologias;
		\item – matemática e suas tecnologias;
		\item – ciências da natureza e suas tecnologias;
		\item – ciências humanas e sociais aplicadas;
		\item – formação técnica e profissional.
\end{enumerate}

Pra cada área do conhecimento, fica subentendida competências específicas articuladas às suas respectivas versões no Ensino Fundamental, dando o caráter de contuinuidade entre as diferentes etapas da formação. A cada uma das competências, são descritas habilidades a serem desenvolvidas ao longo da etapa relacionada. O texto ainda define o limite total de $1800$ horas a cumprir como carga horária destinada a formação geral básica relativa a Base Nacional Comum Curricular.

Compete aos Estados e entes Federativos a tarefa de elaborarem as suas propostas e junto às redes de ensino estabelecerem estratégias para a sua implementação, a fim de promover a reforma educacional em todos os seus ditames e em pleno regime de colaboração com a União.
% subsection A BNCC e o Ensino Médio (end)

\subsection{A PCSC} % (fold)
\label{sub:A PCSC}
Santa Catarina é um estado pioneiro na elaboração de Propostas Curriculares, tendo a sua primeira proposta formulada entre os anos 1988 e 1991 \cite{CATARINA:1991}. Em atenção à \ac{BNCC}, a \ac{PCSC} para a Educação Básica orienta-se por três fios condutores: 

\begin{citacao}
	``1) perspectiva de formação integral, referenciada numa concepção multidimensional de sujeito; 2) concepção de percurso formativo visando superar o etapismo escolar e a razão fragmentária que ainda predomina na educação curricular e 3) atenção à concepção de diversidade no reconhecimento das diferentes configurações identitárias e das novas modalidades da educação.'' \cite[pp.~20]{PCSC:2014}
\end{citacao}
Nestas linhas a \ac{PCSC} aproxima-se de teorias como: a Teoria do Desenvolvimento Psicológico de Lev Vygostsky, ao conceber a formação integral como maneira de abordar todas as dimensões do desenvolvimento humano, incluindo aspectos cognitivos, sociais e emocionais\footnote{Cf. \Ibidem[pp.~36]{PCSC:2014}} sendo perpassadas como um \textit{continuum} ao longo da vida escolar na ideia de \textit{percurso formativo}, nele é preciso \textit{``[...]transcender os componentes curriculares das áreas e suas especificidades promovendo o diálogo com os diferentes aspectos da cultura''}.

\subsection{O CBTCEM} % (fold)
\label{sub:O CBTCEM}
Acompanhando o cenário nacional o Ensino Médio Catarinense é reconhecidamente a etapa mais complexa e problemática da formação, necessitando de atenção redobrada na elaboração de políticas educacionais. A solução proposta pelo Estado à essa etapa formativa, vem acompanhada de mudanças drásticas em todas as esferas, a começar pela carga horária que passa das 800 horas para 1000 horas anuais no ano de 2022, devendo ser ampliada para 1400 horas de forma progressiva.

O \ac{CBTCEM} encontra-se organizado em seis cadernos, sendo eles:
\begin{enumerate}[label=\Roman *)]
		\item \textbf{Caderno 1} -- Disposições Gerais;
		\item \textbf{Caderno 2} -- Formação Geral Básica;
		\item \textbf{Caderno 3} -- Portifólio de Trilhas de Aprofundamento;
		\item \textbf{Caderno 4} -- Componente Curriculares Eletivos: Construindo e Ampliando os Saberes - Portifólio dos(as) Estudantes(as);
		\item \textbf{Caderno 5} -- Trilhas de Aprofundamento da Educação Profissional e Tecnológica;
		\item \textbf{Caderno 6} -- Trilhas de Aprofundamento Formação Docente - Curso Normal e Nível Médio - Magistério.
\end{enumerate}

O currículo do \ac{EM}, -- ou melhor, \ac{NEM} como convencionou-se chamar, -- encontra-se em processo de implantação no Estado, e tal como orienta a \ac{BNCC}, está organizanado por áreas de conhecimento, em que cada componente curricular é vinculada às suas respectivas áreas do conhecimento conforme disposto no \autoref{qua:areas_NEM} -- pág.~\pageref{qua:areas_NEM}.

\begin{quadro}[!ht]
\resizebox{\textwidth}{!}{%
\begin{tabular}{cl}
\rowcolor[HTML]{343434} 
{\color[HTML]{FFFFFF} \textbf{ÁREA DO CONHECIMENTO}} & \multicolumn{1}{c}{\cellcolor[HTML]{343434}{\color[HTML]{FFFFFF} \textbf{COMPONENTE CURRICULAR}}} \\
                                                                                                             & Língua Portuguesa e Literatura \\ \cline{2-2} 
                                                                                                             & Inglês                         \\ \cline{2-2} 
                                                                                                             & Artes                          \\ \cline{2-2} 
\multirow{-4}{*}{\textbf{LINGUAGENS E SUAS TECNOLOGIAS}}                                                     & Educação Física                \\ \hline
\textbf{MATEMÁTICA E SUAS TECNOLOGIAS}                                                                       & Matemática                     \\ \hline
                                                                                                             & Física                         \\ \cline{2-2} 
                                                                                                             & Química                        \\ \cline{2-2} 
\multirow{-3}{*}{\textbf{\begin{tabular}[c]{@{}c@{}}CIÊNCIAS DA NATUREZA E\\ SUAS TECNOLOGIAS\end{tabular}}} & Biologia                       \\ \hline
                                                                                                             & História                       \\ \cline{2-2} 
                                                                                                             & Geografia                      \\ \cline{2-2} 
                                                                                                             & Filosofia                      \\ \cline{2-2} 
\multirow{-4}{*}{\textbf{\begin{tabular}[c]{@{}c@{}}CIÊNCIAS HUMANAS E SOCIAIS\\ APLICADAS\end{tabular}}}    & Sociologia                     \\ \hline
\end{tabular}%
}
\caption{Organização por áreas de conhecimento para o Novo Ensino Médio.}
\label{qua:areas_NEM}
\end{quadro}
Para preencher a carga horária exigida, dividiu-se o currículo em duas partes, uma destinada a contemplar à Formação Geral Básica prevista pela \ac{BNCC}, tendo carga horária máxima de 1800 horas, e a outra correspondente à parte flexível do currículo, com carga horária mínima 1200 horas, vale ressaltar que com essa carga horária assim estabelecida, o \ac{NEM} na modalidade Noturno, passa a ser ofertado não mais com duração de 3 anos como as Matrizes do Diurno e sim em 4 anos. 

A Formação Geral Básica de que trata a \ac{BNCC}, segundo o \ac{CBTCEM}, deve ser promovida de forma a estabelecer
\begin{citacao}
	``[...]o fortalecimento das relações entre os saberes e a sua contextualização para 	a apreensão e a intervenção na realidade, requerendo planejamento e execução conjugadas e cooperativas dos seus professores Portanto, cabe reforçar a obrigatoriedade do trato interdisciplinar e transdisciplinar, interárea e entre áreas, sendo o planejamento integrado e  coletivo indispensável para alcançar este caráter interdisciplinar'' \cite[pp.~58]{CADORI:2022}
\end{citacao}

Na parte flexível o \ac{CBTCEM} busca oportunizar aos estudantes o desenvolovimento de \textit{``competências específicas, preparar-se para o prosseguimento de estudos ou para o mundo do trabalho de forma a contribuir para a construção de soluções de problemas específicos da sociedade\footnote{Cf. \Ibidem[pp.~60]{CADORI:2022}}''}, e ainda, desenvolver habilidades relacionadas aos eixos estruturantes, sendo eles:
\begin{itemize}
		\item Investigação Científica;
		\item Processos Criativos;
		\item Mediação e Intervenção Socio Cultural;
		\item Empreendedorismo.
\end{itemize}
Os eixos estruturantes são responsáveis por organizar os intinerários formativos, de tal sorte que cada intinário deve ser trabalhado a partir de um ou mais eixos estruturantes \footnote{Cf. \Ibidem[60]{CADORI:2022}}, a organização dá-se em termos de componentes curriculares sendo elas: \textit{Projeto de Vida; Componentes Curriculares Eletivos; Segunda Língua Estrangeira e as Trilhas de Aprofundamento}.

Cada componente curricular é elaborado previamente a partir de um tema centralizador que propicie situações de aprendizagem e experiências conectadas ao universo dos estudantes, os \ac{CCE} e as Trilhas de Aprofundamento são disponibilizados através de portifólios nos Cadernos do \ac{CBTCEM}.

Assim, diferentes arranjos curriculares são possíveis, a oferta e construção dos arranjos curriculares, devem seguir as disponibilidades e possibilidades de cada intituição de ensino, primeiramente é feito a escuta ao corpo docente e técnicos da instituição para a elaboração da oferta inicial, em seguida é feito o processo de escuta aos estudantes.
% subsection O CBTCEM (end)
% subsection A PCSC (end)

\section{Apresentação do Trabalho} % (fold)
\label{sec:Apresentação do Trabalho}

Em observação ao exposto, este trabalho foi desenvolvido com base nos cadernos do \ac{CBTCEM} e consequentemente apoiado na \ac{BNCC}. Uma sequência didática para tratar do tema \textit{Furacões}, sujeito à Trilha de Aprofundamento Integrada entre Áreas, cujo a temática central versa, \textit{à priori}, sobre a \textit{Modelagem de Fenômenos Naturais Sociais e seus Impactos}, foi construída e aplicada no decorrer do primeiro semestre do ano de 2023 aos alunos da 2º série do \ac{NEM} -- Noturno, turma pertencente à \ac{EEB} \ac{GPF} anexa ao município de Joinville/SC. A seguir, no \autoref{chap:Apresentação da Concedente}, far-se-á a caracterização da unidade concedente deste estágio, seus aspectos estruturais; o Projeto Político-Pedagógico e suas Matrizes Curriculares em vigência, já no \autoref{chap:Apoio à Docência}, apresentar-se-á os recursos disponíveis na unidade para uso pedagógico. No \autoref{chap:Conteúdo Programático} o programa do componente curricular relacionado à disciplina de Física adotado nas aulas do professor supervisor será exposto, seguido de uma breve análise da aplicação deste programa no \autoref{chap:Atividades de Observação}. A parte das regências, com o perfil da turma, a metodologia utilizada e o planejamento encontra-se no \autoref{chap:Regências} e por fim, encerremos com \autoref{chap:Considerações Finais}.

% section Apresentação do Trabalho (end)




% chapter Introdução (end)
