% -----------------------------------------------%
 %% inserir lista de ilustrações, tabelas, listagem 
% de códigos, abreviaturas, símbolos
% -----------------------------------------------%
\renewcommand{\listfigurename}{Lista de Figuras}
\pdfbookmark[0]{\listfigurename}{lof}
\listoffigures*
\cleardoublepage
% -----------------------------------------------%
% inserir lista de TABELAS
% -----------------------------------------------%
\renewcommand{\listtablename}{Lista de Tabelas}
\pdfbookmark[0]{\listtablename}{lot}
\listoftables*
\cleardoublepage

% -----------------------------------------------%
% inserir lista de QUADROS
% -----------------------------------------------%
\renewcommand{\listofquadrosname}{Lista de Quadros}
\pdfbookmark[0]{\listofquadrosname}{loq}
\listofquadros*
\cleardoublepage

% -----------------------------------------------%
% inserir lista de CÓDIGOS
% -----------------------------------------------%
% \pdfbookmark[0]{\lstlistlistingname}{lol}
% \begin{KeepFromToc}
% \lstlistoflistings
% \end{KeepFromToc}
% \cleardoublepage
% -----------------------------------------------%

% -----------------------------------------------%
% inserir lista de abreviaturas e siglas
% -----------------------------------------------%
\pdfbookmark[0]{Lista de Abreviaturas e Siglas}{loa}
\cleardoublepage
% --%%%%%%%%%%%%%% Como usar o pacote acronym
% \ac{acronimo} -- Na primeira vez que for citado o acronimo, o nome completo irá aparecer
%                  seguido do acronimo entre parênteses. Na proxima vez somente o acronimo
%                  irá aparecer. Se usou a opção footnote no pacote, entao o nome por extenso
%                  irá aparecer aparecer no rodapé
%
% \acf{acronimo} -- Para aparecer com nome completo + acronimo
% \acs{acronimo} -- Para aparecer somente o acronimo
% \acl{acronimo} -- Nome por extenso somente, sem o acronimo
% \acp{acronimo} -- igual o \ac mas deixando no plural com S (ingles)
% \acfp{acronimo}--
% \acsp{acronimo}--
% \aclp{acronimo}--


\chapter*{Lista de Abreviaturas e Siglas}%
% \addcontentsline{toc}{chapter}{Lista de abreviaturas e siglas}
\markboth{Lista de Abreviaturas e Siglas}{}

% Para diminuir o espaçamento entre linhas no ambiente de listas acronym
% \let\oldbaselinestretch=\baselinestretch%
% \renewcommand{\baselinestretch}{.2}%
% \large\normalsize%



\begin{acronym}
	\let\oldbaselinestretch=\baselinestretch%
	\renewcommand{\baselinestretch}{.2}%
	\large\normalsize%
	\acro{SEI}{Sequência de Ensino Investigativa}
	\acro{LDB}{Lei de Diretrizes de Bases da Educação Nacional}
	\acro{PNE}{Plano Nacional de Educação}
	\acro{BNCC}{Base Nacional Comum Curricular}
	\acro{NEM}{Novo Ensino Médio}
	\acro{PCSC}{Proposta Curricular de Santa Catarina}
	\acro{CNE}{Conselho Nacional de Educação}
	\acro{EEB}{Escola de Educação Básica}
	\acro{GPF}{Giovani Pasqualini Faraco}
	\acro{EF}{Ensino Fundamental}
	\acro{PPP}{Projeto Político Pedagógico}
	\acro{ATP}{Assistente Técnico Pedagógico}
	\acro{DVD}{\textit{Digital Versatile Disc}}
	\acro{EI}{Ensino por Investigação}
	\acro{NI/D}{Não-Interativo/Dialógico}
	\acro{I/D}{Interativo-Dialógico}
	\acro{HC}{História da Ciência}
	\acro{TICs}{Tecnologias da Informação e Comunicações}
	\acro{ENEM}{Exame Nacional do Ensino Médio}
	\acro{I/DA}{Interativo/Dialógico de Autoridade}
	\acro{CNT}{Ciências da Natureza e suas Tecnologias}
	\acro{MEC}{Ministério da Educação}
	\acro{TDIC}{Tecnologias Digitais de Informação e Comunicação}
	\acro{ProBNCC}{Programa de Apoio à Implementação da Base Nacional Comum Curricular}
	\acro{CBTCEM}{Currículo Base do Ensino Médio do Território Catarinense}
	\acro{ECS4003}{Estágio Curricular Supervisionado}
	\acro{PISA}{Programa Internacional de Avaliação de Alunos}
	\acro{EM}{Ensino Médio}
	\acro{UDESC}{Universidade do Estado de Santa Catarina}
	\acro{BDTD}{Biblioteca Digital Brasileira de Teses e Dissertações}
	\acro{PPGECMT}{Programa de Pós-Grad. em Ens. de Ciências, Mat. e Tecnologias}
	\acro{CCT}{Centro de Ciências Tecnológicas}
	\acro{TAI}{Tilha de Aprofundamento Integrada}
	\acro{MFNSSI}{Modelagem de Fenômenos Naturais, Sociais e Seus Impactos}
	\acro{SED}{Secretaria de Estado da Educação}
	\acro{SENAI}{Serviço Nacional de Aprendizagem Industrial}
	\acro{PhET}{\textit{Physics Education Technology}}
	\acro{ZCIT}{Zona de Convergência Inter Tropical}
	\acro{UE}{Unidade de Ensino}
	\acro{UNESCO}{Organização das Nações Unidas para a Educação, a Ciência e a Cultura}
	\acro{OCDE}{Organização para a Cooperação e Desenvolvimento Econômico}
	\acro{LLECE}{Laboratório Latino-americano de Avaliações da Qualidade da Educação para a América Latina}
	\acro{CCE}{Componentes Curriculares Eletivos}
	\acro{SAEB}{Sistema de Avaliação da Educação Básica}
	\acro{INEP}{Instituto Nacional de Estudos e Pesquisas Educacionais}
\end{acronym}

% -----------------------------------------------%
% inserir lista de símbolos
% -----------------------------------------------%
% \renewcommand{\listadesimbolosname}{Lista de Símbolos}
% \cleardoublepage
% \begin{simbolos}  
%   \item[$\Delta$] Variações de Grandezas Físicas
% \end{simbolos}
% -----------------------------------------------%

% -----------------------------------------------%
% inserir o SUMÁRIO
% -----------------------------------------------%
\newpage
\pdfbookmark[0]{\contentsname}{toc}
\tableofcontents*
\cleardoublepage
% -----------------------------------------------%
