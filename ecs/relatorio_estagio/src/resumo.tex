\setlength{\absparsep}{18pt} % ajusta o espaçamento dos parágrafos do resumo
\begin{resumo}
	\thispagestyle{empty}
	Este trabalho tem por objetivo relatar as atividades desenvolvidas ao longo do Estágio Currícular Supervisionado IV, componente curricular obrigatório do curso de Licenciatura em Física da \ac{UDESC}. Foi realizado na Escola de Educação Básica Giovani Pasqualini Faraco, situada junto ao município de Joinville. Dentre as principais atividades desenvolvidas, fez-se a leitura e análise dos documentos oficiais da unidade escolar; a observação de aulas da disciplina de Física ministradas pelo professor supervisor, seguida de análises de como se articulam os momentos pedagógicos; a observação e caracterização do ambiente escolar e recursos didáticos disponíveis na unidade concedente e o desenvolvimento de uma Sequência de Ensino contendo dez aulas programadas para a turma de 2ª série do Novo Ensino Médio noturno. Este estágio desenvolveu-se simultâneamente à implantação do Novo Ensino Médio nas turmas de 2ª série, de modo que a Sequência de Ensino desenvolvida, desdobrou-se sobre o tema \textit{Furacões}, sujeito à Trilha: Modelagem de Fenômenos Naturais, Sociais e seus Impactos, componente currícular da parte flexível do Novo Ensino Médio, abordado no contexto da Área de Ciências da Naturezas e suas Tecnologias. A sequência foi organizada em três etapas, na primeira buscou-se apresentar o fenômeno com base nas suas principais características, na segunda conduziu-se o estudo qualitativo à cerca das variáveis fundamentais ao seu entendimento, por fim discutiu-se as causas e efeitos do fenêomeno em escala global afim de promover a compreensão sistêmica do fenômeno.

	\textbf{Palavras-chave}: \firstkey ; \secondkey ; \thirdkey.
\end{resumo}
