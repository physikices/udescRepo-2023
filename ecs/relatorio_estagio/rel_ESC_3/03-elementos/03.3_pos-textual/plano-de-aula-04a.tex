%-----------------------------------------------%
% Início do plano de aula
%-----------------------------------------------%
\thispagestyle{empty}
\begin{center}
	\begin{minipage}[!]{\linewidth}
        \begin{minipage}[!]{.19\linewidth}
            \includegraphics[width=\linewidth]{img/logo.png}           
        \end{minipage}
        \begin{minipage}[!]{.8\linewidth}
            \center
            \ABNTEXchapterfont\normalsize\MakeUppercase{\imprimirinstituicao}
            \par
            \vspace*{10pt}                     
            \ABNTEXchapterfont\normalsize\MakeUppercase{\centro}
            \par
            \vspace*{10pt}           
            \ABNTEXchapterfont\normalsize\MakeUppercase{\disciplina}
        \end{minipage}        
    \end{minipage}
    \\ \vspace{0.5cm}
    \rule{\textwidth}{.5pt}   
\end{center}
    \textual
    \begin{center}
      \section{Lei de Snell-Descartes}
      \par
    \end{center}
    
    \noindent \textbf{Estagiário(a): }\imprimirautor 
    
    \noindent \textbf{U.E.: }EEB Giovani Pasqualini Faraco
    
    \noindent \textbf{Série: }2º Ano\hfill{}\textbf{Turma: }2º--5
    
    \noindent \textbf{Aula:} 004\hfill{}\textbf{Data:} XX/XX/2022\hfill{}\textbf{Duração:} $45\min$
    \rule{\textwidth}{.5pt}
    \bigskip{}  
    

    \noindent
    \begin{center}
      \textbf{Lei de Snell-Descartes (Dedução)}
    \par\end{center}

    \noindent \textbf{Resumo da aula:} Nesta aula será apresentada ambos os modelos para a propagação da luz, o modelo corpuscular e o modelo ondulatório, também será apresentado as principais distinções entre eles e suas respectivas previsões para o comportamento da luz ao propagar-se em meios diferentes. De início, não iremos privilegiar um modelo em detrimento de outro, esta distinção deve ser dada na próxima aula com a apresentação dos fenômenos de interferência e difração.

    \par\noindent \textbf{Habilidades BNCC: }EF03CI02; EF09CI04.
	
    \subsection*{Objetivo de Aprendizagem}
    \begin{itemize}
        \item Reconhecer as principais diferenças entre o modelo corpuscular e o modelo ondulatório para a velocidade de propagação luminosa;
        \item Relacionar as mudanças da trajetória da luz (reflexão e refração), com forças que agem sobre a partícula de luz;
        \item Relacionar as mudanças da trajetória da luz (reflexão e refração) com a Lei da reflexão e a Lei de Snell-Descartes. 
    \end{itemize}
    
    \medskip{}
    
    \noindent \textbf{Núcleo Conceitual:} \emph{Perguntar ao professor orientador do que trata esta parte do plano de aula.}
    \newpage
    
    \section*{Procedimento Didático} 
    \noindent \emph{1º Momento:} Problematização Inicial
	\par\noindent\rule{.3\textwidth}{.5pt}  
    \par\noindent \textbf{Tempo previsto:} 10 minutos

    \noindent \textbf{Dinâmica:} Revisar os conceitos do modelo ondulatório vistos na aula passada tais como:

    \begin{itemize}
        \item Comprimento de onda;
        \item Período;
        \item Velocidade;
    \end{itemize}
    num comparativo entre seus "análogos" corpusculares, sendo eles
    \begin{itemize}
        \item Distância percorrida;
        \item Variação do tempo;
        \item Velocidade média.
    \end{itemize}

    Problematizar: Como é possível descrever os fenômenos observados na simulação, em cada um destes modelos? Será que ambos os modelos dão conta disso? É possível privilegiar um, em detrimento do outro? 

    \vspace{50pt}
    \noindent \emph{2º Momento:} Título do segundo momento.
	\par\noindent\rule{.3\textwidth}{.5pt}    
    \par\noindent \textbf{Tempo previsto: }20 minutos
	

    \noindent \textbf{Dinâmica:} Apresentar o modelo corpuscular. Mostrar por soma vetorial a necessidade de uma força apontando para baixo para que haja mudança na sua trajetória no meio material.

    Apresentar a Lei de Snell Descartes


	\vspace{50pt}
    \noindent \emph{3º Momento:} Título do terceiro momento.
	\par\noindent\rule{.3\textwidth}{.5pt}
    \par\noindent \textbf{Dinâmica:} Descrever a dinâmica do terceiro momento.
%-----------------------------------------------%