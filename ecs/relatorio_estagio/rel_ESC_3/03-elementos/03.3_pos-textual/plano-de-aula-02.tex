\begin{center}
    \begin{minipage}[!]{\linewidth}
        \begin{minipage}[!]{.19\linewidth}
            \includegraphics[width=\linewidth]{img/logo.png}           
        \end{minipage}
        \begin{minipage}[!]{.8\linewidth}
            \center
            \ABNTEXchapterfont\normalsize\MakeUppercase{\imprimirinstituicao}
            \par
            \vspace*{10pt}                     
            \ABNTEXchapterfont\normalsize\MakeUppercase{\centro}
            \par
            \vspace*{10pt}           
            \ABNTEXchapterfont\normalsize\MakeUppercase{\disciplina}
        \end{minipage}        
    \end{minipage}
    \\ \vspace{0.5cm}
    \rule{\textwidth}{.5pt}   
\end{center}

\textual
    \begin{center}
        %\textbf{Plano de Aula: Intervenção Pedagógica nº 00X}
        \section{Índice de Refração (Simulação)}
        \par
    \end{center}        
        \noindent \textbf{Estagiário:} \imprimirautor 
        
        \noindent \textbf{U.E.:} EEB Giovani Pasqualini Faraco
        
        \noindent \textbf{Série:} 2º Ano\hfill{}\textbf{Turma:} 2º--5
        
        \noindent \textbf{Aula:} 002\hfill{}\textbf{Data:} 14/10/2022\hfill{}\textbf{Duração:} $45\min$
        \rule{\textwidth}{.5pt}
        \bigskip{}  
        
        %\section{Plano-01: Primeiro plano de aula}
        \noindent \begin{center}
        \textbf{Título: Construindo o conceito de índice de refração}
        \par\end{center}

        \noindent \textbf{Resumo da aula:} Utilizando simulações \emph{PhET}\footnote{\url{https://phet.colorado.edu/sims/html/bending-light/latest/bending-light_en.html}}, iremos explorar algumas das características da refração da luz em meios diferentes. Os conceitos de índices de refração (relativo e absoluto) serão construídos em conjunto com a turma.

        \par\noindent \textbf{Habilidades BNCC:} EF03CI02.
        \vfill
        \subsection*{Objetivos de Aprendizagem}
        \begin{itemize}
            \item Perceber que, quando a luz passa do meio ar para o meio água ou vidro, esta tem a sua trajetória modificada;
            \item Verificar que a velocidade de deslocamento da luz na água e no vidro são menores que no ar;
            \item Relacionar o desvio das trajetórias da luz em meios diferentes com a mudança na sua velocidade de propagação em cada meio;
            \item Definir os índices de refração relativo e absoluto dos meios estudados;
        \end{itemize}
        
        \medskip{}
        \vfill
        \noindent \textbf{Dimensão Conceitual:} \emph{Domínio Epistêmico; Domínio Conceitual.}
        
        
        \subsection*{Procedimento Didático}
        \vspace*{40pt}
        \noindent \emph{1º Momento:} Observando os desvios do da luz em diferentes meios.
        \par\noindent\rule{.3\textwidth}{.5pt}  
        \par\noindent \textbf{Tempo previsto:} 25 minutos

        \noindent \textbf{Dinâmica:} Abrir a simulação do PhET, destinar $2\min$ da aula para explicar a simulação. Durante a simulação, proceder da seguinte forma:
        \begin{enumerate}
            \item Variar o ângulo de incidência da luz primeiramente nos meios $\text{ar}\to\text{ar}$.
            \begin{enumerate}
                \item Chamar a atenção para o que se observa com relação a trajetória da luz na incidência e na parte refratada;
            \end{enumerate}
            \item Alterar o meio 2 para água de modo que a luz percorra a trajetória $\text{ar}\to\text{água}$, fazer variações na posição do feixe e pedir para que observem o que ocorre.
            \begin{enumerate}
                \item Perguntar qual a diferença entre o observado anteriormente e o que ocorre agora;
                \item Questionar o que esperam que aconteça se alterar o meio 1 para água de modo que a luz percorra os meios $\text{água}\to\text{água}$;
                \item Alterar o meio 1 para água e verificar se a previsão dos alunos se concretiza;
                \item Perguntar o que esperam que aconteça se alterarmos o meio 2 para ar de modo que agora a luz percorra os meios $\text{água}\to\text{ar}$;
                \item Proceder como anteriormente de modo que agora a luz percorra a trajetória do meio $\text{água}\to\text{ar}$
            \end{enumerate}
        \end{enumerate}
        
        Antes de prosseguir, fazer os seguintes questionamentos:
        
        \begin{center}
            \colorbox{gray85}{
            \begin{minipage}{0.9\textwidth}
                \begin{quest}
                    O que deve estar ocorrendo com a luz em cada observação para que ela altere a sua trajetória?
                \end{quest}
                \begin{quest}
                    Qual a relação existente entre trajetória e velocidade?
                \end{quest}
                \begin{quest}
                    Qual a relação entre os meios de propagação da luz e o desvio de sua trajetória?
                \end{quest}
                \begin{quest}
                    Vocês conseguem estabelecer alguma relação entre as observações feitas nas simulações e os experimentos conduzidos na aula passada?
                \end{quest}
            \end{minipage}
            }
        \end{center}
        Destinar $5\min$ da aula para que respondam as questões levantadas.

        
        \begin{mybox}[colback=white, colframe=deepred,colbacktitle=deepred!85!deepred,]{Observação}
            Como explicação para os fenômenos observado na simulação, os alunos facilmente relacionaram o desvio da luz em ambos os casos com alterações na velocidade de propagação, porém não houve consenso em qual dos casos a velocidade da luz deve ser maior ou menor. Constatou-se ainda que conceitos elementares relacionados à ondulatória, como velocidade, comprimento de onda e frequência não haviam sido construído, e que além disto, a noção de partícula bem como as principais características do modelo corpuscular carecia de significância e compreensão. Dessa forma optou-se por adiar o restante da aplicação deste plano de aula e dedicou-se o restante da aula a uma breve revisão de cinemática vetorial bidimensional.
        \end{mybox} 
        
        \vspace{60pt}   
        \par\noindent \emph{2º Momento:} Relacionando os desvios de caminho óptico com a velocidade de propagação da luz em cada meio.
        \par\noindent\rule{.3\textwidth}{.5pt}  
        \par\noindent \textbf{Tempo previsto:} 15 minutos

        \noindent \textbf{Dinâmica:} Neste momento o professor deve responder cada uma das questões utilizando a simulação de forma dialogada com os alunos. Proceder da seguinte forma:
        \begin{enumerate}
            \item Setar a simulação para o modo \emph{Wave} e \emph{Slow Motion} para facilitar a visualização;
            \item Medir a velocidade da luz em cada caso observado no 1º momento da aula;
            \item Pedir para que os alunos calculem a razão entre a velocidade da luz no ar e a velocidade da luz na água;
            \item Pedir para que comparem o resultado com o índice de refração da água.       
        \end{enumerate}
        \begin{table}[!ht]
            \centering
            \begin{tabular}{c|c}
                \textbf{Comprimento de Onda $\lambda(\nm)$} & \textbf{Índice de Refração $n$} \\ \hline
                226,5                                       & 1,393 36                        \\
                361,05                                      & 1,347 95                        \\
                404,41                                      & 1,343 15                        \\
                589                                         & 1,333                           \\
                632,8                                       & 1,332 11                        \\
                1013,98                                     & 1,325 24                       
            \end{tabular}
            \caption{Índices de refração da água ($20\Celsius$). Fonte: www.fq.pt}
            \label{tab:n-agua}
        \end{table}
        
        \vspace{50pt}
        \noindent \emph{3º Momento:} Definindo os índices de refração.
        \par\noindent\rule{.3\textwidth}{.5pt}  
        \par\noindent \textbf{Tempo previsto:} 5 minutos

        \noindent \textbf{Dinâmica:} Colocar no quadro a definição do índice de refração absoluto, como segue:
        \begin{definicao}
            O índice de refração \emph{absoluto} da luz em um meio é a medida da razão entre a velocidade da luz no vácuo $c$ e a velocidade da luz no meio em questão
            \begin{align}
            \boxed{
                n=\frac{c}{v}
            }                
            \end{align}
        \end{definicao}
        O professor deve, então, comentar que a velocidade da luz no vácuo $c$ é muito próxima a velocidade da luz no ar e que para fins práticos, o índice de refração absoluto do ar é $n_{\text{ar}}=1,00$. É importante deixar claro também que, em virtude da velocidade da luz, nunca deve-se observar um índice de refração absoluto menor do que 1.

        Definir o índice de refração relativo a partir da definição acima, como segue:

        Se em um determinado meio a luz percorre sua trajetória com velocidade $v_1$, e ao passar para um segundo meio essa velocidade se altera para $v_2$, podemos estabelecer uma grandeza chamada \emph{índice de refração relativo} $n_{1,2}$ da seguinte forma
        \begin{align}
            n_1& =\frac{c}{v_1} & n_2& =\frac{c}{v_2}            
        \end{align}
        dividindo $n_1$ por $n_2$ obtemos
        \begin{align}
            \frac{n_1}{n_2}&=\left(\frac{c}{v_1}\right)\left(\frac{v_2}{c}\right)=\frac{v_2}{v_1}
        \end{align}
        O índice de refração relativo fica então definido por
        \begin{align}
            \boxed{
                n_{1,2}=\frac{v_2}{v_1}
            }
        \end{align}
        \par\noindent\textbf{Avaliação:} Propor o desafio de descobrirem qual é o material Mistério A e Mistério B na simulação.





