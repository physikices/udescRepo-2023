%-----------------------------------------------%
% Início do plano de aula
%-----------------------------------------------%
\thispagestyle{empty}
\begin{center}
	\begin{minipage}[!]{\linewidth}
        \begin{minipage}[!]{.19\linewidth}
            \includegraphics[width=\linewidth]{img/logo.png}           
        \end{minipage}
        \begin{minipage}[!]{.8\linewidth}
            \center
            \ABNTEXchapterfont\normalsize\MakeUppercase{\imprimirinstituicao}
            \par
            \vspace*{10pt}                     
            \ABNTEXchapterfont\normalsize\MakeUppercase{\centro}
            \par
            \vspace*{10pt}           
            \ABNTEXchapterfont\normalsize\MakeUppercase{\disciplina}
        \end{minipage}        
    \end{minipage}
    \\ \vspace{0.5cm}
    \rule{\textwidth}{.5pt}   
\end{center}
    \textual
    \begin{center}
      \section{A Teoria Éter}
      \par
    \end{center}
    
    \noindent \textbf{Estagiário(a): }\imprimirautor 
    
    \noindent \textbf{U.E.: }EEB Giovani Pasqualini Faraco
    
    \noindent \textbf{Série: }2º Ano\hfill{}\textbf{Turma: }2º--5
    
    \noindent \textbf{Aula:} 008\hfill{}\textbf{Data:} 04/11/2022\hfill{}\textbf{Duração:} $45\min$
    \rule{\textwidth}{.5pt}
    \bigskip{}  
    

    \noindent
    \begin{center}
      \textbf{Qual o meio de propagação da luz?}
    \par\end{center}
    \smallskip{}
    \noindent \textbf{Resumo da aula:} Será dado inicio a uma discussão a cerca de qual meio deve dar suporte às onda eletromagnéticas e consequentemente aos fenômenos luminosos.
    \medskip{}
    \par\noindent \textbf{Habilidades BNCC:} EM13CNT201.
    \bigskip{}
    \subsection*{Objetivo de Aprendizagem}
    \begin{itemize}
        \item Justificar a necessidade da Teoria do Éter na época do desenvolvimento da teoria eletromagnética;
        \item Exemplificar a transitoriedade das questões científicas, no que concerne ao levantamento contínuo de hipóteses cada vez mais elaboradas em decorrência da construção do conhecimento.
    \end{itemize}    
    \smallskip{}    
    \noindent \textbf{Núcleo Conceitual:} \emph{Luz como uma onda eletromagnética; a necessidade do Éter.}
    \newpage
    

    \section*{Procedimento Didático} 
    \noindent\emph{1º Momento:} Breve Revisão
    \par\noindent\rule{.3\textwidth}{.5pt}  
    \par\noindent\textbf{Tempo previsto:} 10 minutos
    \smallskip
    \par\noindent\textbf{Dinâmica:} Retomando o que foi discutido na última aula, o professor segue investigando os fenômenos luminosos que só podem ser explicados, se a luz for concebida como uma onda, inicia sua sua fala recapitulando o que foi visto na sequência:

    \begin{itemize}
        \item Vimos que há dois modelos contraditórios para descrever os fenômenos luminosos, o modelo corpuscular e o modelo ondulatório;
        \item Vimos que no modelo corpuscular:
        \begin{itemize}
            \item A luz é composta por minúsculas partículas de luz;
            \item O fenômeno de reflexão é compreendido pela lei da conservação do momento linear;
            \item O ângulo de reflexão concorda com as observações e é igual ao ângulo de incidência;
            \item A velocidade da luz na reflexão é igual a velocidade de incidência, apenas trocando-se a direção da  componente vertical da velocidade;
            \item O fenômeno de refração é compreendido pela segunda lei de newton, que prevê a existência de uma força que altera somente a componente vertical da velocidade;
            \item A velocidade da luz em meios mais densos é maior;
            \item Alguns fenômenos seguem sem compreensão como: os fenômenos de interferência, difração e etc
        \end{itemize}
        \item Já no modelo ondulatório:
        \begin{itemize}
            \item A luz é concebida como uma onda que se propaga em algum meio [este meio será problematizado ao final desta aula];
            \item Os fenômenos de reflexão e refração são compreendidos pelo princípio de Hyugens, bem como todos os fenômenos de propagação;
            \item O ângulo de incidência e reflexão são iguais e são descritos pela lei de Snell-Descartes;
            \item A lei de Snell-Descartes também descreve o fenômeno de refração;
            \item A velocidade da luz em meios mais densos é menor;
            \item Fenômenos de interferência e difração são explicados pelo princípio de Huygens e observados no experimento de Young;
            \item Os experimentos de Young, também comprovam a que a luz é um tipo de onda;
            \item O fenômeno de ressonância de ondas eletromagnéticas demonstrados por Hertz, somado às medições obtidas para a velocidade da luz e em combinado com o desenvolvimento do eletromagnetismo, sugerem que a luz seja um tipo de onda eletromagnética de velocidade conhecida por $c=\sim 298\kmps$.
        \end{itemize}
    \end{itemize}

    Feita esta revisão inicial, problematizar da seguinte forma: \emph{Muito bem, o que temos? Com Young  nós temos uma comprovação de que a luz é uma onda e com Maxwell que essa onda é a onda eletromagnética. Sabe-se que toda onda precisa de um meio para se propagar, então que meio é esse? Alguma hipótese?} Anotar as hipóteses no quadro.
    

% Segundo os autores é nessa etapa que se apresentam questões
% e/ou situações para discussão com os alunos, visando relacionar o estudo de um conteúdo com
% situações reais que eles conhecem e presenciam, mas que não conseguem interpretar completa ou
% corretamente porque provavelmente não dispõem de conhecimentos científicos suficientes. Ou seja,
% é na problematização que se deseja aguçar explicações contraditórias e localizar as possíveis
% limitações do conhecimento que vem sendo expressado, quando este é cotejado com o conhecimento
% científico que já foi selecionado para ser abordado (Delizoicov, Angotti e Pernambuco, 2002, p. 201).
% Portanto, esse primeiro momento é caracterizado pela compreensão e apreensão da posição dos alunos
% frente ao tema. É desejável ainda, que a postura do professor se volte mais para questionar e lançar
% dúvidas sobre o assunto que para responder e fornecer explicações.

    \bigskip{}
    \noindent\emph{2º Momento:} Organização do Conhecimento
    \par\noindent\rule{.3\textwidth}{.5pt}  
    \par\noindent\textbf{Tempo previsto:} 20 minutos
    \smallskip
    \par\noindent\textbf{Dinâmica:} Nesta parte o professor deve problematizar cada hipótese forçando-os ao pensamento da época de Maxwell e seus contemporâneos. Para refutar as respostas dos estudantes, podemos nos embasar da seguinte forma:

    \begin{itemize}
        \item A hipótese mais óbvia é que a luz tem como meio de propagação o ar, porém esta hipótese é facilmente refutada. A título de exemplo, pode-se argumentar que o som é uma onda que se propaga pelo ar, mas não se propaga no vácuo, no entanto a luz que sai do sol chega até nós mesmo não havendo ar entre o Sol e a Terra.
        \item Qualquer outra hipótese que se utilize de meios mecânicos para justificar o meio de propagação das ondas luminosas, deve ser refutada pelos mesmos argumentos acima.
        \item Outra hipótese que pode surgir é o vácuo como um meio de propagação da luz, porém o vácuo não era concebível à época, além do mais o vácuo não é um meio propriamente dito e sim, talvez, a ausência de um, neste caso, o vácuo é antes um problema do que uma solução. 
    \end{itemize}

    Manter a discussão até se tenha um consenso de que é necessário haver algum meio "especial" cujo o qual dê suporte a propagação das ondas eletromagnéticas.
    

% Delizoicov e Angotti (1990, p. 29) explicam que nesse
% segundo momento os conhecimentos de Física necessários para a compreensão do tema e da
% problematização inicial devem ser sistematicamente estudados sob orientação do professor.
% Definições, conceitos, relações, leis, apresentadas no texto introdutório, serão agora aprofundados.
% De acordo com Albuquerque, Santos e Ferreira (2015, p. 467) esse é o momento em que os
% conhecimentos científicos passam a ser incorporados nas discussões. Os alunos começam a
% desenvolver uma compreensão a respeito da problematização ou situação inicial. Entretanto, para que
% isso ocorra, materiais devem ser consultados e atividades devem ser sugeridas para complementar as
% discussões, no sentido de incentivar e melhorar a sistematização dos conhecimentos.
% Nessa perspectiva, Delizoicov e Angotti (1990) vêm ressaltar a importância de diversificadas
% atividades, com as quais se poderá trabalhar para organizar a aprendizagem. Sugerem exposições,
% pelo professor, de definições e propriedades, além de formulações de questões (exercícios de fixação
% como dos livros didáticos), textos e experiências. Neste sentido, atualmente poderíamos acrescentar
% as mídias tecnológicas, como televisão, vídeos, filmes, programas tecnológicos, aplicativos de
% celulares, simulações, entre outros, de modo a auxiliar no processo da sistematização do
% conhecimento.
    \newpage
    \bigskip
    \noindent\emph{3º Momento:} Postulado
    \par\noindent\rule{.3\textwidth}{.5pt}  
    \par\noindent\textbf{Tempo previsto:} 10 minutos
    \smallskip
    \par\noindent\textbf{Dinâmica:} Ofertar como alternativa a hipótese do Éter Luminífero

    \begin{itemize}
        \item A hipótese do Éter:
        \item A teoria eletromagnética não definia claramente a partir de qual observador a velocidade da luz estava sendo obtida, postulou-se então o seguinte:
        \begin{itemize}
            \item \emph{Existe um meio privilegiado onde são válidas as leis do eletromagnetismo. Tal meio foi chamado de O Éter luminífero.}
            \item O Éter é um fluído material, porém, imponderável, infinito, homogêneo e isotrópico permeando todos os corpos, inclusive o interior deles.
            \item no interior de corpos transparentes o éter seria parcialmente arrastado pelo corpo quando em movimento em relação ao éter exterior, dito estacionário (Proposição créditada à Fresnel). Quanto aos corpos opacos, o éter se manteria estacionário, sem ser perturbado pelo movimento desses.
            \item O éter deve substituir a ideia de referencial absoluto estabelecida na mecânica newtoniana na descrição dos fenômenos físicos, inclusive no que concerne a descrição da própria mecânica.
        \end{itemize}       
        
    \end{itemize}

       Na próxima aula será continuada as investigações e tentativas de detecção do Éter através do interferômetro de Michelson-Morley.
    

% Essa última etapa aborda sistematicamente o
% conhecimento que vem sendo incorporado pelo aluno para analisar e interpretar tanto a situações
% iniciais que determinaram o seu estudo, como outras situações que não estejam diretamente ligadas
% ao motivo inicial, mas que são explicadas pelo mesmo conhecimento. (Delizoicov e Angotti, 1990,
% p. 31).
% Este é o momento importante para que os alunos encontrem relações entre os temas
% abordados, não apenas através dos conceitos, mas também de fenômenos que possam ter alguma
% conexão com as informações apresentadas. No entanto, o professor mantém a postura
% problematizadora, podendo trazer questionamentos que não foram levantados pelos alunos, como
% informações e problemas que surgiram do decorrer dos momentos. Além disso, este é um bom
% momento para o professor formalizar alguns conceitos que não foram aprofundados pelos alunos.
% (Albuquerque, Santos e Ferreira, 2015).

%-----------------------------------------------%
% FIM do plano de aula
%-----------------------------------------------%


% encionar a unificação da eletricidade com o magnetismo feita por James Clerk Maxwell e publicada em  seus trabalhos no ano de 1873. Neste trabalho, uma das conclusões obtidas por Maxwell foi a existência das ondas eletromagnéticas, desta forma, Maxwell unifica duas áreas distintas da Física: a Eletricidade e o Magnetismo, formando o que chamamos hoje de Eletromagnetismo. Ao calcular a rapidez (velocidade) com a qual a onda eletromagnética se propagaria encontrou um valor consistente com as melhores determinações da velocidade da luz até então (medidas do Fizeau). Cogitou então que a luz pudesse ser uma onda eletromagnética. Em 1862 James Clerk Maxwell afirmou:
    
%     \begin{citacao}
%         ``A velocidade das ondas transversais em nosso meio hipotético, calculada a partir dos experimentos electromagnéticos dos Srs. Kolhrausch e Weber, concorda tão exatamente com a velocidade da luz, calculada pelos experimentos óticos do Sr. Fizeau, que é difícil evitar a inferência de que a luz consiste nas ondulações transversais do mesmo meio que é a causa dos fenômenos elétricos e magnéticos.''
%     \end{citacao}



%     Trabalhos posteriores à Maxwell como os de Hertz (existência das ondas eletromagnéticas), Fizeaul