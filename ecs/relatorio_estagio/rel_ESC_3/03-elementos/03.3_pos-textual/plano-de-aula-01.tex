%-----------------------------------------------%
% Modelo de Plano de Aula com três momentos pedagógicos
%
% Autor: Rodrigo Nascimento (2022-08-12)
%-----------------------------------------------%
\begin{center}
    \begin{minipage}[!]{\linewidth}
        \begin{minipage}[!]{.19\linewidth}
            \includegraphics[width=\linewidth]{img/logo.png}           
        \end{minipage}
        \begin{minipage}[!]{.8\linewidth}
            \center
            \ABNTEXchapterfont\normalsize\MakeUppercase{\imprimirinstituicao}
            \par
            \vspace*{10pt}                     
            \ABNTEXchapterfont\normalsize\MakeUppercase{\centro}
            \par
            \vspace*{10pt}           
            \ABNTEXchapterfont\normalsize\MakeUppercase{\disciplina}
        \end{minipage}        
    \end{minipage}
    \\ \vspace{0.5cm}
    \rule{\textwidth}{.5pt}   
\end{center}

\textual
    \begin{center}
        %\textbf{Plano de Aula: Intervenção Pedagógica nº 00X}
        \section{Refração da Luz (Observação Experimental)} 
        \par
    \end{center}        
        \noindent \textbf{Estagiário:} \imprimirautor 
        
        \noindent \textbf{U.E.:} EEB Giovani Pasqualini Faraco
        
        \noindent \textbf{Série:} 2º Ano\hfill{}\textbf{Turma:} 2º--5
        
        \noindent \textbf{Aula:} 001\hfill{}\textbf{Data:} 14/10/2022\hfill{}\textbf{Duração:} $45\min$
        \rule{\textwidth}{.5pt}
        \bigskip{}  
        
        %\section{Plano-01: Primeiro plano de aula}
        \noindent \begin{center}
        \textbf{Título: Investigando a Refração da Luz Visível}
        \par\end{center}

        \noindent \textbf{Resumo da aula:} Por meio de uma atividade experimental demostrativa (\autoref{anx:invisibilidade-refracao}), investigaremos os fenômenos da reflexão e da refração.

        \par\noindent \textbf{Habilidades BNCC:} EF03CI02.
        \vfill
        
        \subsection*{Objetivos de Aprendizagem}
        \begin{itemize}
            \item Analisar o comportamento da luz visível ao atravessar meios de índices de refração diferentes;
            \item Demonstrar que, para meios cujo o índice de refração são muito próximos, a luz não sofre desvio de caminho óptico, tornando os materiais "invisíveis".
        \end{itemize}
        
        \medskip{}
        \vfill
        \noindent \textbf{Dimensão Conceitual:} \emph{Domínio Epistêmico; Domínio Social.}
        
        
        \subsection*{Procedimento Didático} 
        \noindent \emph{1º Momento:} Discussão sobre como vemos objetos.
        \par\noindent\rule{.3\textwidth}{.5pt}  
        \par\noindent \textbf{Tempo previsto:} 5 minutos

        \noindent \textbf{Dinâmica:} Iniciar a aula provocando uma discussão sobre como enxergamos objetos, fazendo questões como:
        \begin{itemize}
            \item Como nós vemos as coisas?
            \item Podemos ver algo no escuro? Se sim, como e por quê?
        \end{itemize}
        Respostas como "óculos de visão noturna" são esperadas, mas, nesse caso, nós estamos utilizando uma tecnologia que nos permite enxergar a radiação térmica dos objetos. O professor deve elucidar que radiação térmica sempre há, mas que não faz parte do espectro visível.
        
        Após atingir o consenso da turma de que só podemos enxergar a luz visível refletida sobre os objetos, perguntar de que forma então poderíamos testar isso? Será que podemos construir alguma maneira de não vermos algum objeto? Como seria isso?
        
        \vspace{50pt}
        \noindent \emph{2º Momento:} Atividade experimental: Invisibilidade observada devido à refração.
        \par\noindent\rule{.3\textwidth}{.5pt}    
        \par\noindent \textbf{Tempo previsto: }20 minutos        

        \noindent \textbf{Dinâmica:} Executar a atividade experimental contida na \autoref{anx:invisibilidade-refracao}

        \vspace{50pt}
        \noindent \emph{3º Momento:} Organização do conteúdo.
        \par\noindent\rule{.3\textwidth}{.5pt}
        \par\noindent \textbf{Tempo previsto: }20 minutos
        
        \par\noindent \textbf{Dinâmica:} Promover uma discussão sobre as perguntas levantadas pelo roteiro. Neste momento o professor recebe as hipóteses e, havendo muitas, tenta reduzi-las estabelecendo um confronto entre elas. Os estudantes devem argumentar a favor de suas hipóteses. 

        Em todo caso, a(s) hipótese(s) que permanecerem devem ser tomadas como nota e, na aula seguinte, deve-se retomá-las a fim de comparar com o modelo teórico. 
        \par\noindent \textbf{Avaliação:} Será feita por meio da análise da qualidade das argumentações e das interações discursivas produzidas pelos alunos no decorrer da aula.        


%  