
\chapter{Acompanhamento de Aulas}
\thispagestyle{empty}
\label{cap:acompanhamento-de-aulas}

Um total de 12 aulas foram assistidas subdivididas em 03 turmas; uma do segundo ano e outras duas do terceiro ano. Os conteúdos abordados encontram-se elencados no \autoref{qua:acompanhamento de aulas}:

\vspace{10pt}
\begin{quadro}[!ht]
\centering
\resizebox{\textwidth}{!}{%
\begin{tabular}{|c|c|c|c|l|}
\hline
\multicolumn{1}{|l|}{\textbf{Aula}} &
  \multicolumn{1}{l|}{\textbf{Dia}} &
  \multicolumn{1}{l|}{\textbf{Período}} &
  \multicolumn{1}{l|}{\textbf{Turma}} &
  \textbf{Conteúdos Abordados} \\ \hline
01 & \multirow{5}{*}{16/09} & \multirow{12}{*}{Noturno} & 3º(4) & Energia Consumida                 \\ \cline{1-1} \cline{4-5} 
02 &                        &                           & 3º(5) & Energia Consumida                 \\ \cline{1-1} \cline{4-5} 
03 &                        &                           & 2º(5) & Conceitos de Óptica Geométrica    \\ \cline{1-1} \cline{4-5} 
04 &                        &                           & 2º(5) & Proposta de Atividade             \\ \cline{1-1} \cline{4-5} 
05 &                        &                           & 3º(4) & Energia Consumida                 \\ \cline{1-2} \cline{4-5} 
06 & \multirow{5}{*}{30/09} &                           & 3º(4) & Energia Consumida                 \\ \cline{1-1} \cline{4-5} 
07 &                        &                           & 3º(5) & Energia Consumida                 \\ \cline{1-1} \cline{4-5} 
08 &                        &                           & 2º(5) & Apresentação de Atividade         \\ \cline{1-1} \cline{4-5} 
09 &                        &                           & 2º(5) & Apresentação de Atividade         \\ \cline{1-1} \cline{4-5} 
10 &                        &                           & 3º(4) & Energia Consumida                 \\ \cline{1-2} \cline{4-5} 
11 & \multirow{2}{*}{25/11} &                           & 2º(5) & O Experimento de Michelson-Morley \\ \cline{1-1} \cline{4-5} 
12 &                        &                           & 2º(5) & As Transformações de Lorentz      \\ \hline
\end{tabular}%
}
\caption{Cronograma de aulas assistidas}
\label{qua:acompanhamento de aulas}
\end{quadro}
\vspace{10pt}

Foi possível identificar três tipos de abordagens, sendo estas classificadas como: \emph{atividade investigativa; aulas de introdução de conteúdo e aulas de exercícios.} A seguir será feita a análise destas abordagens.

\section{Abordagens Relacionadas à Introdução de Conteúdo}
Nesta abordagem o professor frequentemente inicia um conteúdo novo trazendo situações do cotidiano que fazem parte do universo do aluno, por exemplo: Ao introduzir o conceito de \emph{potencia elétrica}, o faz primeiramente dando a definição de potência
\begin{center}
    \begin{minipage}{0.7\textwidth}
        \begin{itemize}
            \item[\textbf{Prof.:}] Potência é [\ldots] a rapidez com que um trabalho é realizado, ou seja, é a medida do trabalho realizado por uma unidade de tempo.
        \end{itemize}
    \end{minipage}
\end{center}
e exemplifica logo na sequência
\begin{center}
    \begin{minipage}{0.7\textwidth}
        \begin{itemize}
            \item[\textbf{Prof.:}] Por exemplo, todos aqui já viram um Fiat Mobi certo? Também sabem o que é um Camaro correto?
            \item[\textbf{Alunos:}] Sim
            \item[\textbf{Prof.:}] Beleza... Entre um Mobi e um Camaro, qual dos dois chega primeiro numa distância, sei lá, daqui até a praia?
            \item[\textbf{Alunos:}] O Camaro né professor!
        \end{itemize}
    \end{minipage}
\end{center}
De acordo com \cite{Carvalho1999} esta estratégia visa aproximar os conceitos representativos da unidade de ensino, ao universo sensível dos estudantes, e é capaz de promover uma passagem da linguagem coloquial para a linguagem científica de forma natural.

Também é possível notar a gênese da formação dos padrões de interação oriundos da dinâmica que vai se construindo em sala de aula conforme discutido por \cite{MORTIMER2002}. Tais padrões se complexificam a medida que professor e alunos vão avançando no decorrer das exposição, como pode visto na passagem a seguir:

\begin{center}
    \begin{minipage}{0.7\textwidth}
    \begin{itemize} 
        \item[\textbf{A[1]:}] \ldots então quer dizer que ele [o Camaro] tem a força de 400 cavalo professor?        
        \item[\textbf{A[2]:}] Meu já pensou, tipo o carro é uma fazenda de cavalo [Risos]
        \item[\textbf{Prof.:}] Tem que ver qual que é força do cavalo
        \item[\textbf{A[1]:}] Não sei
        \item[\textbf{Prof.:}] Nós já vamos ver quanto que vale essa unidade [cavalo-vapor], se vocês quiserem pesquisem ai quanto que um cavalo faz de força
    \end{itemize}        
    \end{minipage}
\end{center}
Esta dinâmica permite a formação das tríades I-R-F a medida que o professor promove a discussão em sala e mantém ativa a fala dos estudantes, retroalimentando o ciclo a cada nova interação.

Este padrão foi observado com menos intensidade na aula 03 em que o professor introduzia os conceitos de óptica geométrica por meio de uma sequência de slides, na ocasião o professor trouxe o exemplo da câmara escura por meio de um vídeo, e fazia questões como

\begin{center}
    \begin{minipage}{0.7\textwidth}
        \begin{itemize}
            \item[\textbf{Prof.:}] Todo mundo aqui já viu uma câmera fotográfica?
            \item[\textbf{Prof.:}] Como vocês imaginam que funciona uma câmera?
            \item[\textbf{A[1]:}] A luz passa pela lente
            \item[\textbf{Prof.:}] Ta, passa pela lente e aí acontece alguma coisa
        \end{itemize}
    \end{minipage}
\end{center}

 Neste episódio as respostas eram mais tímidas e diretas, a abordagem comunicativa não evoluiu para padrões mais complexos. Estima-se que pela complexidade do assunto, os alunos tenham se sentido menos a vontade. Ao final o professor propôs uma atividade investigativa visando a construção de uma câmara escura, falaremos um pouco mais desta atividade na sequência.

 \section{Atividade Investigativa}
 Imediatamente após a aula de introdução do conteúdo relacionado aos \emph{Princípio da Óptica Geométrica}, o professor propôs como atividade investigativa, que os alunos pesquisassem e construíssem uma câmara escura, tiveram 20 dias para completar a tarefa e puderam utilizar o que quisessem de material.

 O professor colocou-se à disposição para orientar no que for necessário.

 Não foi possível acompanhar o desenvolvimento desta atividade uma vez que, trata-se de uma turma ao qual é administrada aulas faixas às sextas-feiras e em decorrência da reunião administrativa marcada para o dia 23/09 e da eleição de primeiro turno para a presidência, marcada para o dia 02/10, houve o interrompimento destas aulas durante as duas sextas-feiras seguintes. Portanto, nos limitaremos a analisar somente os resultados da atividade.

 Embora a sala possua 43 alunos matriculados apenas 44\% comparecem frequentemente às aulas, pelo menos nas sextas-feiras, o que reduz a quantidade efetiva de estudantes para $\sim19$ indivíduos. Estes se dividiram em grupos de 3-4 integrantes, 2 grupos apresentaram seus trabalhos no dia combinado e outros 4 ficaram de entregar na próxima aula. Dos que apresentaram, relataram que pesquisaram na internet vendo vídeos como os do \emph{Manual do Mundo} e não sentiram dificuldades na confecção das câmaras.

 Este tipo de abordagem encontra-se bem fundamentada em \cite{Sasseron2015}, no que se convencionou chamar de \emph{Ensino por Investigação}, tem por característica fundamental o de \emph{``promover o papel ativo do estudante na construção do conhecimento científico'' \Ibidem{Sasseron2015}}. Também está de acordo com as orientações da nova \ac{BNCC} no que tange
 
 \begin{citacao}
     ``\ldots garantir o protagonismo dos estudantes em sua aprendizagem e o desenvolvimento de suas capacidades de abstração, reflexão, interpretação, proposição e ação, essenciais à sua autonomia pessoal, profissional, intelectual e política.'' \cite{BRASIL:2017}
 \end{citacao}

 Ainda assim, notou-se uma certa resistência a adesão da atividade pela turma como um todo, parte desta dificuldade deve justificar-se se considerado for o perfil dos estudantes: alunos quietos; introspectivos; boa parte trabalha durante o dia e estuda a noite e etc.

 \section{Aulas de Exercícios}
 Em todas as aulas deste tipo de abordagem, observou-se o mesmo padrão, o professor indica os exercícios que devem ser entregues e a turma por sua vez limita-se a resolvê-los em duplas, trios e etc. Durante a resolução alguns buscam o auxílio do professor e o mesmo se prontifica a ajudar. A maioria dos alunos resolvem e entregam estes exercícios o que torna a atividade a principal fonte de avaliações do professor.

 Os exercícios propostos são em poucos números, numa lista encontram-se no máximo um total de cinco exercícios. Em geral são exercícios fechados e de aplicação direta o que, segundo as pesquisas, permite pouca ou nenhuma mudança conceitual no aluno \cite{CLEMENT:2012,PEREZ1992,Carvalho1999}.

 É importante ressaltar que tais críticas estão fundadas sobre as formas como tradicionalmente vem se aplicando este tipo de abordagem e não sobre a atividade em si. Problemas e Questões de naturezas fechadas não o são essencialmente ruins, desde que tratados de forma oportuna, e complementar as demais Dimensões do Conteúdo previamente trabalhados em outros momentos \cite{PEREZ1992}, talvez seja neste sentido que se observou estas atividades em conjunto com a atividade investigativa. De todo modo, estas atividades auxiliam na composição da nota dos alunos, principalmente dos que não adeririam aos outros tipos de atividades, configurando-se também como um recurso estratégico do docente.