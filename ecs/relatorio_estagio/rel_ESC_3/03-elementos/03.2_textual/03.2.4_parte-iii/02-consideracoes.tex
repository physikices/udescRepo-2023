\chapter{Considerações Finais}
O ano de 2022 apresentou-se como um ano atípico para educação, se por um lado temos a flexibilização dos dispositivos de controle da Pandemia da COVID-19, crise esta que assolou o planeta ao longo de dois anos, acarretando em mudanças comportamentais drásticas na relação do seu humano com o seu entorno, por outro tem-se a implementação do novo currículo para o ensino médio na forma disposta pela \ac{BNCC}. Estes dois panoramas formaram o plano de fundo a que esteve presente todo o desenvolvimento deste trabalho e não há como negar suas devidas influências.

Ao longo deste trabalho também foi possível observar de que forma tem se estabelecido a adequação dos currículos de modo a atender as disposições previstas pela Legislação. Viu-se que apesar das dificuldades inerentes ao processo, há alguns esforços nesta direção. A análise do planejamento curricular reflete alguns destes esforços, pois percebeu-se abordagens trazidas na literatura como precursoras de um ensino significativo e modernizado, tais como: O Ensino por Investigação, Alinhamento às Dimensões do Conteúdo, Abordagens Discursivas, tópicos de Física Moderna começam a protagonizar presença.

Por fim este primeiro contato com a docência mais intensa do que em estágios anteriores, somado ao desenvolvimento do acadêmico no que tange o reconhecimento das modernas tecnologias de ensino bem como o conhecimento de abordagens diferenciadas, desacortinou uma possível dificuldade e insegurança de dirigir tais abordagens, no contexto de uma sala de aula que, à priori apresente alguma resistência a este tipo de mudança, o que requer um melhor preparo e atenção no trato de situações desta natureza, além do mais, o tempo de preparação, roteirização e planejamento deve ser fortemente considerado neste momento em que ainda está se consolidando a prática docente de forma crítica e reflexiva.