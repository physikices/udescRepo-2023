\chapter{Regências}
\label{cap:regencias}
% \textcolor{gray}{%
%     \begin{citacao}
%         ``\lettrine{A}{quele}\emph{ que já se viu obrigado a subir uma escada rolante quebrada, sabe o desconforto que esta simples situação representa, este desconforto não reside no fato de ter que despender energia para vencer o campo gravitacional nos poucos degraus acima, o ponto é que nosso cérebro, acostumado a passar por estas situações de modo automático, nos prega uma peça, causando-nos a ilusão de que a cada passo dado a escada esta se movendo ou para cima ou para baixo, com os degraus variando de altura a medida que nos deslocamos. O efeito disso é um caminhar meio desengonçado, digno de uma vinheta musical conduzida por uma orquestra de elefantes trompetistas.}'' -- \textbf{o autor}
%     \end{citacao}
% }



\noindent O dia 16 de setembro marcou o início da contagem para a primeira e as demais aulas, reservadas exclusivamente ao exercício da prática supervisionada de estágio tal como prevista no programa. Neste dia o supervisor em reunião com o estagiário, elegeram em conjunto os assuntos que foram administrado e o calendário de ocorrência das aulas.

O acordou predeterminou o início das aulas na segunda semana de outubro, mais especificamente, no dia 14 onde foi dado seguimento ao conteúdo de \emph{Óptica Geométrica} antes inciado pelo supervisor, e tendo por previsão de término o dia 11 de novembro. As aulas ocorreram todas as sextas-feiras no período noturno das 20:00 às 21:20 e somente na única turma de segundo ano do período o 2º(5).

A turma é caracterizada por conter alunos quietos e pouco participativos. É uma turma comportada, do ponto de vista escolar, ou seja, não são bagunceiros nem indisciplinados ao ponto de exigir alguma cautela no trato por parte do professor. Os indivíduos enquadram-se numa faixa etária média entre 16-17 anos de idade, alguns alegam conciliar os estudos com o trabalho e que por isso não conseguem chegar no exato horário de início das aulas. Como já dito anteriormente 44\% da turma não comparecem às aulas de sexta-feira com frequência, em relatos de outros professores esta proporção é um pouco melhor em outros dias da semana, não se sabe exatamente o quanto mas estima-se que não é tão diferente.

Assim fica delimitado o plano de fundo ao qual ocorreram estas aulas, passaremos a sequência da escolha da abordagem, planejamento das atividades e as aulas propriamente ditas.

\section{Bases Metodológicas}
A fim de orientar nossas práticas e alicerçar bases ao que nos foi devidamente delegado, pesquisou-se por materiais que nos auxiliassem na tarefa de elaboração e adequação das aulas, optou-se por limitar o escopo da pesquisa somente à Produtos Educacionais de universidades devidamente reconhecidas pelo \ac{MEC}.

Duas teses de mestrado, com seus respectivos produtos educacionais, foram escolhidas para leitura e análise, a primeira delas foi encontrada no portal eletrônico de teses e dissertações do \ac{MNPEF} e trata-se de \emph{``uma sequência didática composta de dez aulas para trabalhar as interpretações da óptica da visão''} \cite{JAQUELINE:2019}, segundo consta, o produto educacional foi testado e aplicado, o que \emph{``possibilitou delinear estratégias pedagógicas diversificadas e atrativas aos educandos, centradas no diálogo e no trabalho coletivo''} \Ibidem[p.~2]{JAQUELINE:2019}. 

O segundo material, que também está sob a tutela do \ac{MNPEF}, foi encontrado no repositório institucional da \ac{UFF}, este tem por objetivo a aplicação de \emph{``sequências didáticas relacionadas ao ensino da óptica na modalidade Centro de
Estudos de Jovens e Adultos''} \cite{AGNALDO:2018}.

Estas propostas não apenas estão alinhadas no intuito de promover um ensino focado na participação efetiva do estudante, como também são complementares do ponto de vista operacional. Enquanto uma visa explorar os fenômenos ópticos por intermédio da experimentação real, a outra, em síntese, promove o mesmo por meio da simulação com o \emph{PheT}, dessa forma o conhecimento científico que por vezes fica oculto em virtude das limitações inerentes ao processo de experimentação, pode ser explorado sem muita dificuldade na simulação. Uma mescla das duas propostas parece promissora e portanto planejou-se utilizar-se de ambas.

\section{Planejamento}
De acordo com o que foi combinado entre o supervisor e o estagiário, oito aulas foram destinadas para a imersão do estagiário nas atividades de regência, e com base nos referenciais adotados na seção anterior, planejou-se o primeiro cronograma de aulas de que trata o \autoref{qua:cronograma-regencias-01} da página \pageref{qua:cronograma-regencias-01}.
\vspace{15pt}
\begin{quadro}[!ht]
    \centering
    \resizebox{.8\textwidth}{!}{%
        \begin{tabular}{|c|c|l|c|}
            \hline
            \textbf{Aula} & \textbf{Data} & \multicolumn{1}{c|}{\textbf{Assunto}} & \textbf{Recursos}                            \\ \hline
            \textbf{01}   & 14/10         & Índice de Refração                  & \multicolumn{1}{l|}{Experimentos Demostrativos}             \\ \hline
            \textbf{02}   & 14/10         & Reflexão \& Refração                & \multicolumn{1}{l|}{Simulação \emph{PheT}} \\ \hline
            \textbf{03} & 21/10 & Lei de Snell-Descartes & -- \\ \hline
            \textbf{04} & 21/10 & Espelho Plano          & -- \\ \hline
            \textbf{05} & 04/11 & Lentes Convergentes    & -- \\ \hline
            \textbf{06} & 04/11 & Lentes Divergentes     & -- \\ \hline
            \textbf{07} & 11/11 & Instrumentos Óticos    & -- \\ \hline
            \textbf{08} & 11/11 & Óptica da Visão        & -- \\ \hline
        \end{tabular}%
    }    
    \caption{Cronograma de aulas -- I}
    \label{qua:cronograma-regencias-01}
\end{quadro}
\vspace{15pt}

A proposta assim delineada, anseia escolher a metodologia a ser aplicada a medida que vá se evoluindo no decorrer das atividades, sempre dosando entre atividades experimentais e simulações conforme a turma apresente melhor resposta a um ou ao outro e, é claro, a depender da complexidade do que de fato se queira tratar na respectiva atividade.


\section{Aulas}

As duas primeiras aulas teve por objetivo apresentar e discutir os dois fenômenos luminosos mais elementares à qualquer curso de óptica que se preze, o fenômeno da \emph{reflexão} e o fenômeno da \emph{refração}, ambos  decorrentes da propagação da luz em algum meio físico.

\subsection{Recursos Didáticos}
De início, planejou-se seguir o roteiro de atividades descrito na \autoref{anx:invisibilidade-refracao}, no entanto, teve-se de adaptar o experimento para uma versão simplificada, isto é, sem a glicerina e sem o bastão de vidro, uma vez que não foi possível encontrar tais materiais no espaço de tempo decorrido entre a pesquisa, o planejamento e a montagem da atividade. Por se tratar de uma atividade relativamente simples e de observação, também não necessitou disponibilizar o roteiro para a turma, ficando a cargo do estagiário a tarefa de apenas orientar as observações a que se deve fazer.

O experimento em si consiste de observar a descontinuidade da luz ao passar entre os meio ópticos ar-água, ar-óleo e ar-água-óleo, utilizou-se para tanto um canudo em um copo contendo algum dos materiais e para ampliar os efeitos do fenômeno, fez-se permutações e combinações entre os meios ópticos referidos.

Ainda como recurso, manteve-se aberta na lousa digital a simulação do \emph{PheT} de que trata estes fenômenos para que em momento oportuno, seja possível fazer-se o tratamento e a análise dos fenômenos com o auxílio do simulador.

\subsection{Metodologia}
A metodologia aqui desenvolvida, sofre influência direta dos dois Produtos Educacionais tomados por base, cujos os quais o primeiro encontra-se fundamento na teoria de Vygostsky, ao pressupor uma relação de cooperação, respeito e crescimento em que o aluno deve ser considerado como sujeito interativo no seu processo de internalização do conhecimento \cite{JAQUELINE:2019}, já o segundo baseia-se na teoria da Aprendizagem Significativa de David Ausubel em que, segundo o material de consulta \cite{AGNALDO:2018}, é fundamental promover o engajamento do estudante no processo de aprendizagem, além de possibilitar a revisão e aplicação de conceitos inerentes ao processo.

Com isto em mente, os dois primeiros planos de aula foram elaborados na tentativa de atender a tais recomendações e podem ser acessados no \autoref{ape:planos-de-aulas}, destinado a este tipo de conteúdo.

\subsection{Resultados}
A aplicação da proposta, revelou-se extremamente desafiadora, de forma que não obteve-se o resultado esperado já nos primeiros minutos de aula. Na tentativa de promover o devido engajamento da turma, levantou-se questões simples como:

\begin{center}
    \begin{minipage}{.7\textwidth}
        \begin{itemize}
            \item[\textbf{Estagiário:}] Como nós enxergamos?
            \item[\textbf{Alunos:}] \ldots
            \item[\textbf{Estagiário:}] O que é fundamental para que possamos ver algo?
            \item[\textbf{Alunos:}] \ldots
            \item[\textbf{Estagiário:}] Será que somente a visão é necessária para que possamos enxergar alguma coisa?    \item[\textbf{Alunos:}] \ldots      
        \end{itemize}
    \end{minipage}
\end{center}

Não observando qualquer tentativa de solução advindas dos docentes, procedeu-se respondendo-as de maneira explicativa e exemplificada a fim de dar continuidade à atividade.

 Ao apresentar o experimento demonstrativo, alguns alunos responderam verbalmente aos questionamentos do professor, não sem antes manter-se um bom nível de persistência e reformulações. Uma melhora não muito significativa, foi observada ao passar para a parte das simulações, onde um aluno de prontidão respondeu corretamente ao questionamento dos motivos reais pelos quais ocorrem os desvios da luz na mudança entre meios, porém quando convidado a expor mais detalhadamente o seu ponto de vista optou por não responder.

 Dado o que aqui se expôs, optou-se por modificar a abordagem inicial de maneira que não exigisse em excesso a participação dos estudantes, haja visto o perfil desta turma. Tal abordagem deve atender a critérios que permitam ao estagiário um domínio maior de promover a participação conforme os estudantes vão se acostumando mais com este tipo de condução e com a figura do estagiário na qualidade de professor temporário da turma, além disso, deve-se ater também ao tempo disponível para a preparação, pesquisa e replanejamento das próximas seis aulas, bem como da disponibilidade de recursos a que se pode prover.

 Uma abordagem de que se tem conhecimento do potencial para comportar estas exigências, é a abordagem do tema a partir do viés da \ac{HFC}. Assim, após informar ao orientador e supervisor de curso desta pretensão, procedeu-se com a etapa de pesquisas e recondicionamento da proposta o que culminou na atualização do cronograma como consta no \autoref{qua:cronograma-regencias-02} 

\vspace{15pt}
\begin{quadro}[!ht]
    \centering
    \resizebox{1\textwidth}{!}{%
        \begin{tabular}{|c|c|l|l|}
            \hline
            \textbf{Aula} & \textbf{Data} & \multicolumn{1}{c|}{\textbf{Assunto}} & \multicolumn{1}{c|}{\textbf{Recursos}} \\ \hline            
            \textbf{03} & 21/10 & Teoria Corpuscular da Luz                & Exposição              \\ \hline
            \textbf{04} & 21/10 & Ondas I                                  & Texto + Dedução        \\ \hline
            \textbf{05} & 04/11 & Ondas II - Princípio de Huygens          & Simulação              \\ \hline
            \textbf{06} & 04/11 & Teoria Ondulatória da Luz - Lei de Snell & Dedução                \\ \hline
            \textbf{07} & 11/11 & Fenômenos Ondulatórios                   & Exposição + Vídeos     \\ \hline
            \textbf{08} & 11/11 & Éter Luminífero                          & Exposição              \\ \hline
        \end{tabular}%
    }
    \caption{Cronograma de aulas - II}
    \label{qua:cronograma-regencias-02}
\end{quadro}
\vspace{15pt}

Nesta configuração, as aulas de deduções, exposições moderadamente dialogadas, vídeos e simulações foram priorizadas. Não houve um único Produto Educacional prontamente construído de que se pudesse fazer uso, sendo assim baseou-se nos estudos desenvolvido por diversos autores, dentre eles constam: \cite{FABIO:2009,FELIPHE:2019,PEDUZZI:2022,FORATO:2011,FAOBIO:2007}.

O cerne da abordagem reproduziu em sala de aula a célebre disputa entre os defensores da natureza da luz dentro do modelo corpuscular em confronto direto com os defensores do modelo ondulatório, e investigou-se os argumentos a favor de um e de outro, também possibilitou a preparação do plano de fundo para a retomada do professor supervisor que se deu pelo temática \emph{Física Moderna}.

Um texto de apoio aos estudantes foi desenvolvido, para acompanhamento das aulas de dedução teórica, o referido texto foi baseado em \cite{NUSSENZVEIG41997,HALLIDAY42008,HALLIDAY22008,ZEMANSKY42010,NUSSENZVEIG31997,PEDUZZI:2022} e encontra-se no \autoref{ape:texto-de-apoio}.

%\cite{NUSSENZVEIG41997,HALLIDAY42008,HALLIDAY22008,ZEMANSKY42010,NUSSENZVEIG31997,PEDUZZI:2022}